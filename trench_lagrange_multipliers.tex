\documentclass{article}
\usepackage[T1]{fontenc}
\usepackage{textcomp}
\renewcommand{\rmdefault}{ptm}
\usepackage[scaled=0.92]{helvet}
%\usepackage[slantedGreek,zswash,amsbb,mtpcal]{mtpro2}
\usepackage[psamsfonts]{amsfonts}
\usepackage{amsmath, amsbsy,verbatim}
\usepackage[dvips, bookmarks, colorlinks=true, plainpages = false,
  citecolor = blue, urlcolor = blue, filecolor = blue]{hyperref}
\newtheorem{corollary}{Corollary}
\newtheorem{definition}{Definition}
\newtheorem{lemma}{Lemma}
\newtheorem{theorem}{Theorem}
\newtheorem{example}{Example}
\newcommand{\proof}{\noindent{\sc\bf Proof}\quad }
\def\endproof{{\hfill \vbox{\hrule\hbox{%
 \vrule height1.3ex\hskip0.8ex\vrule}\hrule }}\par}
\newcommand{\bbox}{\phantom{1}\hfill{\rule{6pt}{6pt}}}
\newcommand{\pd}[2]{\frac{\partial{#1}}{\partial{#2}}}
\newcommand{\dst}{\displaystyle}
\newcommand{\place}{\bigskip\hrule\bigskip\noindent}
\newcommand{\set}[2]{\left\{#1\, \big|\, #2\right\}}
\newcommand{\solution}{\medskip\noindent{\bf Solution}\hskip.2em}
\newcommand{\exer}[1]{\par\noindent{\bf #1}.}
\newcommand{\boxit}[1]{\bigskip\noindent{\bf
#1}\\\vskip-6pt\hskip-\parindent}

\newcounter{lcal}
\newenvironment{alist}{\begin{list}{\bf (\alph{lcal})}
{\topsep 0pt\partopsep 0pt\labelwidth 14pt
\labelsep 8pt\leftmargin 22pt\itemsep 0pt
\usecounter{lcal}}}{\end{list}}

\newcounter{exercise}
\newenvironment{exerciselist}{\begin{list}{\bf \arabic{exercise}.}
{\topsep 10pt\partopsep 0pt\labelwidth 16pt
\labelsep 12pt\leftmargin 28pt
\itemsep 8pt\usecounter{exercise}}}{\end{list}}
\begin{document}

\noindent

\thispagestyle{empty}
\bf
\begin{center}
{\Huge THE METHOD OF\\\vspace{.2in} LAGRANGE MULTIPLIERS}
\vspace{.5in}
\huge
\bigskip
\vspace{.75in}
\bf\huge
\href{http://ramanujan.math.trinity.edu/wtrench/index.shtml}
{William F. Trench}
\medskip
\\\large
Andrew G. Cowles Distinguished Professor Emeritus\\
Department of Mathematics\\
Trinity University \\
San Antonio, Texas, USA\\
\href{mailto:{wtrench@trinity.edu}}
{wtrench@trinity.edu}
\large
\vspace*{.75in}
\end{center}


\rm
\noindent

\noindent
This is a supplement to the author's
\href{http://ramanujan.math.trinity.edu/wtrench/texts/TRENCH_REAL_ANALYSIS.PDF}
{\large Introduction to Real Analysis}.
It   has been judged to meet the evaluation criteria set by the
Editorial  Board
of the American Institute of Mathematics in connection with the Institute's
\href{http://www.aimath.org/textbooks/}
{Open
Textbook Initiative}.
It may be copied, modified, redistributed, translated,  and
built upon  subject to the Creative
Commons
      \href{http://creativecommons.org/licenses/by-nc-sa/3.0/deed.en_G}
{Attribution-NonCommercial-ShareAlike 3.0 Unported License}.
A complete instructor's solution manual is available by email to
\href{mailto:wtrench@trinity.edu}
{wtrench@trinity.edu},
 subject to verification of the requestor's
faculty status.

\newpage

\centerline{\bf THE METHOD OF LAGRANGE MULTIPLIERS}
\medskip
\medskip

\centerline{\bf William F. Trench}
\medskip

\section{Foreword} \label{section:1}
This is a revised and extended version of Section~6.5 of my
\emph{Advanced Calculus}
(Harper \& Row, 1978).
It is a supplement to my textbook
\href{http://ramanujan.math.trinity.edu/wtrench/texts/TRENCH_REAL_ANALYSIS.PDF}
{\emph{Introduction to Real Analysis}}, which
is referenced via hypertext links.

\section{Introduction} \label{section:2}
To avoid repetition, it is to be understood  throughout   that
 $f$ and $g_{1}$, $g_{2}$,\dots,
$g_{m}$ are continuously differentiable  on
 an open set $D$ in $\mathbb{R}^{n}$.

Suppose   that $m<n$   and
\begin{equation} \label{eq:1}
g_{1}(\mathbf{X}) = g_2(\mathbf{X}) = \cdots = g_{m}(\mathbf{X})=0
\end{equation}
on a  nonempty subset $D_{1}$ of $D$.
  If\, $\mathbf{X}_{0} \in
D_{1}$ and there is a neighborhood $N$ of $\mathbf{X}_{0}$ such that
\begin{equation} \label{eq:2}
f(\mathbf{X}) \le f(\mathbf{X}_{0})
\end{equation}
for every $\mathbf{X}$ in $N \cap D_{1}$,  then $\mathbf{X}_{0}$
 is \emph{a local
maximum point of $f$ subject to the constraints} \eqref{eq:1}.
However, we will usually say ``subject
to'' rather than ``subject to the constraint(s).''

 If \eqref{eq:2} is replaced
by
\begin{equation} \label{eq:3}
f(\mathbf{X}) \ge f(\mathbf{X}_{0}),
\end{equation}
then ``maximum'' is replaced by ``minimum.''  A local maximum or minimum of
$f$ subject to \eqref{eq:1} is also called a {\it local extreme point of
$f$
subject to} \eqref{eq:1}.  More briefly, we also speak of {\it constrained
local maximum, minimum, or extreme points}.  If \eqref{eq:2} or
\eqref{eq:3} holds for all
$\mathbf{X}$ in $D_{1}$, we omit  ``local.''

Recall that
${\bf X}_{0}=(x_{10}, x_{20},\dots,x_{n0})$
 is a \emph{critical
point} of a differentiable function
$L=L(x_{1},x_{2},\dots,x_{n})$ if
$$
L_{x_{i}}(x_{10},x_{20},\dots,x_{n0})=0,\quad 1\le i\le n.
$$
Therefore, every local extreme point of $L$ is a critical point of $L$;
however, a critical point  of $L$  is not necessarily
a   local extreme point of $L$
\href{http://ramanujan.math.trinity.edu/wtrench/texts/TRENCH_REAL_ANALYSIS.PDF}
{(pp.~334-5)}.

Suppose that the
 system \eqref{eq:1} of simultaneous equations  can be
solved for
  $x_{1}$,  \dots, $x_{m}$
in terms of the  $x_{m+1}$, \dots, $x_{n}$; thus,
\begin{equation} \label{eq:4}
x_{j}=h_{j}(x_{m+1},\dots,x_{n}),\quad 1\le j\le m.
\end{equation}
Then a constrained extreme value  of $f$
 is an unconstrained extreme value of
\begin{equation} \label{eq:5}
f(h_{1}(x_{m+1},\dots,x_{n}),\dots,h_{m}(x_{m+1},\dots,x_{n}),x_{m+1},\dots,x_{n}).
\end{equation}
However, it may be  difficult or  impossible to find
explicit formulas for $h_{1}$, $h_{2}$, \dots, $h_{m}$,
 and,  even if it is
possible, the composite function \eqref{eq:5} is almost always
complicated.  Fortunately, there is a better way to to find constrained
extrema, which also requires the solvability assumption,  but does
not require an explicit formula as indicated in  \eqref{eq:4}.   It is
based on
the following theorem. Since the proof is complicated, we  consider two
special cases first.


\begin{theorem} \label{theorem:1}
Suppose that  $n>m.$    If\,  ${\bf X}_{0}$ is a local  extreme point of
$f$
subject to
$$
g_{1}({\bf X})=g_{2}({\bf X})=\cdots =g_{m}({\bf X})=0
$$
and
\begin{equation} \label{eq:6}
\left|\begin{array}{ccccccc}
\dst{\pd{g_{1}(\mathbf{X}_{0})}{x_{r_{1}}}} &
\dst{\pd{g_{1}(\mathbf{X}_{0})}{x_{r_{2}}}}&
&\cdots &
\dst{\pd{g_{1}(\mathbf{X}_{0})}{x_{r_{m}}}} \\\\
\dst{\pd{g_{2}(\mathbf{X}_{0})}{x_{r_{1}}}} &
\dst{\pd{g_{2}(\mathbf{X}_{0})}{x_{r_{2}}}}&
&\cdots &
\dst{\pd{g_{m}(\mathbf{X}_{0})}{x_{r_{m}}}} & \\
 \vdots & \vdots &&\ddots&\vdots\\
\dst{\pd{g_{m}(\mathbf{X}_{0})}{x_{r_{1}}}} &
\dst{\pd{g_{m}(\mathbf{X}_{0})}{x_{r_{2}}}}&
&\cdots &
\dst{\pd{g_{m}(\mathbf{X}_{0})}{x_{r_{m}}}} &
\end{array}\right|\ne0
\end{equation}
for at least one choice of
$r_{1}<r_{2}<\dots <r_{m}$ in $\{1,2,\dots,n\},$ then there are constants
$\lambda_{1},$ $\lambda_{2},$ \dots$,$ $\lambda_{m}$ such that
${\bf X}_{0}$
is a critical point of
$$
f-\lambda_{1}g_{1}-\lambda_{2}g_{2}-\cdots-\lambda_{m} g_{m};
$$
that is$,$
$$
\frac{\partial{f({\bf X}_{0})}}{\partial x_{i}}
-\lambda_{1}\frac{\partial{g_{1}({\bf X}_{0})}}{\partial x_{i}}
-\lambda_{2}\frac{\partial{g_{2}({\bf X}_{0})}}{\partial x_{i}}-\cdots
-\lambda_{m}\frac{\partial{g_{m}({\bf X}_{0})}}{\partial x_{i}}=0,
$$
$1\le i\le n$.
\end{theorem}

The following implementation of this theorem is  the \emph{method of
\href{http://www-history.mcs.st-and.ac.uk/Mathematicians/Lagrange.html}
{Lagrange} \\multipliers}.



\medskip
\begin{alist}
\item  Find the critical points of
$$
f-\lambda_{1}g_{1}-\lambda_{2}g_{2}-\cdots-\lambda_{m} g_{m},
$$
treating $\lambda_{1}$, $\lambda_{2}$, \dots $\lambda_{m}$ as
unspecified constants.

\item Find $\lambda_{1}$, $\lambda_{2}$, \dots, $\lambda_{m}$ so that the
critical points obtained in (a) satisfy the constraints.

\item  Determine which of the critical points are constrained
extreme points of  $f$. This can usually be done
by physical or intuitive arguments.
\end{alist}

\medskip

If $a$ and $b_{1}$, $b_{2}$, \dots, $b_{m}$ are  nonzero constants and $c$
is an arbitrary
constant, then   the local extreme points of  $f$ subject to
 $g_{1}=g_{2}= \cdots =g_{m}=0$  are the same as the
local extreme points of
$af-c$ subject to
$b_{1}g_{1}=b_{2}g_{2}=\cdots=b_{m}g_{m}=0$. Therefore,
we can replace
$f-\lambda_{1} g_{1}-\lambda_{2}g_{2}- \cdots-\lambda_{m} g_{m}$ by
$af-\lambda_{1}b_{1}g_{1}-\lambda_{2}b_{2}g_{2}- \cdots-
\lambda_{m}b_{m}g_{m}-c$ to simplify  computations.
 (Usually, the ``$-c$'' indicates dropping
additive constants.)
 We  will denote the final form by $L$ (for
\emph{Lagrangian}).

\section{Extrema subject to one constraint}  \label{section:3}

Here is Theorem~\ref{theorem:1} with $m=1$.

\begin{theorem} \label{theorem:2}
Suppose that  $n>1.$ If\,  ${\bf X}_{0}$ is a local extreme point of $f$
subject
to
$g({\bf
X})=0$ and $g_{x_{r}}({\bf X}_{0})\ne0$ for some $r\in\{1,2,\dots,n\},$
then there is a constant $\lambda$ such that
\begin{equation} \label{eq:7}
f_{x_{i}}({\bf X}_{0})-\lambda
g_{x_{i}}({\bf X}_{0})=0,\quad
\end{equation}
$1\le i\le n;$
thus$,$ ${\bf X}_{0}$ is a critical point of $f-\lambda g.$
\end{theorem}

\proof
For notational convenience, let $r=1$ and   denote
$$
{\bf U}=(x_{2},x_{3},\dots x_{n})\text{\; and\;\;}
{\bf U}_{0}=(x_{20},x_{30},\dots x_{n0}).
$$
Since $g_{x_{1}}({\bf X}_{0})\ne0$,  the
Implicit Function Theorem
\href{http://ramanujan.math.trinity.edu/wtrench/texts/TRENCH_REAL_ANALYSIS.PDF}
{(Corollary 6.4.2, p.~423)}
implies
that there is a unique continuously differentiable function
$h=h({\bf U}),$ defined on a neighborhood $N \subset{\mathbb R}^{n-1}$ of
${\bf U}_{0},$  such that $(h({\bf U}),{\bf U})\in D$ for all
${\bf U}\in N$,
 $h({\bf U}_{0})=x_{10}$, and
\begin{equation} \label{eq:8}
g(h({\bf U}),{\bf U})=0,\quad  {\bf U}\in N.
\end{equation}
Now define
\begin{equation} \label{eq:9}
\lambda=\frac{f_{x_{1}}({\bf X}_{0})}{g_{x_{1}}({\bf X}_{0})},
\end{equation}
which is permissible, since  $g_{x_{1}}({\bf X}_{0})\ne0$.
This implies \eqref{eq:7} with $i=1$.
If   $i> 1$,  differentiating \eqref{eq:8} with respect to $x_{i}$ yields
\begin{equation} \label{eq:10}
\frac{\partial g(h({\bf U}),{\bf U})}{{\partial  x_{i}}}+
\frac{\partial g(h({\bf U}),{\bf U})}{{\partial  x_{1}}}
\frac{{\partial h({\bf U})}}{{\partial x_{i}}}=0,\quad {\bf U}\in N.
\end{equation}
Also,
\begin{equation} \label{eq:11}
\frac{\partial f({h(\bf U}),{\bf U}))}{\partial x_{i}}=
\frac{\partial f(h({\bf U}),{\bf U})}{{\partial  x_{i}}}+
\frac{\partial f(h({\bf U}),{\bf U})}{{\partial  x_{1}}}
\frac{{\partial h({\bf U})}}{{\partial x_{i}}},\quad {\bf U}\in N.
\end{equation}
Since $(h({\bf U}_{0}),{\bf U}_{0})={\bf X}_{0}$, \eqref{eq:10}
implies that
\begin{equation} \label{eq:12}
\frac{\partial g({\bf X}_{0})}{{\partial  x_{i}}}+
\frac{\partial g({\bf X}_{0})}{{\partial  x_{1}}}
\frac{{\partial h({\bf U}_{0})}}{{\partial x_{i}}}=0.
\end{equation}
If\,
 ${\bf X}_{0}$ is a local extreme point of  $f$
subject to  $g({\bf X})=0$, then ${\bf U}_{0}$
is an unconstrained local extreme point of $f(h({\bf U}),{\bf U})$;
therefore,
\eqref{eq:11}  implies that
\begin{equation} \label{eq:13}
\frac{\partial f({\bf X}_{0})}{{\partial  x_{i}}}+
\frac{\partial f({\bf X}_{0})}{{\partial  x_{1}}}
\frac{{\partial h({\bf U}_{0})}}{{\partial x_{i}}}=0.
\end{equation}
Since a linear homogeneous system
$$
\left[\begin{array}{ccccccc} a&b\\c&d
\end{array}\right] \left[\begin{array}{ccccccc} u\\v \end{array}\right]=
\left[\begin{array}{ccccccc} 0\\0 \end{array}\right]
 $$
has a nontrivial
solution if and only if
$$ \left|\begin{array}{ccccccc} a&b\\c&d
\end{array}\right|=0,
 $$
\href{http://ramanujan.math.trinity.edu/wtrench/texts/TRENCH_REAL_ANALYSIS.PDF}
{(Theorem~6.1.15, p.~376)},
 \eqref{eq:12} and
\eqref{eq:13} imply that
$$
\left|\begin{array}{ccccccc}
\dst{\frac{{\partial f({\bf X}_{0})}}{{\partial x_{i}}}}&
\dst{\frac{{\partial f({\bf X}_{0})}}{{\partial x_{1}}}}\\\\
\dst{\frac{{\partial g({\bf X}_{0})}}{{\partial x_{i}}}}&
\dst{\frac{{\partial g({\bf X}_{0})}}{{\partial x_{1}}}}&
\end{array}\right|=0,\text{\; so\;\;} \left|\begin{array}{ccccccc}
\dst{\frac{{\partial f({\bf X}_{0})}}{{\partial x_{i}}}}&
\dst{\frac{{\partial g({\bf X}_{0})}}{{\partial x_{i}}}}\\\\
\dst{\frac{{\partial f({\bf X}_{0})}}{{\partial x_{1}}}}&
\dst{\frac{{\partial g({\bf X}_{0})}}{{\partial x_{1}}}}
\end{array}\right|=0, $$
since the determinants of a matrix and its transpose are equal.
 Therefore, the system
$$
 \left[\begin{array}{ccccccc}
\dst{\frac{{\partial f({\bf X}_{0})}}{{\partial x_{i}}}}&
\dst{\frac{{\partial g({\bf X}_{0})}}{{\partial x_{i}}}}\\\\
\dst{\frac{{\partial f({\bf X}_{0})}}{{\partial x_{1}}}}&
\dst{\frac{{\partial g({\bf X}_{0})}}{{\partial x_{1}}}} \end{array}\right]
\left[\begin{array}{ccccccc} u\\v \end{array}\right]=
\left[\begin{array}{ccccccc} 0\\0 \end{array}\right] $$
has a nontrivial
solution
\href{http://ramanujan.math.trinity.edu/wtrench/index.shtml}
{(Theorem~6.1.15, p. 376)}.
 Since $g_{x_{1}}({\bf X}_{0})\ne0$, $u$ must be nonzero in a
nontrivial solution. Hence, we may assume that $u=1$, so
\begin{equation} \label{eq:14}
 \left[\begin{array}{ccccccc}
\dst{\frac{{\partial f({\bf X}_{0})}}{{\partial x_{i}}}}&
\dst{\frac{{\partial g({\bf X}_{0})}}{{\partial x_{i}}}}\\\\
\dst{\frac{{\partial f({\bf X}_{0})}}{{\partial x_{1}}}}&
\dst{\frac{{\partial g({\bf X}_{0})}}{{\partial x_{1}}}} \end{array}\right]
\left[\begin{array}{ccccccc} 1\\ v \end{array}\right]=
\left[\begin{array}{ccccccc} 0\\0 \end{array}\right].
\end{equation}
In particular,
$$
\frac{\partial f({\bf X}_{0})}{\partial x_{1}}+
v\frac{\partial g({\bf X}_{0})}{\partial x_{1}}=0, \text{\; so\;\;}
-v=\frac{f_{x_{1}}({\bf X}_{0})}{g_{x_{1}}({\bf X}_{0})}.
$$
Now \eqref{eq:9} implies that $-v=\lambda$, and  \eqref{eq:14}  becomes
$$
 \left[\begin{array}{ccccccc}
\dst{\frac{{\partial f({\bf X}_{0})}}{{\partial x_{i}}}}&
\dst{\frac{{\partial g({\bf X}_{0})}}{{\partial x_{i}}}}\\\\
\dst{\frac{{\partial f({\bf X}_{0})}}{{\partial x_{1}}}}&
\dst{\frac{{\partial g({\bf X}_{0})}}{{\partial x_{1}}}} \end{array}\right]
\left[\begin{array}{rcccccc} 1\\ -\lambda \end{array}\right]=
\left[\begin{array}{ccccccc} 0\\0 \end{array}\right].
$$
Computing the topmost entry of the vector on the left yields
\eqref{eq:7}.
\hfill\bbox

\begin{example} \label{example:1}     \rm
Find the point $(x_{0},y_{0})$ on the line
$$
ax+by=d
$$
 closest to  a given point $(x_{1},y_{1})$.

\solution
We must minimize   $\sqrt{(x-x_{1})^{2}+(y-y_{1})^{2}}$ subject to the
 constraint. This is equivalent to  minimizing
$(x-x_{1})^{2}+(y-y_{1})^{2}$ subject to the constraint, which is simpler.
For, this we could let
$$
L=(x-x_{1})^{2}+(y-y_{1})^{2}-\lambda (ax+by-d);
$$
however,
$$
L=\frac{(x-x_{1})^{2}+(y-y_{1})^{2}}{2}-\lambda (ax+by)
$$
is  better. Since
$$
L_{x}=x-x_{1}-\lambda a\text{\quad  and \quad }
L_{y}=y-y_{1}-\lambda b,
$$
 $(x_{0},y_{0})=(x_{1}+\lambda a, y_{1}+\lambda b)$,
where we must choose $\lambda$ so that $ax_{0}+by_{0}=d$. Therefore,
$$
ax_{0}+by_{0}=ax_{1}+by_{1}+\lambda(a^{2}+b^{2})=d,
$$
so
$$
\lambda= \frac{d-ax_{1}-by_{1}}{a^{2}+b^{2}},
$$
$$
x_{0}=x_{1}+\frac{(d-ax_{1}-by_{1})a}{a^{2}+b^{2}},
\text{\; and\;\;}
y_{0}=y_{1}+\frac{(d-ax_{1}-by_{1})b}{a^{2}+b^{2}}.
$$
The distance from $(x_{1},y_{1})$ to the line is
$$
\sqrt{(x_{0}-x_{1})^{2}+(y_{0}-y_{1})^{2}}=
\frac{|d-ax_{1}-by_{1}|}{\sqrt{a^{2}+b^{2}}}.
$$
\end{example}

\begin{example} \label{example:2}        \rm
Find the  extreme values of $f(x,y)=2x+y$  subject to
$$
x^{2}+y^{2}=4.
$$

\noindent
{\bf Solution}
Let
$$
L=2x+y-\frac{\lambda}{2}(x^{2}+y^{2});
$$
then
$$
L_{x}=2-\lambda x\text{\; and\;\;}  L_{y}=1-\lambda y,
$$
so
$(x_{0},y_{0})=(2/\lambda,1/\lambda)$. Since
$x_{0}^{2}+y_{0}^{2}=4$,  $\lambda=\pm \sqrt{5}/2$. Hence, the
constrained maximum is $2\sqrt{5}$, attained at $(4/\sqrt{5},2/\sqrt{5})$,
and the constrained minimum is
$-2\sqrt{5}$, attained at $(-4/\sqrt{5},-2/\sqrt{5})$.
\end{example}

\begin{example} \label{example:3}  \rm
Find the point in the plane
\begin{equation} \label{eq:15}
3x+4y+z=1
\end{equation}
closest to $(-1,1,1)$.

\medskip
\solution
We must minimize
$$
f(x,y,z)=(x+1)^{2}+(y-1)^{2}+(z-1)^{2}
$$
subject to \eqref{eq:15}.  Let
$$
L=\frac{(x+1)^{2}+(y-1)^{2}+(z-1)^{2}}{2}-\lambda(3x+4y+z);
$$
then
$$
L_{x}= x+1-3\lambda,\quad
L_{y}= y-1-4\lambda,\text{\; and\;\;}
L_{z}= z-1-\lambda,
$$
so
$$
x_{0}=-1+3\lambda,\quad y_{0}=1+4\lambda,\quad z_{0}=1+\lambda.
$$
From \eqref{eq:15},
$$
3(-1+3\lambda)+4(1+4\lambda)+(1+\lambda)-1=1+26\lambda=0,
$$
so
$\lambda=-1/26$  and
$$
(x_{0},y_{0},z_{0})=
\left(-\frac{29}{26},\frac{22}{26},\frac{25}{26}\right).
$$
The distance from $(x_{0},y_{0},z_{0})$ to $(-1,1,1)$  is
$$
\sqrt{(x_{0}+1)^{2}+(y_{0}-1)^{2}+(z_{0}-1)^{2}}=\frac{1}{\sqrt{26}}.
$$
\end{example}

\begin{example} \label{example:4}  \rm
Assume that $n\ge 2$ and $x_{i}\ge 0$, $1\le i\le n$.
\begin{alist}
\item % (a)
Find  the extreme values of
 $\dst{\sum_{i=1}^{n}x_{i}}$ subject to
$\dst{\sum_{i=1}^{n}x_{i}^{2}=1}$.

\item
Find  the
minimum  value of
$\dst{\sum_{i=1}^{n}x_{i}^{2}}$ subject to
$\dst{\sum_{i=1}^{n}x_{i}=1}$.
\end{alist}

\solution {\bf (a)}
Let
$$
L= \sum_{i=1}^{n}x_{i}-\frac{\lambda}{2}\sum_{i=1}^{n}x_{i}^{2};
$$
then
$$
L_{x_{i}}=1-\lambda x_{i}, \text{\; so\;\;} x_{i0}=\frac{1}{\lambda}, \quad
1\le i\le n.
$$
Hence, $\dst{\sum_{i=1}^{n}x_{i0}^{2}}=n/\lambda^{2}$, so
$\lambda=\pm\sqrt{n}$\, and
$$
(x_{10},x_{20},\dots,x_{n0})=
\pm\left(\frac{1}{\sqrt{n}},\frac{1}{\sqrt{n}}, \dots,
\frac{1}{\sqrt{n}}\right).
$$
Therefore, the constrained maximum is $\sqrt{n}$
and the constrained minimum is $-\sqrt{n}$.

\solution {\bf (b)}
Let
$$
L=\frac{1}{2}
\sum_{i=1}^{n}x_{i}^{2}-\lambda\sum_{i=1}^{n}x_{i};
$$
then
$$
L_{x_{i}}=x_{i}-\lambda, \text{\; so\;\;} x_{i0}=\lambda,\quad 1\le i\le n.
$$
Hence, $\dst{\sum_{i=1}^{n}x_{i0}}=n\lambda=1$, so
$x_{i0}=\lambda=1/n$
and  the constrained minimum is
$$
\dst{\sum_{i=1}^{n}x_{i0}^{2}}=\frac{1}{n}
$$
There is no constrained maximum. (Why?)
\end{example}

\begin{example} \label{example:5}   \rm
Show that
$$
x^{1/p}y^{1/q} \le \frac{x}{p}+\frac{y}{q}, \quad x,y \ge 0,
$$
if
\begin{equation} \label{eq:16}
\frac{1}{p} +\frac{1}{q} = 1, \quad p > 0, \text{\; and\;\;} q > 0.
\end{equation}

\solution
We first find the maximum of
$$
f(x,y) = x^{1/p}y^{1/q}
$$
subject to
\begin{equation} \label{eq:17}
\frac{x}{p}+\frac{y}{q} =
\sigma, \quad x \ge 0, \quad y \ge 0,
\end{equation}
where $\sigma$ is a fixed but arbitrary positive number.  Since $f$ is
continuous, it must
assume a maximum at some point $(x_{0},y_{0})$ on the line segment \eqref{eq:17}, and
$(x_{0},y_{0})$ cannot be an endpoint of the segment, since $f(p\sigma,0) =
f(0,q\sigma)=0$. Therefore, $(x_{0},y_{0})$ is in the open first quadrant.

Let
$$
L = x^{1/p}y^{1/q} -\lambda
\left(\frac{x}{p}+\frac{y}{q}\right).
$$
Then
$$
 L_x = \frac{1}{px} f(x,y) - \frac{\lambda}{p}
\text{\; and\;\;}
 L_y = \frac{1}{qy} f(x,y) - \frac{\lambda}{q}=0,
$$
so $x_{0} = y_{0}=f(x_{0},y_{0})/\lambda$. Now\eqref{eq:16} and
\eqref{eq:17} imply that
$x_{0} =y_{0} =
\sigma$.  Therefore,
$$
f(x,y) \le f(\sigma,\sigma) = \sigma^{1/p}\sigma^{1/q} =
\sigma=\frac{x}{p}+\frac{y}{q}.
$$
\end{example}

This  can be generalized (Exercise~\ref{exer:53}).
  It can also be used to
generalize
\href{http://www-history.mcs.st-and.ac.uk/Mathematicians/Schwarz.html}
{Schwarz's}
 inequality
(Exercise~\ref{exer:54}).

\section{Constrained Extrema of Quadratic Forms} \label{section:4}

In this section it is convenient to write
$$
{\bf X}=
\left[\begin{array}{ccccccc}
x_{1}\\x_{2}\\\vdots\\x_{n}
\end{array}\right].
$$

An {\it eigenvalue} of a square matrix $\mathbf{A} = [a_{ij}]_{i,j=1}^{n}$
is a number $\lambda$ such that the system
$$
\mathbf{A}\mathbf{X} = \lambda \mathbf{X},
$$
or, equivalently,
$$
(\mathbf{A}-\lambda \mathbf{I})\mathbf{X}=\mathbf{0},
$$
has a solution $\mathbf{X} \ne \mathbf{0}$. Such a solution is called
an {\it eigenvector} of $\mathbf{A}$. You probably know from
linear algebra  that
$\lambda$ is an eigenvalue of $\mathbf{A}$ if and only if
$$
\det(\mathbf{A} -\lambda \mathbf{I}) = {0}.
 $$

Henceforth we assume that
$\mathbf{A}$ is symmetric $(a_{ij} = a_{ji}, 1 \le i, j \le n)$. In this
case,
$$
\det(\mathbf{A}-\lambda \mathbf{I}) =
(-1)^{n}(\lambda-\lambda_{1})(\lambda-\lambda_2) \cdots
(\lambda-\lambda_{n}),
$$
where $\lambda_{1},\lambda_2,\dots,\lambda_{n}$ are real numbers.

The function
$$
Q(\mathbf{X}) = \sum^{n}_{i,j=1} a_{ij} x_{i}x_{j}
$$
is a \emph{quadratic form}.
To find its maximum or minimum
subject to
$\dst{\sum^{n}_{i=1} x^{2}_{i}=1}$,
we form the Lagrangian
$$
L=Q(\mathbf{X}) - \lambda \sum^{n}_{i=1}x^{2}_{i}.
$$
Then
$$
L_{x_{i}}= 2 \sum^{n}_{j=1} a_{ij}x_{j} - 2\lambda x_{i}=0,
\quad 1 \le i \le n,
$$
so
$$
\sum_{j=1}^{n}a_{ij}x_{j0}=\lambda x_{i0},\quad  1\le i\le n.
$$
Therefore,  $\mathbf{X_{0}}$ is a constrained critical
point of $Q$ subject to $\dst{\sum^{n}_{i=1} x^{2}_{i}=1}$ if and only if\,
${\mathbf A}{\mathbf X}_{0}=\lambda{\mathbf X}_{0}$ for some $\lambda$;
that
is, if and only if $\lambda$ is an eigenvalue and $\mathbf{X}_{0}$ is an
associated unit eigenvector of $\mathbf{A}$. If ${\mathbf
A}\mathbf{X}_{0}={\bf X}_{0}$ and
$\dst{\sum_{i}^{n}x_{i0}^{2}}=1$, then
\begin{eqnarray*}
Q(\mathbf{X}_{0}) & =& \sum^{n}_{i=1} \left(\sum^{n}_{j=1} a_{ij}x_{j0}
\right) x_{i0} =
\sum^{n}_{i=1} (\lambda x_{i0})x_{i0} \\
& =& \lambda \sum^{n}_{i=1} x^{2}_{i0} = \lambda;
\end{eqnarray*}
therefore, the largest and smallest eigenvalues of ${\bf A}$ are the
maximum
and minimum values of $Q$ subject to $\dst{\sum_{i=1}^{n}x_{i}^{2}}=1$.

\begin{example} \label{example:6} \rm
Find the maximum and minimum values
$$
Q(\mathbf{X}) = x^{2}+y^{2}+2z^{2}-2xy + 4xz + 4yz
$$
subject to the constraint
\begin{equation} \label{eq:18}
x^{2}+y^{2}+z^{2}=1.
\end{equation}


\solution
The matrix of $Q$ is
$$
\mathbf{A} =
\left[\begin{array}{rrrrr}
1 & -1 & 2 \\
-1 & 1 & 2 \\
2&2&2
\end{array}\right]
$$
and
\begin{eqnarray*}
\det(\mathbf{A} - \lambda \mathbf{I}) & =&
\left|\begin{array}{ccccccc}
1-\lambda &-1 & 2 \\
-1 & 1-\lambda & 2 \\
2 & 2 & 2-\lambda
\end{array}\right| \\
& =& -(\lambda+2)(\lambda-2)(\lambda-4),
\end{eqnarray*}
 so
$$
\lambda_{1}=4, \quad \lambda_2=2, \quad \lambda_3=-2
$$
are the eigenvalues of $\mathbf{A}$. Hence, $\lambda_{1}=4$ and
$\lambda_3 = -2$ are the maximum and minimum values of $Q$ subject to
\eqref{eq:18}.

To find the points $(x_{1},y_{1},z_{1})$ where $Q$ attains its
constrained maximum, we first find an eigenvector of ${\bf A}$
corresponding
to $\lambda_{1}=4$. To do this, we find a nontrivial solution of the
system
$$
(\mathbf{A}-4\mathbf{I})
\left[\begin{array}{ccccccc}
 x_{1}\\ y_{1}\\ z_{1}
\end{array}\right]=
\left[\begin{array}{ccccccc}
 -3 & -1 & \phantom{-}2 \\
-1 & -3 & \phantom{-}2 \\
\phantom{-}2 & \phantom{-}2 & -2
\end{array}\right]
\left[\begin{array}{ccccccc}
x_{1}\\y_{1}\\z_{1}
\end{array}\right]=
\left[\begin{array}{ccccccc}
0\\0\\0
\end{array}\right].
$$
All such solutions are multiples of
$
\left[\begin{array}{ccccccc}
1\\1\\2
\end{array}\right].
$
Normalizing this  to satisfy \eqref{eq:18} yields
$$
{\bf X}_{1}=\frac{1}{\sqrt6}
\left[\begin{array}{ccccccc}
x_{1}\\y_{1}\\z_{1}
\end{array}\right]=\pm
\left[\begin{array}{ccccccc}
1\\ 1\\1
\end{array}\right].
$$

To find the points $(x_{3},y_{3},z_{3})$ where $Q$ attains its
constrained minimum, we first find an eigenvector of ${\bf A}$
corresponding
to $\lambda_{3}=-2$. To do this, we find a nontrivial solution of the
system
$$
(\mathbf{A}+2\mathbf{I})
\left[\begin{array}{ccccccc}
 x_{3}\\ y_{3}\\ z_{3}
\end{array}\right]=
\left[\begin{array}{rrrcccc}
 3 & -1 & \phantom{-}2 \\
-1 & 3 & \phantom{-}2 \\
\phantom{-}2 & \phantom{-}2 & 4
\end{array}\right]
\left[\begin{array}{ccccccc}
x_{3}\\y_{3}\\z_{3}
\end{array}\right]=
\left[\begin{array}{ccccccc}
0\\0\\0
\end{array}\right].
$$
All such solutions are multiples of
$
\left[\begin{array}{rcccccc}
1\\1\\-1
\end{array}\right].
$
Normalizing this  to satisfy \eqref{eq:18} yields
$$
{\bf X}_{3}=
\left[\begin{array}{ccccccc}
x_{2}\\y_{2}\\z_{2}
\end{array}\right]=\pm \frac{1}{\sqrt{3}}
\left[\begin{array}{rcccccc}
1\\ 1\\-1
\end{array}\right].
$$

As for    the eigenvalue $\lambda_{2}=2$, we leave it you to verify that
the only unit vectors   that satisfy
${\bf A}{\bf X}_{2}=2{\bf X}_{2}$ are
$$
{\bf X}_{2}=\pm \frac{1}{\sqrt{2}}
\left[\begin{array}{rcccccc}
1\\ 1\\-1
\end{array}\right].
$$
\end{example}

For more on this subject, see  Theorem~\ref{theorem:4}.

\section{Extrema subject to  two constraints} \label{section:5}
Here is Theorem~\ref{theorem:1} with $m=2$.

\begin{theorem} \label{theorem:3}
Suppose that $n>2.$ If\, ${\bf X}_{0}$ is a local extreme point of $f$
subject to
$g_{1}({\bf X})=g_{2}({\bf X})=0$ and
\begin{equation} \label{eq:19}
\left|\begin{array}{ccccccc}
\dst{\frac{{\partial g_{1}({\bf X}_{0})}}{{\partial  x_{r}}}}&
\dst{\frac{{\partial g_{1}({\bf X}_{0})}}{{\partial  x_{s}}}}\\\\
\dst{\frac{{\partial g_{2}({\bf X}_{0})}}{{\partial  x_{r}}}}&
\dst{\frac{{\partial g_{2}({\bf X}_{0})}}{{\partial  x_{s}}}}\\
\end{array}\right|\ne0
\end{equation}
for some $r$ and $s$ in $\{1,2,\dots,n\},$  then
there are constants  $\lambda$ and $\mu$  such that
\begin{equation} \label{eq:20}
\frac{{\partial f({\bf X}_{0})}}{{\partial x_{i}}}-
\lambda\frac{{\partial g_{1}({\bf X}_{0})}}{{\partial x_{i}}}-
\mu\frac{{\partial g_{2}({\bf X}_{0})}}{{\partial x_{i}}}=0,
\end{equation}
$1\le i\le n$.
\end{theorem}

\proof
For notational convenience, let $r=1$ and $s=2$.  Denote
$$
{\bf U}=(x_{3},x_{4},\dots x_{n})\text{\; and\;\;}
{\bf U}_{0}=(x_{30},x_{30},\dots x_{n0}).
$$
Since
\begin{equation} \label{eq:21}
\left|\begin{array}{ccccccc}
\dst{\frac{{\partial g_{1}({\bf X}_{0})}}{{\partial  x_{1}}}}&
\dst{\frac{{\partial g_{1}({\bf X}_{0})}}{{\partial  x_{2}}}}\\\\
\dst{\frac{{\partial g_{2}({\bf X}_{0})}}{{\partial  x_{1}}}}&
\dst{\frac{{\partial g_{2}({\bf X}_{0})}}{{\partial  x_{2}}}}\\
\end{array}\right|\ne0,
\end{equation}
the Implicit Function Theorem
\href{http://ramanujan.math.trinity.edu/wtrench/texts/TRENCH_REAL_ANALYSIS.PDF}
{(Theorem~6.4.1, p.~420)}
implies that there  are unique continuously differentiable functions
$$
h_{1}=h_{1}(x_{3},x_{4},\dots,x_{n})\text{\; and\;\;}
h_{2}=h_{1}(x_{3},x_{4},\dots,x_{n}),
$$
 defined on a neighborhood $N\subset{\mathbb R}^{n-2}$ of
${\bf U}_{0},$  such that $(h_{1}({\bf
U}),h_{2}({\bf U}),{\bf U})\in D$ for all ${\bf U}\in N$,
  $h_{1}({\bf U}_{0})=x_{10}$,
$h_{2}({\bf U}_{0})=x_{20}$, and
\begin{equation} \label{eq:22}
g_{1}(h_{1}({\bf U}),h_{2}({\bf U}),{\bf U})=
g_{2}(h_{1}({\bf U}),h_{2}({\bf U}),{\bf U})=0,\quad  {\bf U}\in N.
\end{equation}

From   \eqref{eq:21}, the system
\begin{equation} \label{eq:23}
\left[\begin{array}{ccccccc}
\dst{\frac{{\partial g_{1}({\bf X}_{0})}}{{\partial  x_{1}}}}&
\dst{\frac{{\partial g_{1}({\bf X}_{0})}}{{\partial  x_{2}}}}\\\\
\dst{\frac{{\partial g_{2}({\bf X}_{0})}}{{\partial  x_{1}}}}&
\dst{\frac{{\partial g_{2}({\bf X}_{0})}}{{\partial  x_{2}}}}\\
\end{array}\right]
\left[\begin{array}{ccccccc}
\lambda\\\mu
\end{array}\right]=
\left[\begin{array}{ccccccc}
f_{x_{1}}({\bf X}_{0})\\f_{x_{2}}({\bf X}_{0})\\
\end{array}\right]
\end{equation}
has a unique solution
\href{http://ramanujan.math.trinity.edu/wtrench/texts/TRENCH_REAL_ANALYSIS.PDF}
{(Theorem~6.1.13, p. 373)}.
This implies \eqref{eq:20} with $i=1$ and $i=2$.
If  $3\le i\le n$, then differentiating \eqref{eq:22} with respect to
$x_{i}$ and recalling that
$(h_{1}({\bf U}_{0}),h_{2}({\bf U}_{0}),{\bf U}_{0})={\bf X}_{0}$
yields
$$
\frac{\partial g_{1}({\bf X}_{0})}{\partial x_{i}}+
\frac{\partial g_{1}({\bf X}_{0})}{\partial x_{1}}
\frac{\partial h_{1}({\bf U}_{0})}{\partial x_{i}}+
\frac{\partial g_{1}({\bf X}_{0})}{\partial x_{2}}
\frac{\partial h_{2}({\bf U}_{0})}{\partial x_{i}}=0
$$
and
$$
\frac{\partial g_{2}({\bf X}_{0})}{\partial x_{i}}+
\frac{\partial g_{2}({\bf X}_{0})}{\partial x_{1}}
\frac{\partial h_{1}({\bf U}_{0})}{\partial x_{i}}+
\frac{\partial g_{2}({\bf X}_{0})}{\partial x_{2}}
\frac{\partial h_{2}({\bf U}_{0})}{\partial x_{i}}=0.
$$
If
 ${\bf X}_{0}$ is a local extreme point of  $f$
subject to  $g_{1}({\bf X})=g_{2}({\bf X})=0$, then ${\bf U}_{0}$
is an unconstrained local extreme point of
$f(h_{1}({\bf U}),h_{2}({\bf U}),{\bf U})$; therefore,
$$
\frac{\partial f({\bf X}_{0})}{\partial x_{i}}+
\frac{\partial f({\bf X}_{0})}{\partial x_{1}}
\frac{\partial h_{1}({\bf U}_{0})}{\partial x_{i}}+
\frac{\partial f({\bf X}_{0})}{\partial x_{2}}
\frac{\partial h_{2}({\bf U}_{0})}{\partial x_{i}}=0.
$$
The last three equations imply that
$$
\left|\begin{array}{ccccccc}
\dst{\frac{{\partial f({\bf X}_{0})}}{{\partial  x_{i}}}}&
\dst{\frac{{\partial f({\bf X}_{0})}}{{\partial  x_{1}}}}&
\dst{\frac{{\partial f({\bf X}_{0})}}{{\partial  x_{2}}}}\\\\
\dst{\frac{{\partial g_{1}({\bf X}_{0})}}{{\partial  x_{i}}}}&
\dst{\frac{{\partial g_{1}({\bf X}_{0})}}{{\partial  x_{1}}}}&
\dst{\frac{{\partial g_{1}({\bf X}_{0})}}{{\partial  x_{2}}}}\\\\
\dst{\frac{{\partial g_{2}({\bf X}_{0})}}{{\partial  x_{i}}}}&
\dst{\frac{{\partial g_{2}({\bf X}_{0})}}{{\partial  x_{1}}}}&
\dst{\frac{{\partial g_{2}({\bf X}_{0})}}{{\partial  x_{2}}}}\\
\end{array}\right|=0,
$$
$$
\left|\begin{array}{ccccccc}
\dst{\frac{{\partial f({\bf X}_{0})}}{{\partial  x_{i}}}}&
\dst{\frac{{\partial g_{1}({\bf X}_{0})}}{{\partial  x_{i}}}}&
\dst{\frac{{\partial g_{2}({\bf X}_{0})}}{{\partial  x_{i}}}}\\\\
\dst{\frac{{\partial f({\bf X}_{0})}}{{\partial  x_{1}}}}&
\dst{\frac{{\partial g_{1}({\bf X}_{0})}}{{\partial  x_{1}}}}&
\dst{\frac{{\partial g_{2}({\bf X}_{0})}}{{\partial  x_{1}}}}\\\\
\dst{\frac{{\partial f({\bf X}_{0})}}{{\partial  x_{2}}}}&
\dst{\frac{{\partial g_{1}({\bf X}_{0})}}{{\partial  x_{2}}}}&
\dst{\frac{{\partial g_{2}({\bf X}_{0})}}{{\partial  x_{2}}}}\\\\
\end{array}\right|=0.
$$
Therefore, there are constants $c_{1}$, $c_{2}$, $c_{3}$, not all zero,
such
that
\begin{equation} \label{eq:24}
\left[\begin{array}{ccccccc}
\dst{\frac{{\partial f({\bf X}_{0})}}{{\partial  x_{i}}}}&
\dst{\frac{{\partial g_{1}({\bf X}_{0})}}{{\partial  x_{i}}}}&
\dst{\frac{{\partial g_{2}({\bf X}_{0})}}{{\partial  x_{i}}}}\\\\
\dst{\frac{{\partial f({\bf X}_{0})}}{{\partial  x_{1}}}}&
\dst{\frac{{\partial g_{1}({\bf X}_{0})}}{{\partial  x_{1}}}}&
\dst{\frac{{\partial g_{2}({\bf X}_{0})}}{{\partial  x_{1}}}}\\\\
\dst{\frac{{\partial f({\bf X}_{0})}}{{\partial  x_{2}}}}&
\dst{\frac{{\partial g_{1}({\bf X}_{0})}}{{\partial  x_{2}}}}&
\dst{\frac{{\partial g_{2}({\bf X}_{0})}}{{\partial  x_{2}}}}\\\\
\end{array}\right]
\left[\begin{array}{ccccccc}
c_{1}\\c_{2}\\c_{3}
\end{array}\right]=
\left[\begin{array}{ccccccc}
0\\0\\0
\end{array}\right].
\end{equation}
If $c_{1}=0$, then
$$
\left[\begin{array}{ccccccc}
\dst{\frac{{\partial g_{1}({\bf X}_{0})}}{{\partial  x_{1}}}}&
\dst{\frac{{\partial g_{1}({\bf X}_{0})}}{{\partial  x_{2}}}}\\\\
\dst{\frac{{\partial g_{2}({\bf X}_{0})}}{{\partial  x_{1}}}}&
\dst{\frac{{\partial g_{2}({\bf X}_{0})}}{{\partial  x_{2}}}}\\
\end{array}\right]
\left[\begin{array}{ccccccc}
c_{2}\\c_{3}
\end{array}\right]=
\left[\begin{array}{ccccccc}
0\\0
\end{array}\right],
$$
so \eqref{eq:19} implies that $c_{2}=c_{3}=0$;
hence, we
may assume that $c_{1}=1$ in a nontrivial solution of \eqref{eq:24}.
Therefore,
\begin{equation} \label{eq:25}
\left[\begin{array}{ccccccc}
\dst{\frac{{\partial f({\bf X}_{0})}}{{\partial  x_{i}}}}&
\dst{\frac{{\partial g_{1}({\bf X}_{0})}}{{\partial  x_{i}}}}&
\dst{\frac{{\partial g_{2}({\bf X}_{0})}}{{\partial  x_{i}}}}\\\\
\dst{\frac{{\partial f({\bf X}_{0})}}{{\partial  x_{1}}}}&
\dst{\frac{{\partial g_{1}({\bf X}_{0})}}{{\partial  x_{1}}}}&
\dst{\frac{{\partial g_{2}({\bf X}_{0})}}{{\partial  x_{1}}}}\\\\
\dst{\frac{{\partial f({\bf X}_{0})}}{{\partial  x_{2}}}}&
\dst{\frac{{\partial g_{1}({\bf X}_{0})}}{{\partial  x_{2}}}}&
\dst{\frac{{\partial g_{2}({\bf X}_{0})}}{{\partial  x_{2}}}}\\\\
\end{array}\right]
\left[\begin{array}{ccccccc}
1\\c_{2}\\c_{3}
\end{array}\right]=
\left[\begin{array}{ccccccc}
0\\0\\0
\end{array}\right],
\end{equation}
which implies that
$$
\left[\begin{array}{ccccccc}
\dst{\frac{{\partial g_{1}({\bf X}_{0})}}{{\partial  x_{1}}}}&
\dst{\frac{{\partial g_{1}({\bf X}_{0})}}{{\partial  x_{2}}}}\\\\
\dst{\frac{{\partial g_{2}({\bf X}_{0})}}{{\partial  x_{1}}}}&
\dst{\frac{{\partial g_{2}({\bf X}_{0})}}{{\partial  x_{2}}}}\\
\end{array}\right]
\left[\begin{array}{ccccccc}
-c_{2}\\-c_{3}
\end{array}\right]=
\left[\begin{array}{ccccccc}
f_{x_{1}}({\bf X}_{0})\\f_{x_{2}}({\bf X}_{0})\\
\end{array}\right].
$$
Since  \eqref{eq:23}  has only one solution,
 this implies that
$c_{2}=-\lambda$ and $c_{2}=-\mu$, so \eqref{eq:25}   becomes
$$
\left[\begin{array}{ccccccc}
\dst{\frac{{\partial f({\bf X}_{0})}}{{\partial  x_{i}}}}&
\dst{\frac{{\partial g_{1}({\bf X}_{0})}}{{\partial  x_{i}}}}&
\dst{\frac{{\partial g_{2}({\bf X}_{0})}}{{\partial  x_{i}}}}\\\\
\dst{\frac{{\partial f({\bf X}_{0})}}{{\partial  x_{1}}}}&
\dst{\frac{{\partial g_{1}({\bf X}_{0})}}{{\partial  x_{1}}}}&
\dst{\frac{{\partial g_{2}({\bf X}_{0})}}{{\partial  x_{1}}}}\\\\
\dst{\frac{{\partial f({\bf X}_{0})}}{{\partial  x_{2}}}}&
\dst{\frac{{\partial g_{1}({\bf X}_{0})}}{{\partial  x_{2}}}}&
\dst{\frac{{\partial g_{2}({\bf X}_{0})}}{{\partial  x_{2}}}}\\\\
\end{array}\right]
\left[\begin{array}{rcccccc}
1\\-\lambda\\-\mu
\end{array}\right]=
\left[\begin{array}{ccccccc}
0\\0\\0
\end{array}\right].
$$
Computing the topmost entry of the vector on the left
yields \eqref{eq:20}.
\hfill\bbox

\begin{example}\label{example:7}\rm
Minimize
$$
f(x,y,z,w) = x^{2}+y^{2}+z^{2}+w^{2}
$$
subject to
\begin{equation} \label{eq:26}
x+y+z+w  = 10 \text{\; and\;\;}
x-y+z+3w = 6.
\end{equation}
\solution
Let
$$
L =
\frac{x^{2}+y^{2}+z^{2}+w^{2}}{2}-\lambda(x+y+z+w)-\mu(x-y+z+3w);
$$
then
\begin{eqnarray*}
L_x & =& x-\lambda-\mu \\
L_y & =& y-\lambda+\mu \\
L_z & =& z-\lambda-\mu \\
L_w & =& w-\lambda-3\mu,
\end{eqnarray*}
so
\begin{equation} \label{eq:27}
x_{0} = \lambda+\mu, \quad y_{0} = \lambda-\mu, \quad z_{0} = \lambda+\mu, \quad
w_{0} = \lambda+3\mu.
\end{equation}
This and \eqref{eq:26} imply that
\begin{eqnarray*}
(\lambda+\mu)+(\lambda-\mu)+(\lambda+\mu) + (\lambda+3\mu) & =& 10 \\
(\lambda+\mu)-(\lambda-\mu)+(\lambda+\mu)+ (3\lambda+9\mu) & =&
\phantom{1}6.
\end{eqnarray*}
Therefore,
\begin{eqnarray*}
4\lambda + \phantom{1}4\mu & =& 10 \\
4\lambda + 12\mu & = &\phantom{1}6,
\end{eqnarray*}
so
$\lambda=3$ and $\mu = -1/2$.
Now \eqref{eq:27} implies that
$$
(x_{0},y_{0},z_{0},w_{0}) =
\left(\frac{5}{2},\frac{7}{2},\frac{5}{2}
\frac{3}{2}\right).
$$
Since $f(x,y,z,w)$ is the square of the distance from $(x,y,z,w)$   to the
origin, it attains a  minimum value (but not
a maximum  value) subject to  the constraints; hence the
constrained minimum value is
$$
f\left(\frac{5}{2},\frac{7}{2},\frac{5}{2},
 \frac{3}{2}\right)=27.
$$
\end{example}

\begin{example} \label{example:8} \rm
The distance between two curves in $\mathbb{R}^{2}$ is the minimum value of
$$
\sqrt{(x_{1}-x_{2})^{2}+(y_{1}-y_{2})^{2}},
$$
 where $(x_{1},y_{1})$ is on one curve and
$(x_{2},y_{2})$ is on the other. Find the distance between the ellipse
$$
x^{2}+2y^{2}=1
$$
and the line
\begin{equation} \label{eq:28}
x+y=4.
\end{equation}

\solution
We must minimize
$$
d^{2}=(x_{1}-x_{2})^{2} + (y_{1}-y_{2})^{2}
$$
subject to
$$
x_{1}^{2} + 2y_{1}^{2}  =1 \text{\; and\;\;} x_{2}+y_{2} = 4.
$$
Let
$$
L = \frac{(x_{1}-x_{2})^{2} + (y_{1}-y_{2})^{2} -
\lambda(x_{1}^{2} + 2y_{1}^{2})}{2} -\mu(x_{2}+y_{2});
$$
then
\begin{eqnarray*}
L_{x_{1}}&=&x_{1}-x_{2}-\lambda x_{1}\\
 L_{y_{1}}&=&y_{1}-y_{2}-2\lambda y_{1}\\
 L_{x_{2}}&=&x_{2}-x_{1}-\mu\\
L_{y_{2}}&=&y_{2}-y_{1}-\mu,
\end{eqnarray*}
so
\begin{eqnarray*}
x_{10}-x_{20}&=&\lambda x_{10} \text{\; \quad (i)}\\
 y_{10}-y_{20}&=&2\lambda y_{10}\text{\quad (ii)}\\
 x_{20}-x_{10}&=&\mu\text{\quad \quad \;\;(iii)}  \\
y_{20}-y_{10}&=&\mu.\text{\quad \quad \;\;(iv)}
\end{eqnarray*}
From (i) and (iii), $\mu=-\lambda x_{10}$; from (ii)  and (iv),
$\mu=-2\lambda y_{10}$. Since the curves do not intersect, $\lambda\ne0$,
so $x_{10}=2y_{10}$. Since $x_{10}^{2}+2y_{10}^{2}=1$ and
$(x_{0},y_{0})$ is in the first quadrant,
\begin{equation} \label{eq:29}
(x_{10},y_{10})=\left(\frac{2}{\sqrt{6}},\frac{1}{\sqrt{6}}\right).
\end{equation}
Now (iii), (iv), and \eqref{eq:28} yield the simultaneous system
$$
x_{20}-y_{20}=x_{10}-y_{10}=\frac{1}{\sqrt{6}},\quad
x_{20}+y_{20}=4,
$$
so
$$
(x_{20},y_{20}) = \left(2+\frac{1}{2\sqrt{6}},
 2-\frac{1}{2\sqrt{6}}\right).
$$
From this and \eqref{eq:29}, the distance between the curves is
$$
\left[\left(2+\frac{1}{2\sqrt{6}} -\frac{2}{\sqrt{6}}
\right)^{2} + \left(2-
\frac{1}{2\sqrt{6}} - \frac{1}{
\sqrt{6}}\right)^{2}\right]^{1/2}
= \sqrt{2} \left(2-\frac{3}{2\sqrt{6}}\right).
$$
\end{example}



\section{Proof of Theorem~1} \label{section:6}
\proof
For notational convenience, let $r_{\ell}=\ell$,
$1\le \ell\le m$, so \eqref{eq:6} becomes
\begin{equation} \label{eq:30}
\left|\begin{array}{ccccccc}
\dst{\frac{\partial{g_{1}({\bf X}_{0})}}{\partial{x_{1}}}}&
\dst{\frac{\partial{g_{1}({\bf X}_{0})}}{\partial{x_{2}}}}&
\cdots&
\dst{\frac{\partial{g_{1}({\bf X}_{0})}}{\partial{x_{m}}}}\\ \\
\dst{\frac{\partial{g_{2}({\bf X}_{0})}}{\partial{x_{1}}}}&
\dst{\frac{\partial{g_{2}({\bf X}_{0})}}{\partial{x_{2}}}}&
\cdots&
\dst{\frac{\partial{g_{2}({\bf X}_{0})}}{\partial{x_{m}}}}\\
\vdots&\vdots&\ddots&\vdots\\
\dst{\frac{\partial{g_{m}({\bf X}_{0})}}{\partial{x_{1}}}}&
\dst{\frac{\partial{g_{m}({\bf X}_{0})}}{\partial{x_{2}}}}&
\cdots&
\dst{\frac{\partial{g_{m}({\bf X}_{0})}}{\partial{x_{m}}}}\\ \\
\end{array}\right|\ne0
\end{equation}
 Denote
$$
{\bf U}=(x_{m+1},x_{m+2},\dots x_{n})\text{\; and\;\;}
{\bf U}_{0}=(x_{m+1,0},x_{m+2,0},\dots x_{n0}).
$$
From \eqref{eq:30}, the   Implicit Function Theorem
implies
that there  are unique continuously differentiable functions
$h_{\ell}=h_{\ell}({\bf U})$, $1\le \ell\le m$,
 defined on a neighborhood $N$ of
${\bf U}_{0}$,  such that
$$
(h_{1}({\bf U}),h_{2}({\bf U}),\dots, h_{m}({\bf U}),
{\bf U})\in D,
\text{\; for all\;\;} {\bf U}\in N,
$$
\begin{equation} \label{eq:31}
(h_{1}({\bf U_{0}}),h_{2}({\bf U_{0}}),\dots, h_{m}({\bf U_{0}}),{\bf
U}_{0})={\bf X}_{0},
\end{equation}
and
\begin{equation} \label{eq:32}
g_{\ell}(h_{1}({\bf U}),h_{2}({\bf U}),\dots, h_{m}({\bf U}),{\bf
U})=0,\quad
 {\bf U}\in N, \quad 1\le \ell\le m.
\end{equation}
Again from   \eqref{eq:30}, the system
\begin{equation} \label{eq:33}
\left[\begin{array}{ccccccc}
\dst{\frac{\partial{g_{1}({\bf X}_{0})}}{\partial{x_{1}}}}&
\dst{\frac{\partial{g_{1}({\bf X}_{0})}}{\partial{x_{2}}}}&
\cdots&
\dst{\frac{\partial{g_{1}({\bf X}_{0})}}{\partial{x_{m}}}}\\ \\
\dst{\frac{\partial{g_{2}({\bf X}_{0})}}{\partial{x_{1}}}}&
\dst{\frac{\partial{g_{2}({\bf X}_{0})}}{\partial{x_{2}}}}&
\cdots&
\dst{\frac{\partial{g_{2}({\bf X}_{0})}}{\partial{x_{m}}}}\\
\vdots&\vdots&\ddots&\vdots\\
\dst{\frac{\partial{g_{m}({\bf X}_{0})}}{\partial{x_{1}}}}&
\dst{\frac{\partial{g_{m}({\bf X}_{0})}}{\partial{x_{2}}}}&
\cdots&
\dst{\frac{\partial{g_{m}({\bf X}_{0})}}{\partial{x_{m}}}}\\ \\
\end{array}\right]
\left[\begin{array}{ccccccc}
\lambda_{1}\\\lambda_{2}\\ \vdots\\\lambda_{m}
\end{array}\right]=
\left[\begin{array}{ccccccc}
f_{x_{1}}({\bf X}_{0})\\f_{x_{2}}({\bf X}_{0})\\\vdots\\ f_{x_{m}}({\bf
X}_{0})
\end{array}\right]
\end{equation}
has a unique solution.
This implies that
\begin{equation} \label{eq:34}
\frac{\partial{f({\bf X}_{0})}}{\partial x_{i}}
-\lambda_{1}\frac{\partial{g_{1}({\bf X}_{0})}}{\partial x_{i}}
-\lambda_{2}\frac{\partial{g_{2}({\bf X}_{0})}}{\partial x_{i}}-\cdots
-\lambda_{m}\frac{\partial{g_{m}({\bf X}_{0})}}{\partial x_{i}}=0
\end{equation}
for  $1\le i\le m$.

If $m+1\le i\le n$, differentiating \eqref{eq:32} with respect to $x_{i}$
and recalling \eqref{eq:31} yields
$$
\frac{\partial g_{\ell}({\bf X}_{0})} {\partial x_{i}} +\sum_{j=1}^{m}
\frac{\partial g_{\ell}({\bf X}_{0})}{\partial x_{j}}
\frac{\partial h_{j}({\bf X}_{0})}{\partial x_{i}}=0, \quad 1\le \ell\le m.
$$
If
 ${\bf X}_{0}$ is local extreme point   $f$
subject to  $g_{1}({\bf X})=g_{2}({\bf X})= \cdots =g_{m}({\bf X})=0$, then
${\bf U}_{0}$ is an unconstrained local extreme point of
$f(h_{1}({\bf U}),h_{2}({\bf U}), \dots h_{m}({\bf U}),{\bf U})$;
therefore,
$$
\frac{\partial f({\bf X}_{0})} {\partial x_{i}} +\sum_{j=1}^{m}
\frac{\partial f({\bf X}_{0})}{\partial x_{j}}
\frac{\partial h_{j}({\bf X}_{0})}{\partial x_{i}}=0.
$$
The last two equations imply that
$$
\left|\begin{array}{ccccccc}
\dst{\frac{\partial{f({\bf X}_{0})}}{\partial{x_{i}}}}&
\dst{\frac{\partial{f({\bf X}_{0})}}{\partial{x_{1}}}}&
\dst{\frac{\partial{f({\bf X}_{0})}}{\partial{x_{2}}}}&
\cdots&
\dst{\frac{\partial{f({\bf X}_{0})}}{\partial{x_{m}}}}\\ \\
\dst{\frac{\partial{g_{1}({\bf X}_{0})}}{\partial{x_{i}}}}&
\dst{\frac{\partial{g_{1}({\bf X}_{0})}}{\partial{x_{1}}}}&
\dst{\frac{\partial{g_{1}({\bf X}_{0})}}{\partial{x_{2}}}}&
\cdots&
\dst{\frac{\partial{g_{1}({\bf X}_{0})}}{\partial{x_{m}}}}\\ \\
\dst{\frac{\partial{g_{2}({\bf X}_{0})}}{\partial{x_{i}}}}&
\dst{\frac{\partial{g_{2}({\bf X}_{0})}}{\partial{x_{1}}}}&
\dst{\frac{\partial{g_{2}({\bf X}_{0})}}{\partial{x_{2}}}}&
\cdots&
\dst{\frac{\partial{g_{2}({\bf X}_{0})}}{\partial{x_{m}}}}\\
\vdots&\vdots&\vdots&\ddots&\vdots\\
\dst{\frac{\partial{g_{m}({\bf X}_{0})}}{\partial{x_{i}}}}&
\dst{\frac{\partial{g_{m}({\bf X}_{0})}}{\partial{x_{1}}}}&
\dst{\frac{\partial{g_{m}({\bf X}_{0})}}{\partial{x_{2}}}}&
\cdots&
\dst{\frac{\partial{g_{m}({\bf X}_{0})}}{\partial{x_{m}}}}\\ \\
\end{array}\right|=0,
$$
so
$$
\left|\begin{array}{ccccccc}
\dst{\frac{\partial f({\bf X}_{0})}{\partial x_{i}}}&
\dst{\frac{\partial g_{1}({\bf X}_{0})}{\partial x_{i}}}&
\dst{\frac{\partial g_{2}({\bf X}_{0})}{\partial x_{i}}}&\dots&
\dst{\frac{\partial g_{m}({\bf X}_{0})}{\partial x_{i}}}\\\\
\dst{\frac{\partial f({\bf X}_{0})}{\partial x_{1}}}&
\dst{\frac{\partial g_{1}({\bf X}_{0})}{\partial x_{1}}}&
\dst{\frac{\partial g_{2}({\bf X}_{0})}{\partial x_{1}}}&\dots&
\dst{\frac{\partial g_{m}({\bf X}_{0})}{\partial x_{1}}}\\\\
\dst{\frac{\partial f({\bf X}_{0})}{\partial x_{2}}}&
\dst{\frac{\partial g_{1}({\bf X}_{0})}{\partial x_{2}}}&
\dst{\frac{\partial g_{2}({\bf X}_{0})}{\partial x_{2}}}&\dots&
\dst{\frac{\partial g_{m}({\bf X}_{0})}{\partial x_{2}}}\\
\vdots&\vdots&\vdots&\ddots&\vdots\\
\dst{\frac{\partial f({\bf X}_{0})}{\partial x_{m}}}&
\dst{\frac{\partial g_{1}({\bf X}_{0})}{\partial x_{m}}}&
\dst{\frac{\partial g_{2}({\bf X}_{0})}{\partial x_{m}}}&\dots&
\dst{\frac{\partial g_{m}({\bf X}_{0})}{\partial x_{m}}}
\end{array}\right|=0.
$$
Therefore, there are constant $c_{0}$, $c_{1}$, \dots $c_{m}$, not all
zero, such that
\begin{equation} \label{eq:35}
\left[\begin{array}{ccccccc}
\dst{\frac{\partial f({\bf X}_{0})}{\partial x_{i}}}&
\dst{\frac{\partial g_{1}({\bf X}_{0})}{\partial x_{i}}}&
\dst{\frac{\partial g_{2}({\bf X}_{0})}{\partial x_{i}}}&\dots&
\dst{\frac{\partial g_{m}({\bf X}_{0})}{\partial x_{i}}}\\\\
\dst{\frac{\partial f({\bf X}_{0})}{\partial x_{1}}}&
\dst{\frac{\partial g_{1}({\bf X}_{0})}{\partial x_{1}}}&
\dst{\frac{\partial g_{2}({\bf X}_{0})}{\partial x_{1}}}&\dots&
\dst{\frac{\partial g_{m}({\bf X}_{0})}{\partial x_{1}}}\\\\
\dst{\frac{\partial f({\bf X}_{0})}{\partial x_{2}}}&
\dst{\frac{\partial g_{1}({\bf X}_{0})}{\partial x_{2}}}&
\dst{\frac{\partial g_{2}({\bf X}_{0})}{\partial x_{2}}}&\dots&
\dst{\frac{\partial g_{m}({\bf X}_{0})}{\partial x_{2}}}\\
\vdots&\vdots&\vdots&\ddots&\vdots\\
\dst{\frac{\partial f({\bf X}_{0})}{\partial x_{m}}}&
\dst{\frac{\partial g_{1}({\bf X}_{0})}{\partial x_{m}}}&
\dst{\frac{\partial g_{2}({\bf X}_{0})}{\partial x_{m}}}&\dots&
\dst{\frac{\partial g_{m}({\bf X}_{0})}{\partial x_{m}}}\\
\end{array}\right]
\left[\begin{array}{ccccccc}
c_{0}\\c_{1}\\c_{3}\\\vdots\\c_{m}
\end{array}\right]=
\left[\begin{array}{ccccccc}
0\\0\\0\\\vdots\\0
\end{array}\right].
\end{equation}
If  $c_{0}=0$, then
$$
\left[\begin{array}{ccccccc}
\dst{\frac{\partial{g_{1}({\bf X}_{0})}}{\partial{x_{1}}}}&
\dst{\frac{\partial{g_{1}({\bf X}_{0})}}{\partial{x_{2}}}}&
\cdots&
\dst{\frac{\partial{g_{1}({\bf X}_{0})}}{\partial{x_{m}}}}\\ \\
\dst{\frac{\partial{g_{2}({\bf X}_{0})}}{\partial{x_{1}}}}&
\dst{\frac{\partial{g_{2}({\bf X}_{0})}}{\partial{x_{2}}}}&
\cdots&
\dst{\frac{\partial{g_{2}({\bf X}_{0})}}{\partial{x_{m}}}}\\
\vdots&\vdots&\ddots&\vdots\\
\dst{\frac{\partial{g_{m}({\bf X}_{0})}}{\partial{x_{1}}}}&
\dst{\frac{\partial{g_{m}({\bf X}_{0})}}{\partial{x_{2}}}}&
\cdots&
\dst{\frac{\partial{g_{m}({\bf X}_{0})}}{\partial{x_{m}}}}\\ \\
\end{array}\right]
\left[\begin{array}{ccccccc}
c_{1}\\c_{2}\\\vdots\\c_{m}
\end{array}\right]=
\left[\begin{array}{ccccccc}
0\\0\\\vdots\\0
\end{array}\right]
$$
and \eqref{eq:30} implies that $c_{1}=c_{2}=\cdots = c_{m}=0$; hence, we
may assume that $c_{0}=1$ in a nontrivial solution of \eqref{eq:35}.
Therefore,
\begin{equation} \label{eq:36}
\left[\begin{array}{ccccccc}
\dst{\frac{\partial f({\bf X}_{0})}{\partial x_{i}}}&
\dst{\frac{\partial g_{1}({\bf X}_{0})}{\partial x_{i}}}&
\dst{\frac{\partial g_{2}({\bf X}_{0})}{\partial x_{i}}}&\dots&
\dst{\frac{\partial g_{m}({\bf X}_{0})}{\partial x_{i}}}\\\\
\dst{\frac{\partial f({\bf X}_{0})}{\partial x_{1}}}&
\dst{\frac{\partial g_{1}({\bf X}_{0})}{\partial x_{1}}}&
\dst{\frac{\partial g_{2}({\bf X}_{0})}{\partial x_{1}}}&\dots&
\dst{\frac{\partial g_{m}({\bf X}_{0})}{\partial x_{1}}}\\\\
\dst{\frac{\partial f({\bf X}_{0})}{\partial x_{2}}}&
\dst{\frac{\partial g_{1}({\bf X}_{0})}{\partial x_{2}}}&
\dst{\frac{\partial g_{2}({\bf X}_{0})}{\partial x_{2}}}&\dots&
\dst{\frac{\partial g_{m}({\bf X}_{0})}{\partial x_{2}}}\\
\vdots&\vdots&\vdots&\ddots&\vdots\\
\dst{\frac{\partial f({\bf X}_{0})}{\partial x_{m}}}&
\dst{\frac{\partial g_{1}({\bf X}_{0})}{\partial x_{m}}}&
\dst{\frac{\partial g_{2}({\bf X}_{0})}{\partial x_{m}}}&\dots&
\dst{\frac{\partial g_{m}({\bf X}_{0})}{\partial x_{m}}}\\
\end{array}\right]
\left[\begin{array}{ccccccc}
1\\c_{1}\\c_{2}\\\vdots\\c_{m}
\end{array}\right]=
\left[\begin{array}{ccccccc}
0\\0\\0\\\vdots\\0
\end{array}\right],
\end{equation}
which implies that
$$
\left[\begin{array}{ccccccc}
\dst{\frac{\partial{g_{1}({\bf X}_{0})}}{\partial{x_{1}}}}&
\dst{\frac{\partial{g_{1}({\bf X}_{0})}}{\partial{x_{2}}}}&
\cdots&
\dst{\frac{\partial{g_{1}({\bf X}_{0})}}{\partial{x_{m}}}}\\ \\
\dst{\frac{\partial{g_{2}({\bf X}_{0})}}{\partial{x_{1}}}}&
\dst{\frac{\partial{g_{2}({\bf X}_{0})}}{\partial{x_{2}}}}&
\cdots&
\dst{\frac{\partial{g_{2}({\bf X}_{0})}}{\partial{x_{m}}}}\\
\vdots&\vdots&\ddots&\vdots\\
\dst{\frac{\partial{g_{m}({\bf X}_{0})}}{\partial{x_{1}}}}&
\dst{\frac{\partial{g_{m}({\bf X}_{0})}}{\partial{x_{2}}}}&
\cdots&
\dst{\frac{\partial{g_{m}({\bf X}_{0})}}{\partial{x_{m}}}}\\ \\
\end{array}\right]
\left[\begin{array}{ccccccc}
-c_{1}\\-c_{2}\\ \vdots\\-c_{m}
\end{array}\right]=
\left[\begin{array}{ccccccc}
f_{x_{1}}({\bf X}_{0})\\f_{x_{2}}({\bf X}_{0})\\\vdots\\ f_{x_{m}}({\bf
X}_{0})
\end{array}\right]
$$
Since  \eqref{eq:33}  has only one solution,
 this implies that
$c_{j}=-\lambda_{j}$, $1\le j\le n$, so \eqref{eq:36}  becomes
$$
\left[\begin{array}{ccccccc}
\dst{\frac{\partial f({\bf X}_{0})}{\partial x_{i}}}&
\dst{\frac{\partial g_{1}({\bf X}_{0})}{\partial x_{i}}}&
\dst{\frac{\partial g_{2}({\bf X}_{0})}{\partial x_{i}}}&\dots&
\dst{\frac{\partial g_{m}({\bf X}_{0})}{\partial x_{i}}}\\\\
\dst{\frac{\partial f({\bf X}_{0})}{\partial x_{1}}}&
\dst{\frac{\partial g_{1}({\bf X}_{0})}{\partial x_{1}}}&
\dst{\frac{\partial g_{2}({\bf X}_{0})}{\partial x_{1}}}&\dots&
\dst{\frac{\partial g_{m}({\bf X}_{0})}{\partial x_{1}}}\\\\
\dst{\frac{\partial f({\bf X}_{0})}{\partial x_{2}}}&
\dst{\frac{\partial g_{1}({\bf X}_{0})}{\partial x_{2}}}&
\dst{\frac{\partial g_{2}({\bf X}_{0})}{\partial x_{2}}}&\dots&
\dst{\frac{\partial g_{m}({\bf X}_{0})}{\partial x_{2}}}\\
\vdots&\vdots&\vdots&\ddots&\vdots\\
\dst{\frac{\partial f({\bf X}_{0})}{\partial x_{m}}}&
\dst{\frac{\partial g_{1}({\bf X}_{0})}{\partial x_{m}}}&
\dst{\frac{\partial g_{2}({\bf X}_{0})}{\partial x_{m}}}&\dots&
\dst{\frac{\partial g_{m}({\bf X}_{0})}{\partial x_{m}}}\\
\end{array}\right]
\left[\begin{array}{ccccccc}
1\\-\lambda_{1}\\-\lambda_{2}\\\vdots\\-\lambda_{m}
\end{array}\right]=
\left[\begin{array}{ccccccc}
0\\0\\0\\\vdots\\0
\end{array}\right].
$$
Computing the topmost entry of the vector on the left yields
yields \eqref{eq:34},
which completes the proof.\endproof


\begin{example}\label{example:9}\rm
Minimize
$\dst{\sum_{i=1}^{n}x_{i}^{2}}$ subject to
\begin{equation} \label{eq:37}
\sum_{i=1}^{n}a_{r i}x_{i}=c_{r}, \quad   1\le r\le m,
\end{equation}
where
\begin{equation} \label{eq:38}
\sum_{i=1}^{n}a_{ri}a_{si}=
\begin{cases}
1 &\text{if } r=s,\\
0  &\text{if }r\ne s.
\end{cases}
\end{equation}

\solution \quad
Let
$$
L =\frac{1}{2}
\sum_{i=1}^{n}x_{i}^{2}-\sum_{s=1}^{m}\lambda_{s}
\sum_{i=1}^{n}a_{s i}x_{i}.
$$
Then
$$
L_{x_{i}}=x_{i}-\sum_{s=1}^{m}\lambda_{s}a_{si},\quad 1\le i\le n,
$$
so
\begin{equation} \label{eq:39}
x_{i0}=\sum_{s=1}^{m}\lambda_{s}a_{s i}\quad 1\le i\le n,
\end{equation}
and
$$
a_{ri}x_{i0}=\sum_{s=1}^{m}\lambda_{s}a_{ri}a_{s i}.
$$
Now \eqref{eq:38} implies that
$$
\sum_{i=1}^{n}a_{ri}x_{i0}=\sum_{s=1}^{m}\lambda_{s}
\sum_{i=1}^{n}a_{ri}a_{s i}=\lambda_{r}.
$$
From this and \eqref{eq:37},
$\lambda_{r}=c_{r}$, $1\le r\le m$, and    \eqref{eq:39} implies that
$$
x_{i0}=\sum_{s=1}^{m}c_{s}a_{s i},\quad 1\le i\le n.
$$
Therefore,
$$
x_{i0}^{2}=\sum_{r,s=1}^{m}c_{r}c_{s}a_{r i}a_{si},\quad  1\le i\le n,
$$
and \eqref{eq:38} implies that
$$
\sum_{i=1}^{n}x_{i0}^{2}=\sum_{r,s=1}^{m}c_{r}c_{s}
\sum_{i=1}^{n}a_{r i}a_{si}=\sum_{r=1}^{m}c_{r}^{2}.
$$

\end{example}

The next theorem provides further information on the relationship between
the eigenvalues of a symmetric matrix and constrained extrema of its
quadratic form. It can be proved by successive applications of
Theorem~\ref{theorem:1}; however,  we omit the proof.

\begin{theorem}\label{theorem:4}
Suppose that ${\bf A}=[a_{rs}]_{r,s=1}^{n}\in {\mathbb R}^{n\times n}$ is
symmetric and let
$$
Q({\bf x})=\sum_{r,s=1}^{n}a_{rs}x_{r}x_{s}.
$$
  Suppose    also
that
$$
{\bf x}_{1}=
\left[\begin{array}{ccccccc}
x_{11}\\x_{21}\\\vdots\\x_{n1}
\end{array}\right]
$$
 minimizes $Q$  subject to  $\sum_{i=1}^{n}x_{i}^{2}$. For $2\le r\le n$,
suppose  that
$$
{\bf x}_{r}=
\left[\begin{array}{ccccccc}
x_{1r}\\x_{2r}\\\vdots\\x_{nr}
\end{array}\right],
$$
 minimizes $Q$ subject to
$$
\sum_{i=1}^{n}x_{i}^{2} =1 \text{\; and\;\;}
\sum_{i=1}^{n}x_{is}x_{i}=0,\quad 1\le s\le r-1.
$$
Denote
$$
\lambda_{r}=\sum_{i,j=1}^{n}a_{ij}x_{ir}x_{jr}, \quad 1\le r\le n.
$$
Then
$$
\lambda_{1}\le \lambda_{2}\le \cdots\le \lambda_{n}
\text{\; and\;\;}   Ax_{r}=\lambda_{r}x_{r},\quad 1\le r\le n.
$$
\end{theorem}

\newpage
\section{Exercises} \label{section:7}

\begin{exerciselist}
\item\label{exer:1}
Find the point on the plane $2x+3y+z=7$ closest to $(1,-2,3)$.

\item\label{exer:2}
Find the extreme values of $f(x,y)=2x+y$ subject to $x^{2}+y^{2}=5$.

\item\label{exer:3}
Suppose that $a,b>0$ and $a\alpha^{2}+b\beta^{2}=1$. Find the extreme
values of
$f(x,y)=\beta x+\alpha y$ subject to $ax^{2}+by^{2}=1$.

\item\label{exer:4}
Find the points on the circle $x^{2}+y^{2}=320$  closest to and
farthest from $(2,4)$.

\item\label{exer:5}
Find the extreme values of
$$
f(x,y,z)=2x+3y+z\text{\quad subject to\quad}
x^{2}+2y^{2}+3z^{2}=1.
$$

\item\label{exer:6}
Find the maximum value of $f(x,y)=xy$ on the line $ax+by=1$, where $a,b>0$.

\item\label{exer:7}
A rectangle has perimeter $p$. Find  its largest possible area.

\item\label{exer:8}
A rectangle has area $A$. Find its  smallest possible perimeter.

\item\label{exer:9}
A closed rectangular box has surface area $A$.
Find it largest possible volume.

\item\label{exer:10}
The sides and bottom of a rectangular box have total area $A$. Find  its
largest possible volume.

\item\label{exer:11}
A rectangular box with no top has volume $V$. Find
its smallest possible   surface area.

\item\label{exer:12} Maximize $f(x,y,z)=xyz$ subject to $$
\frac{x}{a}+\frac{y}{b}+\frac{z}{c}=1, $$ where $a$, $b$, $c>0$.

\item\label{exer:13}
Two vertices of a triangle are $(-a,0)$ and $(a,0)$, and the third
is on the ellipse
$$
\frac{x^{2}}{a^{2}}+\frac{y^{2}}{b^{2}}=1.
$$
Find its largest possible area.

\item\label{exer:14}
 Show that the triangle with the greatest possible area for a
given perimeter is equilateral, given that the area of
a triangle with sides $x$, $y$, $z$ and perimeter $s$ is
$$
A= \sqrt{s(s-x)(s-y)(s-z)}.
$$

\item\label{exer:15}
A box with sides parallel to the coordinate planes
has its vertices on the ellipsoid
$$
\frac{x^{2}}{a^{2}}+\frac{y^{2}}{b^{2}}+\frac{z^{2}}{c^{2}}=1.
$$
Find its largest possible volume.

\item\label{exer:16}
Derive a formula for the distance from  $(x_{1},y_{1},z_{1})$
to the plane
$$
ax+by+cz=\sigma.
$$

\item\label{exer:17}
 Let $\mathbf{X}_{i}=(x_{i},y_{i},z_{i})$, $1 \le i \le n$.
Find the point in the plane
$$
ax+by+cz=\sigma
$$
for which $\sum_{i=1}^{n}|\mathbf{X}-\mathbf{X}_{i}|^{2}$
is a minimum.  Assume that none of the ${\bf X}_{i}$ are in the plane.

\item\label{exer:18}
Find the extreme values of
$f({\bf X})=\dst{\sum_{i=1}^{n}(x_{i}-c_{i})^{2}}$  subject to
$\dst{\sum_{i=1}^{n}x_{i}^{2}}=1$.

\item\label{exer:19}
Find the extreme values of
$$
f(x,y,z)=2xy+2xz+2yz\text{\quad subject to\quad}
x^{2}+y^{2}+z^{2}=1.
$$

\item\label{exer:20}
Find the extreme values of
$$
f(x,y,z)=3x^{2}+2y^{2}+3z^{2}+2xz\text{\quad subject to\quad}
x^{2}+y^{2}+z^{2}=1.
$$

\item\label{exer:21}
Find the extreme values of
$$
f(x,y)=x^{2}+8xy+4y^{2}
\text{\quad subject to\quad} x^{2}+2xy+4y^{2}=1.
$$

\item\label{exer:22}
Find the extreme value  of $f(x,y)=\alpha+\beta xy$ subject to
$(ax+by)^{2}=1$.
 Assume that $ab\ne0$.

\item\label{exer:23}
Find the extreme values  of $f(x,y,z)=x+y^{2}+2z$ subject to
$$
4x^{2}+9y^{2}-36z^{2}=36.
$$

\item\label{exer:24}
Find the extreme values of $f(x,y,z,w)=(x+z)(y+w)$ subject to
$$
x^{2}+y^{2}+z^{2}+w^{2}=1.
$$

\item\label{exer:25}
Find the extreme values of $f(x,y,z,w)=(x+z)(y+w)$ subject to
$$
x^{2}+y^{2}=1 \text{\;and \;\;} z^{2}+w^{2}=1.
$$

\item\label{exer:26}
Find the extreme values of $f(x,y,z,w)=(x+z)(y+w)$ subject to
$$
x^{2}+z^{2}=1 \text{\;and \;\;} y^{2}+w^{2}=1.
$$

\item\label{exer:27}
Find the distance between the circle $x^{2}+y^{2}=1$
the hyperbola $xy=1$.

\item\label{exer:28}
Minimize
 $f(x,y,x)=\dst{\frac{x^{2}}{\alpha^{2}}+\frac{y^{2}}{\beta^{2}}
+\frac{z^{2}}{\gamma^{2}}}$\;
subject to $ax+by+cz=d$ and $x$, $y$, $z>0$.

\item\label{exer:29}
Find the distance  from
$(c_{1},c_{2},\dots,c_{n})$ to the plane
$$
a_{1}x_{1}+a_{2}x_{2}+\cdots+a_{n}x_{n}=d.
$$

\item\label{exer:30}
Find the maximum value of $f({\bf X})=\dst{\sum_{i=1}^{n}a_{i}x_{i}^{2}}$
subject to
$\dst{\sum_{i=1}^{n}b_{i}x_{i}^{4}}=1$, where $p,$ $q>0$ and
$a_{i}$,  $b_{i}$ $x_{i}>0$,
$1\le i\le n$.

\item\label{exer:31}
Find the extreme value of $f({\bf X})=\dst{\sum_{i=1}^{n}a_{i}x_{i}^{p}}$
subject to
$\dst{\sum_{i=1}^{n}b_{i}x_{i}^{q}}=1$, where $p$, $q$>0  and
 $a_{i}$,  $b_{i}$, $x_{i}>0$,
$1\le i\le n$.

\item\label{exer:32}
Find the minimum value of
$$
f(x,y,z,w)=x^{2}+2y^{2}+z^{2}+w^2
$$
subject to
\begin{eqnarray*}
x+y+\phantom{2}z+3w&=&1\\
x+y+2z+\phantom{3}w&=&2.
\end{eqnarray*}

\item\label{exer:33}
Find the minimum value of
$$
f(x,y,z)=
\frac{x^{2}}{a^{2}}+\frac{y^{2}}{b^{2}}+\frac{z^{2}}{c^{2}}
$$
subject to $p_{1}x+p_{2}y+p_{3}z=d$, assuming that at least one of
$p_{1}$,
 $p_{2}$, $p_{3}$ is nonzero.

\item\label{exer:34}
Find the extreme values of
$f(x,y,z)= p_{1}x+p_{2}y+p_{3}z$  subject to
$$
\frac{x^{2}}{a^{2}}+\frac{y^{2}}{b^{2}}+\frac{z^{2}}{c^{2}}=1,
$$
 assuming that at least one of
$p_{1}$,
 $p_{2}$, $p_{3}$ is nonzero.

\item\label{exer:35}
Find the distance from  $(-1,2,3)$  to the intersection of  the
planes \\$x+2y-3z=4$ and $2x-y+2z=5$.

\item\label{exer:36}
Find the extreme values  of $f(x,y,z)=2x+y+2z$  subject to $x^{2}+y^{2}=4$
and
$x+z=2$.

\item\label{exer:37}
Find the distance between the parabola $y=1+x^{2}$ and the
line $x+y=-1$.

\item\label{exer:38}
Find the distance between the ellipsoid
$$
3x^{2}+9y^{2}+6z^{2}=10
$$
and the plane
$$
3x+3y+6z=70.
$$

\item\label{exer:39}
Show that the extreme values of $f(x,y,z)=xy+yz+zx$  subject to
$$
\frac{x^{2}}{a^{2}}+\frac{y^{2}}{b^{2}}+\frac{z^{2}}{c^{2}}=1
$$
are  the largest and smallest eigenvalues of the matrix
$$
\left[\begin{array}{ccccccc}
0&a^{2}&a^{2}\\ b^{2}&0&b^{2}\\c^{2}&c^{2}&0
\end{array}\right].
$$

\item\label{exer:40}
Show that the extreme values of $f(x,y,z)=xy+2yz+2zx$  subject to
$$
\frac{x^{2}}{a^{2}}+\frac{y^{2}}{b^{2}}+\frac{z^{2}}{c^{2}}=1
$$
are the largest and smallest eigenvalues of the matrix
$$
\left[\begin{array}{ccccccc}
0&a^{2}/2&a^{2}\\
b^{2}/2&0&b^{2}\\c^{2}&c^{2}&0
\end{array}\right].
$$
\item\label{exer:41}
Find the extreme values of  $x(y+z)$   subject to
$$
\frac{x^{2}}{a^{2}}+\frac{y^{2}}{b^{2}}+\frac{z^{2}}{c^{2}}=1.
$$

\item\label{exer:42}
Let $a$, $b$, $c$,  $p$, $q$, $r$, $\alpha$, $\beta$, and
$\gamma$ be positive constants.
Find the maximum value of $f(x,y,z)=x^{\alpha}y^{\beta}z^{\gamma}$  subject
to
$$
ax^{p}+by^{q}+cz^{r}=1 \text{\; and\;\;} x,y,z>0 .
$$

\item\label{exer:43}
 Find the extreme values  of
$$
f(x,y,z,w)=xw-yz \text{\quad subject to\quad}
x^{2}+2y^{2}=4\text{\quad and\quad} 2z^{2}+w^{2}=9.
$$

\item\label{exer:44}
Let $a$, $b$, $c$,and $d$ be positive. Find  the extreme values   of
$$
f(x,y,z,w)=xw-yz
$$
subject to
$$
ax^{2}+by^{2}=1, \quad cz^{2}+dw^{2}=1,
$$
if {\bf(a)}  $ad\ne bc$; {\bf(b)} $ad=bc.$

\item\label{exer:45}
Minimize $f(x,y,z)=\alpha x^{2}+\beta y^{2}+\gamma z^{2}$  subject to
$$
a_{1}x+a_{2}y+a_{3}z=c\text{\; and\;\;} b_{1}x+b_{2}y+b_{3}z=d.
$$
Assume that
$$
\alpha,\beta,\gamma>0,\quad a_{1}^{2}+a_{2}^{2}+a_{3}^{2}\ne0,
\text{\; and\;\;}
b_{1}^{2}+b_{2}^{2}+b_{3}^{2}\ne 0.
$$
Formulate and apply a required additional assumption.

\item\label{exer:46}
Minimize $f({\bf X},{\bf Y})=\dst{\sum_{i=1}^{n}(x_{i}-\alpha_{i})^{2}}$
subject to
$$
\dst{\sum_{i=1}^{n}a_{i}x_{i}=c} \text{\; and\;\;}
\dst{\sum_{i=1}^{n}b_{i}x_{i}=d},
$$
 where
$$
\sum_{i=1}^{n}a_{i}^{2}=\sum_{i=1}^{n}b_{i}^{2}=1
\text{\; and\;\;}
\sum_{i=1}^{n}a_{i}b_{i}=0.
$$

\item\label{exer:47}
Find $(x_{10,x_{20}},\dots,x_{n0})$ to minimize
$$
Q(\mathbf{X})=\sum_{i=1}^{n}x_{i}^{2}
$$
subject to
$$
\sum_{i=1}^{n}x_{i}=1\text{\quad and\quad}  \sum_{i=1}^{n}ix_{i}=0.
$$
Prove explicitly that if
$$
\sum_{j=1}^{n}y_{i}=1,\quad   \sum_{i=1}^{n}iy_{i}=0
$$
and $y_{i}\ne x_{i0}$  for some $i\in\{1,2,\dots,n\}$,  then
$$
\sum_{i=1}^{n}y_{i}^{2}>\sum_{i=1}^{n}x_{i0}^{2}.
$$

\item\label{exer:48}
Let $p_{1}$, $p_{2}$, \dots, $p_{n}$ and $s$ be positive numbers.
Maximize
$$
f({\bf X})=
(s-x_{1})^{p_{1}}(s-x_{2})^{p_{2}}\cdots(s-x_{n})^{p_{n}}
$$
subject to  $x_{1}+x_{2}+\cdots+x_{n}=s$.

\item\label{exer:49}
Maximize $f({\bf X})=x_{1}^{p_{1}}x_{2}^{p_{2}}\cdots x_{n}^{p_{n}}$
subject to  $x_{i}>0$, $1\le i\le n$, and
$$
\sum_{i=1}^{n}\frac{x_{i}}{\sigma_{i}} = S,
$$
where $p_{1}$, $p_{2}$,\dots, $p_{n}$, $\sigma_{1}$, $\sigma_{2}$, \dots,
$\sigma_{n}$, and
$V$ are given positive numbers.

\item\label{exer:50}
Maximize
$$
 f({\bf X})=\sum_{i=1}^{n}\frac{x_{i}}{\sigma_{i}}
$$
subject to  $x_{i}>0$, $1\le i\le n$, and
$$
x_{1}^{p_{1}}x_{2}^{p_{2}}\cdots x_{n}^{p_{n}}=V,
$$
where $p_{1}$, $p_{2}$,\dots, $p_{n}$, $\sigma_{1}$, $\sigma_{2}$, \dots,
$\sigma_{n}$, and
$S$ are given positive numbers.

\item\label{exer:51}
Suppose that $\alpha_{1}$, $\alpha_{2}$, \dots $\alpha_{n}$ are positive
and at least one of $a_{1}$, $a_{2}$, \dots, $a_{n}$ is nonzero.
Let $(c_{1},c_{2},\dots,c_{n})$ be given.
 Minimize
$$
Q({\bf X})=\sum_{i=1}^{n}\frac{(x_{i}-c_{i})^{2}}{\alpha_{i}}
$$
subject to
$$
a_{1}x_{1}+a_{2}x_{2}+\cdots+a_{n}x_{n}=d.
$$

\item\label{exer:52}
Schwarz's inequality  says that $(a_{1},a_{2},\dots,a_{n})$  and
$(x_{1},x_{2},\dots,x_{n})$  are arbitrary $n$-tuples of real
numbers, then
$$
|a_{1}x_{1}+a_{2}x_{2}+\cdots+a_{n}x_{n}|\le
(a_{1}^{2}+a_{2}^{2}+ \cdots+ a_{n}^{2})^{1/2}
(x_{1}^{2}+x_{2}^{2}+ \cdots+ x_{n}^{2})^{1/2}.
$$
Prove this by finding the extreme values of
$f({\bf X})=\dst{\sum_{i=1}^{n}a_{i}x_{i}}$
subject to $\dst{\sum_{i=1}^{n}x_{i}^{2}}~=~\sigma^{2}$.

\item\label{exer:53}
Let $x_{1}$, $x_{2}$, \dots, $x_{m}$, $r_{1}$, $r_{2}$, \dots, $r_{m}$
be positive and
$$
r_{1}+r_{2}+\cdots+r_{m}=r.
$$
Show that
$$
\left(x_{1}^{r_{1}}x_{2}^{r_{2}}\cdots x_{m}^{r_{m}}\right)^{1/r}
 \le \frac{r_{1}x_{1}+r_{2}x_{2}+\cdots r_{m}x_{m}}{r},
$$
and give necessary and sufficient conditions for equality.
(Hint: Maximize $x_{1}^{r_{1}}x_{2}^{r_{2}}\cdots
x_{m}^{r_{m}}$ subject to $\sum_{j=1}^{m}r_{j}x_{j}=\sigma>0$,
$x_{1}>0$, $x_{2}>0$, \dots, $x_{m}>0$.)

\item\label{exer:54}
Let $\mathbf{A}=[a_{ij}]$ be an $m\times n$ matrix. Suppose that
$p_{1}$, $p_{2}$, \dots, $p_{m}>0$  and
$$
\sum_{j=1}^{m}\frac{1}{p_{j}}=1,
$$
and define
$$
\sigma_{i}=\sum_{j=1}^{n}|a_{ij}|^{p_{i}}, \quad 1 \le i \le m.
$$
Use Exercise~\ref{exer:53} to show that
$$
\left|\sum_{j=1}^{n}a_{ij}a_{2j}\cdots a_{mj}\right| \le
\sigma_{1}^{1/p_{1}}\sigma_{2}^{1/p_{2}}\cdots \sigma_{m}^{1/p_{m}}.
$$
(With $m=2$ this is
\href{http://www-history.mcs.st-and.ac.uk/Mathematicians/Holder.html}
{\emph{H\"{o}lder's}}
 \emph{inequality}, which  reduces to
Schwarz's
inequality if $p_{1}=p_{2}=2$.)
%###

\item\label{exer:55}
Let $c_{0}$, $c_{1}$, \dots, $c_{m}$  be given constants and $n\ge m+1$.
Show that the minimum value of
$$
Q({\bf X})=\sum_{r=0}^{n}x_{r}^{2}
$$
subject to
$$
\sum_{r=0}^{n}x_{r}r^{s}=c_{s},\quad 0\le s \le  m,
$$
is attained when
$$
x_{r}=\sum_{s=0}^{m}\lambda_{s}r^{s},\quad 0\le r\le n,
$$
where
$$
\sum_{\ell=0}^{m}\sigma_{s+\ell}\lambda_{\ell}=c_{s}
\text{\; and\;\;} \sigma_{s}= \sum_{r=0}^{n}r^{s},\quad 0\le s\le m.
$$
Show  that if
 $\{x_{r}\}_{r=0}^{n}$ satisfies the constraints and
$x_{r}\ne x_{r0}$ for some $r$, then
$$
\sum_{r=0}^{n}x_{r}^{2}>\sum_{r=0}^{n}x_{r0}^{2}.
$$

\item\label{exer:56}
Suppose that $n> 2k$. Show that the minimum value of
$f({\bf W})=\dst{\sum_{i=-n}^{n}w_{i}^{2}}$, subject to the constraint
$$
\sum_{i=-n}^{n}w_{i}P(r-i)=P(r)
$$
whenever $r$  is an integer and $P$ is a polynomial of degree $\le 2k$,
is attained with
$$
w_{i0}=\sum_{r=0}^{2k}\lambda_{r}i^{r},\quad 1\le i\le n,
$$
where
$$
\sum_{r=0}^{2k}\lambda_{r}\sigma_{r+s}=
\begin{cases} 1& \text{if } s=0,\\ 0&\text{if  }1\le s\le 2k,  \end{cases}
\text{\; and\;\;}
 \sigma_{s}=\sum_{j=-n}^{n}j^{s}.
$$
Show  that if
 $\{w_{i}\}_{i=-n}^{n}$ satisfies the constraint and
$w_{i}\ne w_{i0}$ for some $i$, then
$$
\sum_{i=-n}^{n}w_{i}^{2}>\sum_{i=-n}^{n}w_{i0}^{2}.
$$

\item\label{exer:57}
Suppose that $n\ge k$. Show that the minimum value of
$f\dst{\sum_{i=0}^{n}w_{i}^{2}}$, subject to the constraint
$$
\sum_{i=0}^{n}w_{i}P(r-i)=P(r+1)
$$
whenever $r$  is an integer and $P$ is a polynomial of degree $\le k$,
is attained with
$$
w_{i0}=\sum_{r=0}^{k}\lambda_{r}i^{r},\quad 0\le i\le n,
$$
where
$$
\sum_{r=0}^{k}\sigma_{r+s}\lambda_{r}=(-1)^{s},\quad 0\le s \le k,
\text{\quad and\quad }  \sigma_{\ell}=\sum_{i=0}^{n}i^{\ell},\quad 0\le
\ell\le 2k.
$$
Show  that if
$$
\sum_{i=0}^{n}u_{i}P(r-i)=P(r+1)
$$
whenever $r$  is an integer and $P$ is a polynomial of degree $\le k$,
and
$u_{i}\ne w_{i0}$ for some $i$, then
$$
\sum_{i=0}^{n}u_{i}^{2}>\sum_{i=0}^{n}w_{i0}^{2}.
$$

\item\label{exer:58}
Minimize
$$
f({\bf X})=\sum_{i=1}^{n}\frac{(x_{i}-c_{i})^{2}}{\alpha_{i}}
$$
subject to
$$
\sum_{i=1}^{n}a_{ir}x_{i}=d_{r},\quad  1\le r \le m
$$
Assume that  $m>1$, $\alpha_{1}$, $\alpha_{2}$, \dots $\alpha_{m}>0$,
and
$$
\sum_{i=1}^{n}\alpha_{i}a_{ir}a_{is}=
\begin{cases}
1 & \text{ if } r=s,\\0 & \text{ if }r\ne s.
\end{cases}
$$

\end{exerciselist}

\newpage
\setlength{\parindent}{0pt}
\section{Answers to selected exercises}\label{section:8}

\medskip

{\bf \ref{exer:1}.}
$\left(\frac{15}{7} -\frac{2}{7},\frac{25}{7}\right)$
\quad
{\bf \ref{exer:2}.} $\pm5$
\quad
{\bf \ref{exer:3}.} $1/\sqrt{ab}$, $-1/\sqrt{ab}$

\medskip

{\bf \ref{exer:4}.}  $(8,16)$ is closest, $(-8,-16)$ is farthest.
\quad
{\bf \ref{exer:5}.}  $\pm\sqrt{53/6}$
\quad
{\bf \ref{exer:6}.} $1/4ab$
\quad
{\bf \ref{exer:7}.} $p^{2}/4$

\medskip
{\bf \ref{exer:8}.}  $4\sqrt{A}$
\quad
{\bf \ref{exer:9}.} $A^{3/2}/6\sqrt{6}$
\quad
{\bf \ref{exer:10}.} $A^{3/2}/6\sqrt{3}$
\quad
{\bf \ref{exer:11}.} $3(2V)^{2/3}$
\quad
{\bf \ref{exer:12}.} $abc/27$

\medskip
{\bf \ref{exer:13}.} $ab$
\quad
{\bf \ref{exer:15}.} $8abc/3\sqrt{3}$

\quad

{\bf \ref{exer:18}.}
$(1-\mu)^{2}$ and  $(1+\mu)^{2}$, where
$\mu =\dst{\left(\sum_{j=1}^{n}c_{j}^{2}\right)^{1/2}}$
\quad
{\bf \ref{exer:19}.} $-1$, $2$
\quad
{\bf \ref{exer:20}.} $2$, $4$

\medskip
{\bf \ref{exer:21}.} $-2/3$, $2$
\quad
{\bf \ref{exer:22}.} $\alpha\pm|\beta|/4|ab|$
\quad
{\bf \ref{exer:23}.} $-\sqrt{5}$, $73/16$
\quad
{\bf \ref{exer:24}.}  $\pm1$
\quad
{\bf \ref{exer:25}.}   $\pm2$

\medskip
{\bf \ref{exer:26}.}   $\pm2$
\quad
{\bf \ref{exer:27}.}  $\sqrt2-1$
\quad
{\bf \ref{exer:28}.}
$\dst{\frac{d^{2}}{(a\alpha)^{2}+(b\beta^{2})+(c\gamma)^{2}}}$

\medskip
{\bf \ref{exer:29}.}
$\dst{\frac{|d-a_{1}c_{1}-a_{2}c_{2}-\cdots-a_{n}c_{n})a_{i}|}
{\sqrt{a_{1}^{2}+a_{2}^{2}+\cdots a_{n}^{2}}}}$
\quad
{\bf \ref{exer:30}.}
$\dst{\left(\sum_{i=1}^{n}\frac{a_{i}^{2}}{b_{i}}\right)^{1/2}}$


{\bf \ref{exer:31}.}
$\dst{\left(\sum_{i=1}^{n}a_{i}^{q/(q-p)} b_{i}^{p/(p-q)}\right)^{1-p/q}}$
is a constrained maximum if $p<q$, a constrained minimum if $p>q$

\medskip
{\bf \ref{exer:32}.} $689/845$
\quad
{\bf \ref{exer:33}.}
$\dst{\frac{d^{2}}{p_{1}^{2}a^{2}+p_{2}^{2}b^{2}+p_{3}^{2}c^{2}}}$
\quad
{\bf \ref{exer:34}.} $\pm (p_{1}^{2}a^{2}+p_{2}^{2}b^{2}+p_{3}^{2}c^{2})^{1/2}$
\quad


\medskip

{\bf \ref{exer:35}.}
\quad
$\sqrt{693/45}$
{\bf \ref{exer:36}.}  $2$, $6$
\quad
{\bf \ref{exer:37}.} $7/4\sqrt{2}$
\quad
{\bf \ref{exer:38}.} $10\sqrt{6}/3$
\quad
{\bf \ref{exer:41}.}
$\pm|c|\sqrt{a^{2}+b^{2}}/2$

\medskip
{\bf \ref{exer:42}.}
$\dst{\frac{\alpha\beta\gamma}{pqr}
\left(\frac{\alpha}{p}+\frac{\beta}{q}+\frac{\gamma}{r}\right)^{-3}}$
{\bf \ref{exer:43}.}     $\pm3$
\quad
{\bf \ref{exer:44}.} {\bf (a)} $\pm1/\sqrt{bc}$
{\bf (b)} $\pm1/\sqrt{ad}=\pm1/\sqrt{bc}$

\medskip

{\bf \ref{exer:46}.}
$\dst{\left(c-\sum_{i=1}^{n}a_{i}\alpha_{i}\right)^{2}
+\left(d-\sum_{i=1}^{n}b_{i}\alpha_{i}\right)^{2}}$
\quad
{\bf \ref{exer:47}.} $x_{i0}=(4n+2-6i)/n(n-1)$

\medskip

{\bf \ref{exer:48}.}
$\left[\frac{(n-1)s}{P}\right]^{P}p_{1}^{p_{1}}p_{2}^{p_{2}}\cdots
p_{n}^{p_{n}}$

\medskip

{\bf \ref{exer:49}.}
$\dst{\left(\frac{S}{p_{1}+p_{2}+\cdots+
p_{n}}\right)^{p_{1}+p_{2}+\cdots+p_{n}}
(p_{1}\sigma_{1})^{p_{1}}
(p_{2}\sigma_{2})^{p_{2}} \cdots
(p_{n}\sigma_{n})^{p_{n}}}$

\medskip

{\bf \ref{exer:50}.}
$\dst{(p_{1}+p_{2}+\cdots+p_{n})
\left(\frac{V}{(\sigma_{1}p_{1})^{p_{1}}(\sigma_{2}p_{2})^{p_{2}}
\cdots (\sigma_{ n}p_{n})^{p_{n}}}\right)^{\frac{1}{p_{1}+p_{2}+\cdots+p_{n}}}}$


{\bf \ref{exer:51}.}
$\dst{\left(d-\sum_{i=1}^{n}a_{i}c_{i}\right))^{2}/
\left(\sum_{i=1}^{n}a_{i}^{2}\alpha_{i}\right)}$
\quad
{\bf \ref{exer:52}.}
$\dst{\pm\left(\sum_{i=1}^{n}a_{i}^{2}\right)^{1/2}
\left(\sum_{i=1}^{n}x_{i0}^{2}\right)^{1/2}}$

\medskip
{\bf \ref{exer:58}.} $\dst\sum_{r=1}^{m}
\left(d_{r}-\sum_{i=1}^{n}a_{ir}c_{i}\right)^{2}$

\enlargethispage{\baselineskip}

\end{document}
\newpage


\noindent

\thispagestyle{empty}
\bf
\begin{center}
{\Large INSTRUCT0R'S SOLUTIONS MANUAL}

\medskip

{\Huge THE METHOD OF \\ \medskip LAGRANGE MULTIPLIERS}

\vspace{1in}
\huge
\href{http://ramanujan.math.trinity.edu/wtrench/index.shtml}
{William F. Trench}
\\ \medskip\large
Professor Emeritus\\
Department of Mathematics\\
Trinity University \\
San Antonio, Texas, USA\\
\href{mailto:wtrench@trinity.edu}
{wtrench@trinity.edu}
\vspace*{1in}

\bigskip


\medskip

\end{center}

\noindent
{\bf \copyright Copyright November 2012  William F. Trench, all rights
reserved.
 No part of this document may  be circulated or  posted on any
website without the author's  permission. Under US copyright law,}
\medskip

{\bf
\begin{quote}``Uploading or downloading works protected by copyright
without the authority of the
copyright owner is an infringement of the copyright owner's exclusive  rights  of
reproduction and/or distribution. Anyone found to have infringed a copyrighted work may be
liable for statutory damages up to \$30,000 for each work infringed and, if willful
infringement is proven by the copyright owner, that amount may be increased up to \$150,000
for each work infringed. In addition, an infringer of a work may also be liable for the
attorney's fees incurred by the copyright owner to enforce his or her rights.''
\end{quote}}

\rm

\newpage

\setlength{\parindent}{0pt}
 \medskip
\centerline{\bf SOLUTIONS OF  EXERCISES}

\bigskip
{\bf \ref{exer:1}.}
\quad
\quad $L=\dst{\frac{(x-1)^{2}+(y+2)^{2}+(z-3)^{2}}{2}-\lambda(2x+3y+z)}$
$$
L_{x}=x-1-2\lambda,\quad
L_{y}=y+2-3\lambda, \quad
L_{z}=z-3-\lambda
$$
$$
x_{0}=1+2\lambda,  \quad
y_{0}=-2+3\lambda,  \quad
z_{0}=3+\lambda \quad
$$
$$
2(1+2\lambda)+3(-2+3\lambda)+(3+\lambda)=7, \quad
\lambda=\dst{\frac{4}{7}}
$$
$$
x_{0}=\dst{\frac{15}{7}}, \quad
y_{0}=-\dst{\frac{2}{7}},\quad  z_{0}=\dst{\frac{25}{7}}
$$
The distance from $(x_{01},y_{01},z_{01})$ to the plane is
$$
\sqrt{(x_{0}-1)^{2}+(y_{0}+2)^{2}+(z_{0}-3)^{2}}=
\sqrt{4\lambda^{2}+9\lambda^{2}+\lambda^{2}}=4\sqrt{\frac{2}{7}}.
$$

\bigskip
{\bf \ref{exer:2}.}
\centerline{$L=2x+y-\dst{\frac{\lambda}{2}}(x^{2}+y^{2})$,\quad
$L_{x}=2-\lambda x$,\quad
$L_{y}=1-\lambda y$}
$$
 x_{0}=2y_{0},\quad  5y_{0}^{2}=5, \quad
(x_{0},y_{0})=\pm(2,1)
$$
Constrained minimum $=-5$, constrained maximum
$=5$.

\bigskip
{\bf \ref{exer:3}.}
\centerline{$L=\beta x+\alpha y-\dst{\frac{\lambda}{2}}(ax^{2}+by^{2})$}
$$
L_{x}=\beta-\lambda  ax,\quad
L_{y}=\alpha-\lambda by,\quad
x_{0}=\dst{\frac{\beta}{\lambda a}}, \quad
y_{0}=\dst{\frac{\alpha}{\lambda b}}
$$
$$
1=ax_{0}^{2}+by_{0}^{2}=\frac{1}{\lambda^{2}}
\left(\frac{\beta^{2}}{a}+\frac{\alpha^{2}}{b}\right)
=\frac{1}{ab\lambda^{2}}(a\alpha^{2}+b\beta^{2})=\frac{1}{ab\lambda^{2}}.
$$
$\dst{\frac{1}{\lambda}}=\pm\sqrt{ab}$;
$(x_{0},y_{0})=\pm\dst{\left(\beta\sqrt{\frac{b}{a}},\alpha\sqrt\frac{a}{b}\right)}$.
Choosing ``$+$'' yields the constrained maximum
$$
f(x_{0},y_{0})=\beta^{2}\sqrt{\frac{b}{a}}+\alpha^{2}\sqrt{\frac{a}{b}}
=\frac{b\beta^{2}}{\sqrt{ab}}+\frac{a\alpha^{2}}{\sqrt{ab}}=\frac{1}{\sqrt{ab}}.
$$
Choosing ``$-$'' yields the constrained minimum
$-\dst{\frac{1}{\sqrt{ab}}}$.

\bigskip
{\bf \ref{exer:4}.}
\centerline{$L=\dst{\frac{(x-2)^{2}+(y-4)^{2}-\lambda(x^{2}+y^{2})}{2}}$}
$$
L_{x}(x,y)=(x-2)-\lambda x,\quad
L_{y}(x,y)=(y-4)-\lambda y
$$
$$
\frac{x_{0}-2}{x_{0}}=\frac{y_{0}-4}{y_{0}}=\lambda, \text{\; so\;\;}
y_{0}=2x_{0}.
$$
 Therefore, $x_{0}^{2}+y_{0}^{2}=5x_{0}^{2}=320$,
$(x_{0},y_{0})=\pm(8,16)$ so the constrained critical
points
are $(8,16)$ and $(-8,-16)$; $(8,16)$ is closest to $(2,4)$ and $(-8,-16)$
is farthest.

\bigskip
{\bf \ref{exer:5}.}
\centerline{$L=2x+3y+z-\dst{\frac{\lambda}{2}}(x^{2}+2y^{2}+3z^{2})$}
$$
L_{x}=2-\lambda x,\quad  L_{y}=3-2\lambda y,\quad  L_{z}=1-3\lambda z
$$
$$
x_{0}=\frac{2}{\lambda}\quad  y_{0}=\frac{3}{2\lambda},  \quad
z_{0}=\frac{1}{3\lambda}, \quad
x_{0}^{2}+2y_{0}^{2}+3z_{0}^{2}=\dst{\frac{53}{6\lambda^{2}}}=1,\quad
\lambda=\pm\sqrt{53/6}.
$$
Since $f(2/\lambda,3/2\lambda,1/3\lambda)=\dst{\frac{53}{6\lambda}}=\pm
\lambda$, the constrained extreme values are  $\pm\sqrt{53/6}$.

\bigskip
{\bf \ref{exer:6}.}
\centerline{$L=xy-\lambda (ax+by)$, $L_{x}(x,y)=y-\lambda a$,
$L_{y}=x-\lambda b$}
$$
x_{0}=\lambda b,\quad  y_{0}=\lambda a, \quad ax_{0}+by_{0}=2\lambda
ab=1,\quad
\lambda=\frac{1}{2ab}
$$
$$
 x_{0}=\frac{1}{2a},\quad
y_{0}=\frac{1}{2b},\quad
x_{0}y_{0}=\frac{1}{4ab}=\text{constrained maximum\;\;}
$$

\bigskip
{\bf \ref{exer:7}.}
$p=2x+2y$, $A=xy$,
$L=xy-\lambda(x+y)$, $L_{x}=y-\lambda$, $L_{y}=x-\lambda$,
$y_{0}=x_{0}$,
 $x_{0}=p/4$, $A_{\text max}=p^{2}/4$.

\bigskip
{\bf \ref{exer:8}.}
Let $x$ and $y$ denote lengths of sides. We must mimimize
$x+y$  subject to $xy=A$.
$$
L=x+y-\lambda xy,\quad L_{x}=1-\lambda y,\;  L_{y}=1-\lambda x, \;
x_{0}=y_{0},\; x_{0}y_{0}=A,\; x_{0}=\sqrt{A}.
$$
The minimum perimeter is $4\sqrt{A}$.

\bigskip
{\bf \ref{exer:9}.}
Denote the vertices of the box by $(0,0,0)$, $(x,0,0)$, $(0,y,0)$, and
$(0,0,z)$.
$$
V=xyz,\quad  A=2xz+2yz +2xy,\quad
L=xyz-\lambda(xz+yz+xy)
$$
$$
L_{x}=yz-\lambda(z+y),\quad
 L_{y}=xz-\lambda(z+x), \quad
L_{z}=xy-\lambda(x+y)
$$
$$
y_{0}z_{0}=\lambda(z_{0}+y_{0}),\quad
x_{0}z_{0}=\lambda(z_{0}+x_{0}), \quad
x_{0}y_{0}=\lambda(x_{0}+y_{0})
$$
$$
x_{0}z_{0}+x_{0}y_{0}=z_{0}y_{0}+x_{0}y_{0}
=x_{0}z_{0}+y_{0}z_{0},\quad
x_{0}=y_{0}=z_{0}
$$
$$
A=6z_{0}^{2}, \quad
z=\sqrt{\frac{A}{6}},\quad
V_{\text{max}}=z_{0}^{3}=\dst{\frac{A^{3/2}}{6\sqrt{6}}}.
$$

\bigskip
{\bf \ref{exer:10}.}
Denote the vertices of the box by
 $(0,0,0)$, $(x,0,0)$, $(0,y,0)$, and
$(0,0,z)$.
$$
V=xyz,\quad  A=2xz+2yz+xy,\quad
L=xyz-\lambda(2xz+2yz+xy)
$$
$$
L_{x}=yz-\lambda(2z+y),\quad   L_{y}=xz-\lambda(2z+x), \quad
L_{z}=xy-\lambda(2x+2y)
$$
$$
y_{0}z_{0}=\lambda(2z_{0}+y_{0}),\quad
x_{0}z_{0}=\lambda(2z_{0}+x_{0}),\quad
x_{0}y_{0}=\lambda(2x_{0}+2y_{0})
$$
$$
x_{0}y_{0}z_{0}=\lambda x_{0}(2z_{0}+y_{0}),\quad
x_{0}y_{0}z_{0}=\lambda y_{0}(2z_{0}+x_{0}),\quad
x_{0}y_{0}z_{0}=\lambda z_{0}(2x_{0}+2y_{0})
$$
$$
2x_{0}z_{0}+x_{0}y_{0}=2y_{0}z_{0}+x_{0}y_{0}=2x_{0}z_{0}+2y_{0}z_{0}
$$
$$
x_{0}=y_{0}=2z_{0}, \quad
A=12z_{0}^{2},  \quad
z_{0}=\sqrt{\frac{A}{12}},\quad
V_{\text{max}}=z_{0}^{3}=\dst{\frac{A^{3/2}}{6\sqrt{3}}}.
$$

\bigskip
{\bf \ref{exer:11}.}
Denote the vertices of the box by  $(0,0,0)$, $(x,0,0)$, $(0,y,0)$, and
$(0,0,z)$.
$$
 V=xyz,\quad   A=2xz+2yz+xy, \quad
L=2xz+2yz+xy-\lambda xyz,\quad
$$
$$
L_{x}=2z+y-\lambda yz, \quad  L_{y}=2z+x-\lambda xz, \quad
L_{z}=2x+2y-\lambda xy
$$
$$
2z_{0}+y_{0}=\lambda y_{0}z_{0}, \quad  2z_{0}+x_{0}=\lambda x_{0}z_{0},
\quad 2x_{0}+2y_{0}-\lambda x_{0}y_{0}
$$
$$
2x_{0}z_{0}+x_{0}y_{0}=2y_{0}z_{0}+x_{0}y_{0}=2x_{0}z_{0}+2y_{0}z_{0}
$$
$$
 x_{0}=y_{0}=2z_{0},\; V=4z_{0}^{3}, \;
z_{0}=\frac{(2V)^{1/3}}{2},\;
x_{0}=y_{0}=(2V)^{1/3},\;  A_\text{min}=3(2V)^{2/3}
$$


\bigskip
{\bf \ref{exer:12}.}
$L=xyz-\lambda\dst{\left(\dst{\frac{x}{a}+\frac{y}{b}+\frac{z}{c}}\right)}$,\;
$L_{x}=yz-\dst{\frac{\lambda}{a}}$,\;
$L_{y}=xz-\dst{\frac{\lambda}{b}}$, \;
 $L_{z}=xy-\dst{\frac{\lambda}{c}}$
$$
 y_{0}z_{0}=\dst{\frac{\lambda}{a}},\quad
x_{0}z_{0}=\dst{\frac{\lambda}{b}}, \quad
x_{0}y_{0}=\dst{\frac{\lambda}{c}},\quad
\dst{\frac{x_{0}}{a}} =
\dst{\frac{y_{0}}{b}}=\dst{\frac{z_{0}}{c}}=\dst{\frac{1}{3}},\quad
V_{\text{max}}=\frac{abc}{27}.
$$

\bigskip
{\bf \ref{exer:13}.}
We may assume without loss of generality that $y>0$, so $A=ay$.
$$
L=\dst{ay-\frac{\lambda}{2}\left(\frac{x^{2}}{a^{2}}+\frac{y^{2}}{b^{2}}\right)},\quad
\dst{L_{x}=\frac{\lambda x}{a^{2}}},\quad
x_{0}=0,\quad  y_{0}=b,\quad   A_{\text{max}}=ab.
$$

\bigskip
{\bf \ref{exer:14}.}
We must maximize $A^{2}=s(s-x)(s-y)(s-z)$ subject to
$x+y+z=s$.
$$
L=-s(s-x)(s-y)(s-z)-\lambda(x+y+z)
$$
$$
L_{x}=s(s-y)(s-z)-\lambda,\quad
L_{y}=s(s-x)(s-z)-\lambda,\quad
L_{z}=s(s-x)(s-y)-\lambda\quad
$$
$$
s(s-y_{0})(s-z_{0})=
s(s-x_{0})(s-z_{0})=
s(s-x_{0})(s-y_{0})=\lambda,\quad
x_{0}=y_{0}=z_{0}=\frac{s}{3}.
$$

\bigskip
{\bf \ref{exer:15}.}
We must maximize $V=8xyz$ subject to   \quad
$\dst{\frac{x^{2}}{a^{2}}+\frac{y^{2}}{b^{2}}+\frac{z^{2}}{c^{2}}}=1.$
$$
L=xyz-\frac{\lambda}{2}\left(\frac{x^{2}}{a^{2}}+\frac{y^{2}}{b^{2}}+\frac{z^{2}}{c^{2}}\right)
$$
$$
L_{x}=yz-\frac{\lambda x}{a^{2}}, \quad
L_{y}=xz-\frac{\lambda y}{b^{2}}, \quad
L_{z}=xy-\frac{\lambda z}{c^{2}}
$$
$$
y_{0}z_{0}=\frac{\lambda x_{0}}{a^{2}}, \quad
x_{0}z_{0}=\frac{\lambda y_{0}}{b^{2}}, \quad
x_{0}y_{0}=\frac{\lambda z}{c^{2}}\quad
$$
$$
\dst{\frac{x_{0}^{2}}{a^{2}}}=\dst{\frac{y_{0}^{2}}{b^{2}}}=\dst{\frac{z_{0}^{2}}{c^{2}}}
=\lambda x_{0}y_{0}z_{0}
$$
To satisfy the constraint,
$x_{0}=\dst{\frac{a}{\sqrt{3}}}$,
$y_{0}=\dst{\frac{b}{\sqrt{3}}}$,
$z_{0}=\dst{\frac{c}{\sqrt{3}}}$,
so $V_{\text max}=\dst{\frac{8abc}{3\sqrt{3}}}$.

\bigskip
{\bf \ref{exer:16}.}
Let $(x_{0},y_{0},z_{0})$  be the point on the plane closest to
$(x_{1},y_{1},z_{1})$, so
\begin{equation} \tag{A}
ax_{0}+by_{0}+cz_{0}=\sigma.
\end{equation}
$$
L=\dst{\frac{(x-x_{1})^{2}+(y-y_{1})^{2}+(z-z_{1})^{2}}{2}}-\lambda(ax+by+cz)
$$
$$
L_{x}=(x-x_{1})-\lambda a,\quad
L_{y}=(y-y_{1})-\lambda b,\quad
L_{z}=(z-z_{1})-\lambda c
$$
\begin{equation}
x_{0}=x_{1}+\lambda a,\quad y_{0}=y_{1}+\lambda b,\text{\quad and\quad}
z_{0}=z_{1}+\lambda c, \tag{B}
\end{equation}
\begin{equation} \tag{C}
d^{2}=\lambda^{2}(a^{2}+b^{2}+c^{2})
\end{equation}
 (A) and (B) imply that
$$
ax_{1}+by_{1}+cz_{1}+\lambda(a^{2}+b^{2}+c^{2})=\sigma,
$$
so
$$
\lambda=\frac{\sigma-ax_{1}-by_{1}-cz_{1}}{a^{2}+b^{2}+c^{2}},
$$
and  (C) implies that
$$
d=\frac{|\sigma-ax_{1}-by_{1}-cz_{1}|}{\sqrt{a^{2}+b^{2}+c^{2}}}.
$$

\bigskip
{\bf \ref{exer:17}.}
\quad $\dst{
L=\frac{1}{2}\sum_{i=1}^{n}\left[(x-x_{i})^{2}+(y-y_{i})^{2}+(z-z_{i})^{2}\right]
-\lambda(ax+by+cz)}$
$$
L_{x}=nx - \lambda a -\sum_{i=1}^{n}x_{i},\quad
L_{y}=ny - \lambda b -\sum_{i=1}^{n}y_{i},\quad
L_{z}=nz - \lambda b -\sum_{i=1}^{n}z_{i}.
$$
$$
x_{0}=\dst{\frac{1}{n}\left[\lambda a+\sum_{i=1}^{n}x_{i}\right]},\quad
y_{0}=\dst{\frac{1}{n}\left[\lambda b+\sum_{i=1}^{n}y_{i}\right]},\quad
z_{0}=\dst{\frac{1}{n}\left[\lambda c+\sum_{i=1}^{n}z_{i}\right]}
$$
$$
ax_{0}+by_{0}+cz_{0}=\frac{1}{n}
\left[\lambda(a^{2}+b^{2}+c^{2})+\sum_{i=1}^{n}(ax_{i}+by_{i}+cz_{i})\right]
$$
Since $ax_{0}+by_{0}+cz_{0}=\sigma$,
$$
 \lambda=(a^{2}+b^{2}+c^{2})^{-1}\dst{\sum_{i=1}^{n}
(\sigma-ax_{i}-by_{i}-cz_{i})}.
$$

\bigskip
{\bf \ref{exer:18}.}
$L=\dst{\frac{1}{2}}\dst\left({\sum_{i=1}^{n}(x_{i}-c_{i})^{2}-
\lambda\sum_{i=1}^{n}x_{i}^{2}}\right)$,\,
$L_{x_{i}}=x_{i}-c_{i}-\lambda x_{i}$,\, $x_{i0}=(1-\lambda)^{-1} c_{i}$
\medskip

$\dst{\sum_{i=1}^{n}x_{i0}^{2}=(1-\lambda)^{-2}\sum_{j=1}^{n}c_{j}^{2}}=1$,
so
$\lambda =1\pm \mu$ where
$\mu =\dst{\left(\sum_{j=1}^{n}c_{j}^{2}\right)^{1/2}}$
Since   $x_{i0}=c_{i}+\lambda x_{i0}$  and
$\dst{\sum_{i=1}^{n}x_{i0}^{2}}=1$,
$\dst{\sum_{i=1}^{n}(x_{i0}-c_{i})^{2}=\lambda^{2}}$.  Since
$x_{i0}=(1-\lambda)^{-1}c_{i}$, the constrained maximum is $(1+\mu)^{2}$,
attained with $x_{i0}=-c_{i}/\mu$, $1\le i\le n$, and the constrained
 minimum is $(1-\mu)^{2}$,
attained with $x_{i0}=c_{i}/\mu$, $1\le i\le n$.

{\bf \ref{exer:19}.}
\centerline{$L=xy+xz+yz-\dst{\frac{\lambda}{2}(x^{2}+y^{2}+z^{2})}$}
$$
L_{x}=y+z-\lambda x,\quad  L_{y}=x+z-\lambda y,\quad
L_{z}=x+y-\lambda z
$$

$$
\left[\begin{array}{ccccccc}
0&1&1\\1&0&1\\1&1&0
\end{array}\right]
\left[\begin{array}{ccccccc}
x_{0}\\y_{0}\\z_{0}
\end{array}\right]=\lambda
\left[\begin{array}{ccccccc}
x_{0}\\y_{0}\\z_{0}
\end{array}\right].
$$
The eigenvalues of the matrix are $2$ and $-1$, which are therefore the
extremes  of $Q$ subject to the constraint.

\bigskip
{\bf \ref{exer:20}.}
\quad \quad \quad
\centerline{$L=\dst{\frac{3x^{2}+2y^{2}+3z^{2}+2xz}{2}}-\dst{\frac{\lambda}{2}(x^{2}+y^{2}+z^{2})}$}
$$
L_{x}=3x+z-\lambda x,\quad  L_{y}=2y-\lambda y,\quad
L_{z}=3z+x-\lambda z
$$

$$
\left[\begin{array}{ccccccc}
3&0&1\\0&2&0\\1&0&3
\end{array}\right]
\left[\begin{array}{ccccccc}
x_{0}\\y_{0}\\z_{0}
\end{array}\right]=\lambda
\left[\begin{array}{ccccccc}
x_{0}\\y_{0}\\z_{0}
\end{array}\right]
$$
The largest and smallest eigenvalues of the matrix are $4$ and $2$, which
are therefore the extremes  of $Q$ subject to the constraint.

\bigskip
{\bf \ref{exer:21}.}
\quad \quad
\quad $L=\dst{\frac{x^{2}+8xy+4y^{2}-\lambda(x^{2}+2xy+4y^{2})}{2}}$
\begin{eqnarray*}
L_{x}&=&(x+4y)-\lambda(x+y)=(1-\lambda)x+(4-\lambda)y   \\
L_{y}&=& (4x+4y)-\lambda(x+4y)=(4-\lambda)x+4(1-\lambda y)
\end{eqnarray*}
$$
\left[\begin{array}{ccccccc}
1-\lambda & 4-\lambda \\ 4-\lambda &4(1-\lambda)
\end{array}\right]
\left[\begin{array}{ccccccc}
x_{0}\\ y_{0}
\end{array}\right]=
\left[\begin{array}{ccccccc}
0\\0
\end{array}\right],
$$
so
$$
4(\lambda-1)^{2}-(\lambda-4)^{2}=3(\lambda-2)(\lambda+2)=0.
$$
If $\lambda=2$,  then  $x_{0}=2y_{0}$. To satisfy the constraint,
$(x_{0},y_{0})=\pm\left(\frac{1}{\sqrt3},\frac{1}{2\sqrt3}\right)$  and
$f(x_{0},y_{0})=2$.
If $\lambda=-2$, then  $x_{0}=-2y_{0}$. To satisfy the constraint,
$(x_{0},y_{0})=\pm\left(-\frac{1}{\sqrt3},\frac{1}{2\sqrt3}\right)$  and
$f(x_{0},y_{0})=-\frac{2}{3}$.

\bigskip
{\bf \ref{exer:22}.}
$L=\alpha+\beta xy-\dst{\frac{\lambda}{2}(ax+by)^{2}}$,\,
$L_{x}=\beta y-\lambda a(ax+by)$,\,
$L_{y}=\beta x-\lambda b(ax+by)$

\centerline{$x_{0}=\dst{\frac{\lambda b(ax_{0}+by_{0})}{\beta}}$,\quad
$y_{0}=\dst{\frac{\lambda a(ax_{0}+by_{0})}{\beta}}$,\quad
$(x_{0},y_{0})=\dst{\pm\left(\frac{\lambda
b}{\beta},\frac{\lambda a}{\beta}\right)}$}\quad

\centerline{$ax_{0}+by_{0}=\dst{\frac{2\lambda ab}{\beta}=\pm1}$,\quad
$\lambda=\dst{\pm\dst{\frac{\beta}{2ab}}}$,\quad
$(x_{0},y_{0})=\dst{\pm\left(\frac{1}{2a},\frac{1}{2b}\right)}$}

\centerline{$(\alpha+\beta
x_{0}y_{0})_\text{max}=\dst{\alpha+\frac{|\beta|}{4|ab|}}$,\quad
$(\alpha+\beta
x_{0}y_{0})_\text{min}=\alpha-\dst{\frac{|\beta|}{4|ab|}}$}


\bigskip
{\bf \ref{exer:23}.}
\centerline{$L=x+y^{2}+2z-\dst{\frac{\lambda}{2}(4x^{2}+9y^{2}-36z^{2})}$}
$$
L_{x}=1-4\lambda x,\;
L_{y}=2y-9\lambda y, \;
L_{z}=2+36\lambda z
$$
$$
x_{0}=\dst{\frac{1}{4\lambda}},\;
z_{0}=-\dst{\frac{1}{18\lambda}}=-\frac{2}{9}x_{0},
$$
  and
either $y_{0}=0$ or $\lambda=\dst{\frac{2}{9}}$.
If $y_{0}=0$,  then
$$
36=4x_{0}^{2}-36z_{0}^{2}=\left(4-36\left(\frac{2}{9}\right)^{2}\right)x_{0}^{2}
=\frac{20}{9}x_{0}^{2},\text{\quad so\quad}
(x_{0},z_{0})=\pm \left(\frac{9}{\sqrt{5}},-\frac{2}{\sqrt{5}}\right)
$$
and $f(x_{0},0,z_{0})=x_{0}+2z_{0}=\pm\sqrt{5}$.
If $\lambda=\dst{\frac{2}{9}}$, then $x_{0}=\dst{\frac{9}{8}}$
and
$z_{0}=-\dst{\frac{1}{4}}$, so
$$
9y_{0}^{2}=36(1+z_{0}^{2})-4x_{0}^{2}=36\left(1+\frac{1}{16}\right)-4\left(\frac{81}{64}\right)
=\frac{531}{16},
\text{\; so\;\;}y_{0}=\pm\dst{\frac{\sqrt{59}}{4}}.
$$
Therefore, the constrained maximum is
$f\left(\frac{9}{8},\frac{\sqrt{59}}{4},-\frac{1}{4}\right)=
f\left(\frac{9}{8},-\frac{\sqrt{59}}{4},-\frac{1}{4}\right)=\frac{73}{16}$
and the constrained minimum is
$f\left(-\frac{9}{\sqrt{5}},0,\frac{2}{\sqrt{5}}\right)=-\sqrt{5}$.

\bigskip
{\bf \ref{exer:24}.}
\quad \quad \quad \quad
$L=(x+z)(y+w)-\dst{\frac{\lambda}{2}(x^{2}+y^{2}+z^{2}+w^{2})}$

$$
L_{x}=y+w-\lambda x,\,
 L_{y}=x+z-\lambda y,\, L_{z}=y+w -\lambda z, \,
L_{w}=x+z-\lambda w
$$
$$
x_{0}+z_{0}=\lambda y_{0}=\lambda w_{0},  \quad
y_{0}+w_{0}=\lambda x_{0}=\lambda z_{0}
$$

If $\lambda=0$, then
all  $(x_{0},y_{0},-x_{0},w_{0})$ with  $2x_{0}^{2}+y_{0}^{2}+w_{0}^{2}=1$
and all $(x_{0},y_{0},z_{0},-y_{0})$ with
$x_{0}^{2}+2y_{0}^{2}+z_{0}^{2}=1$ are constrained critical points, with
$f(x_{0},y_{0},-x_{0},w_{0})=0$ and $f(x_{0},y_{0},z_{0},-y_{0})$.

\medskip

If $\lambda  \ne0$, then $y_{0}=w_{0}$ and $x_{0}=z_{0}$, so
$$
2x_{0}=\lambda y_{0},\quad  2y_{0}=\lambda x_{0},\quad
2z_{0}=\lambda w_{0},\quad 2w_{0} =\lambda z_{0},
$$
and
$$
2x_{0}=\lambda y_{0} =\frac{\lambda}{2}(2y_{0})=\frac{\lambda^{2}}{2}x_{0}
\text{\; and \;\;}2z_{0}=\lambda w_{0}=\frac{\lambda}{2}(2w_{0})=
\frac{\lambda^{2}}{2}z_{0}.
$$
If $\lambda\ne2$, then
 $x_{0}=y_{0}=z_{0}=w_{0}$, which does not satisfy the constraint.
 If $\lambda=2$,  then
$$
x_{0}=y_{0}=z_{0}=w_{0}=\pm\frac{1}{2}\text{\; and\;\;}
(x_{0}+z_{0})(y_{0}+w_{0})=1.
$$
If $\lambda=-2$, then
$$
x_{0}=-y_{0}=z_{0}=-w_{0}=\pm\frac{1}{2}
\text{\; and\;\;}
(x_{0}+z_{0})(y_{0}+w_{0})=-1.
$$
Therefore, the constrained maximum  is $1$, attained at
$\pm\left(\frac{1}{2},\frac{1}{2},\frac{1}{2},\frac{1}{2}\right)$
 the constrained minimum  is $-1$, attained at
$\pm\left(\frac{1}{2},-\frac{1}{2},\frac{1}{2},-\frac{1}{2}\right)$.

\bigskip
{\bf \ref{exer:25}.}
\quad \quad \quad \quad
$L=(x+z)(y+w)-\dst{\frac{\lambda}{2}(x^{2}+y^{2})}
-\dst{\frac{\mu}{2}(z^{2}+w^{2})}$

$$
L_{x}=y+w-\lambda x,\,
 L_{y}=x+z-\lambda y,\, L_{z}=y+w -\mu z, \,
L_{w}=x+z-\mu w
$$
\begin{equation} \tag{A}
x_{0}+z_{0}=\lambda y_{0}=\mu w_{0}, \quad
y_{0}+w_{0}=\lambda x_{0}=\mu z_{0}
\end{equation}

\medskip

If $\lambda=\mu=0$, then $z_{0}=-x_{0}$ and $w_{0}=-y_{0}$,
$(x_{0},y_{0},-x_{0},-y_{0})$ satisfies the constraints and
$f(x_{0},y_{0},-x_{0},-y_{0})=0$  for all $(x_{0},y_{0})$
such that $x_{0}^{2}+y_{0}^{2}=1$.

\medskip
If $\lambda=0$
and $\mu\ne0$, then $z_{0}=w_{0}=0$, which does not satisfy the constraint
$z^{2}+y^{2}=1$. If $\mu=0$  and $\lambda\ne0$, then $x_{0}=y_{0}=0$,
which does not satisfy the constraint  $x^{2}+y^{2}=1$.

\medskip
Now assume that $\lambda$, $\mu\ne0$.  From (A),
 $\lambda(x_{0}^{2}+y_{0}^{2})=\mu^{2}(z_{0}^{2}+w_{0}^{2})$, so
$\lambda=\pm\mu$.    If $\lambda=-\mu$,   (A) implies  that
$y_{0}=-w_{0}$ and $x_{0}=-z_{0}$, so  again
$(x_{0},y_{0},-x_{0},-y_{0})$ satisfies the constraints and
$f(x_{0},y_{0},-x_{0},-y_{0})=0$  for all $(x_{0},y_{0})$
such that $x_{0}^{2}+y_{0}^{2}=1$.

\medskip

 If $\lambda=\mu$,  (A) becomes
$$
x_{0}+z_{0}=\lambda y_{0}= \lambda w_{0},\quad
y_{0}+w_{0}=\lambda x_{0}=\lambda z_{0},
$$
so
$y_{0}=w_{0}$, $x_{0}=z_{0}$,  $2x_{0}=\lambda y_{0}$, and $2y_{0}=\lambda
x_{0}$,  $4x_{0}=2\lambda y_{0}=\lambda^{2}x_{0}$, so $\lambda=\pm2$ .

\medskip
If
$\lambda=2$, $x_{0}=y_{0}=z_{0}=w_{0}$. To satisfy the constraints,

$$
(x_{0},y_{0},z_{0},w_{0})=\pm\left( \frac{1}{\sqrt2},
 \frac{1}{\sqrt2}, \frac{1}{\sqrt2},
\frac{1}{\sqrt2}
\right), \text{\; so\;\;}
$$
and the constrained maximum  is  $f(x_{0},y_{0},z_{0},w_{0})=2$.

\medskip

If
$\lambda=-2$, $x_{0}=-y_{0}=z_{0}=-w_{0}$. To satisfy the constraints,
$$
(x_{0},y_{0},z_{0},w_{0})=\pm\left( \frac{1}{\sqrt2},
- \frac{1}{\sqrt2}, \frac{1}{\sqrt2},
-\frac{1}{\sqrt2}
\right), \text{\; so\;\;}
$$
and the constrained minimum  is  $f(x_{0},y_{0},z_{0},w_{0})=-2$.

\bigskip
{\bf \ref{exer:26}.}
\centerline{$L=(x+z)(y+w)-\dst{\frac{\lambda}{2}}(x^{2}+z^{2})
-\dst{\frac{\mu}{2}}(y^{2}+w^{2})$}

$$
L_{x}=y+w-\lambda x,\;
L_{y}=x+z-\mu y,\;
L_{w}=x+z-\mu w,\;
L_{z}=y+w-\lambda z
$$

\centerline{$y_{0}+w_{0}=\lambda x_{0}$,\;
$x_{0}+z_{0}=\mu y_{0}$,\;
$x_{0}+z_{0}=\mu w_{0}$,\;
$y_{0}+w_{0}=\lambda z_{0}$}

If $\mu=0$, then $x_{0}=-z_{0}$, so the constrained critical points
are  $\pm\left(\frac{1}{\sqrt2},y_{0},-\frac{1}{\sqrt2},w_{0}\right)$
for all $(y_{0},w_{0})$ such that $y_{0}^{2}+w_{0}^{2}=1$; $f=0$
at all such points.

If $\lambda=0$, then $y_{0}=-w_{0}$, so the constrained critical points
are  $\pm\left(x_{0},\frac{1}{\sqrt2},z_{0},-\frac{1}{\sqrt2}\right)$
for all $(x_{0},z_{0})$ such that $x_{0}^{2}+z_{0}^{2}=1$; $f=0$
at all such points.

\medskip

Now suppose  that
$\lambda\mu\ne0$.   Since
$\lambda x_{0}=\lambda z_{0}$ and
$\mu y_{0}=\mu w_{0}$,
 $x_{0}=z_{0}$ and $y_{0}=w_{0}$. Therefore,
$(x_{0},z_{0})=\pm\left(\frac{1}{\sqrt2},\frac{1}{\sqrt2}\right)$
and
$(y_{0},w_{0})=\pm\left(\frac{1}{\sqrt2},\frac{1}{\sqrt2}\right)$, so
the constrained maximum is $2$, attained at
$\pm(\frac{1}{\sqrt2},\frac{1}{\sqrt2},\frac{1}{\sqrt2},\frac{1}{\sqrt2})$,
and constrained minimum is $-2$, attained
$\pm\left(\frac{1}{\sqrt2},-\frac{1}{\sqrt2},\frac{1}{\sqrt2},-\frac{1}{\sqrt2}\right)$,

\bigskip
{\bf \ref{exer:27}.}
\quad \quad \quad $L=\dst{\frac{(x_{1}-x_{2})^{2}+(y_{1}-y_{2})^{2}}{2}}-
\dst{\frac{\lambda}{2}}(x_{1}^{2}+y_{1}^{2})-\mu x_{2}y_{2}$

$$
L_{x_{1}}=x_{1}-x_{2}-\lambda x_{1},\; L_{x_{2}}=x_{2}-x_{1}-\mu
y_{2},\;  L_{y_{1}}=y_{1}-y_{2}-\lambda y_{1},\;
L_{y_{2}}=y_{2}-y_{1}-\mu x_{2}
$$


\centerline{(i) $x_{10}-x_{20}=\lambda  x_{10}$,\quad  (ii)
$y_{10}-y_{20}=\lambda y_{10}$}

\medskip

\centerline{(iii) $x_{20}-x_{10}=\mu y_{20}$,\quad
(iv) $y_{20}-y_{10}=\mu x_{20}$}
\medskip

Since $0<x_{10}<x_{20}$ and  $0<y_{10}<y_{20}$,  $\lambda<0$
and $\mu>0$
Since $x_{20}\ne0$, $\lambda \ne 1$, (i) and (ii) imply that
(v) $x_{10}y_{20}=y_{10}x_{20}$.  From (i) and (iii),
(vi) $\lambda x_{10}=-\mu y_{20}$; from (ii) and (iv), (vii) $\lambda
y_{10}=-\mu x_{20}$. Since $x_{20}y_{20}=1$, (vi) and (vii)  imply that
$x_{10}x_{20}=y_{10}y_{20}$. This and (v) imply that
$$
\frac{x_{10}}{y_{10}}=\frac{x_{20}}{y_{20}}=\frac{y_{20}}{x_{20}}.
$$
Therefore, $x_{20}=y_{20}=1$  and $x_{10}=y_{10}=\frac{1}{\sqrt2}$,
so
$$
(x_{10}-x_{20})^{2}+(y_{10}-y_{20})^{2}=
2\left(1-\frac{1}{\sqrt 2}\right)^{2}
$$
and the distance between the curves is $\sqrt2-1$.

\bigskip
{\bf \ref{exer:28}.}
\centerline{$L=\dst{\frac{1}{2}}\left(\dst{\frac{x^{2}}{\alpha^{2}}+\frac{y^{2}}{\beta^{2}}}
+\frac{z^{2}}{\gamma^{2}}\right)
-\lambda (ax+by+cz)$}
$$
L_{x}=\frac{x}{\alpha^{2}}-\lambda a,\quad
L_{y}=\frac{y}{\beta^{2}}-\lambda b,\quad
L_{z}=\frac{z}{\gamma^{2}}-\lambda c
$$
$$
x_{0}=\lambda a \alpha^{2},\quad
y_{0}=\lambda b \beta^{2},\quad
z_{0}=\lambda c \gamma^{2}
$$
$$
ax_{0}+by_{0}+cz_{0}=
\lambda[(a \alpha)^{2}+(b\beta)^{2}+(c\gamma^{2})]=d,
\quad
\lambda=\frac{d}{(a\alpha)^{2}+(b\beta^{2})+(c\gamma)^{2}}
$$
$$
\frac{x_{0}^{2}}{\alpha^{2}}+\frac{y_{0}^{2}}{\beta^{2}}
+\frac{z_{0}^{2}}{\gamma^{2}}=
\lambda^{2}[(a\alpha)^{2}+(b\beta)^{2}+(c\gamma)^{2}]
= \frac{d^{2}}{(a\alpha)^{2}+(b\beta^{2})+(c\gamma)^{2}}.
$$

\bigskip
{\bf \ref{exer:29}.}
\quad
$\dst{L(x_{1},x_{2},\dots,x_{n})=\frac{(x_{1}-c_{1})^{2}+(x_{2}-c_{2})^2+
\cdots+(x_{n}-c_{n})^2}{2}}$
$$
-\lambda(a_{1}x_{1}+a_{2}x_{2}+\cdots+a_{n}x_{n})
$$
$L_{x_{i}}=x_{i}-c_{i}-\lambda a_{i}$, $1\le i\le n$. We must choose
$\lambda$ so that if $x_{i0}=c_{i}+\lambda a_{i}$, $1\le i\le n$, then
\begin{eqnarray*}
a_{1}x_{10}+a_{2}x_{20}+\cdots+a_{n}x_{n0}&=&a_{1}c_{1}+a_{2}c_{2}+\dots +
a_{n}c_{n}\\ &+& \lambda (a_{1}^{2}+a_{2}^{2}+\cdots+ a_{n}^{2})=d,
\end{eqnarray*}
which holds if and only if
$$
\lambda=\frac{d-a_{1}c_{1}-a_{2}c_{2}-\cdots-a_{n}c_{n}}
{a_{1}^{2}+a_{2}^{2}+\cdots +a_{n}^{2}}.
$$
Therefore,
$$
x_{i0}=c_{i}+
\frac{(d-a_{1}c_{1}-a_{2}c_{2}-\cdots-a_{n}c_{n})a_{i}}
{a_{1}^{2}+a_{2}^{2}+\cdots a_{n}^{2}},\quad  1\le i\le n,
$$
and the distance from $(x_{10},x_{10},\dots,x_{n0})$  to
 $(c_{1},c_{2},\dots,c_{n})$  is
$$
\frac{|(d-a_{1}c_{1}-a_{2}c_{2}-\cdots-a_{n}c_{n})a_{i}|}
{\sqrt{a_{1}^{2}+a_{2}^{2}+\cdots a_{n}^{2}}}.
$$

\bigskip
{\bf \ref{exer:30}.}
\centerline{$L=\dst{\frac{1}{2}\sum_{i=1}^{n}a_{i}x_{i}^{2}-
\frac{\lambda}{4}\sum_{i=1}^{n}b_{i}x_{i}^{4}}$,\quad
$L_{x_{i}}=a_{i}x_{i}-\lambda b_{i}x_{i}^{3}$,\quad
$a_{i}x_{i0}^{2}=\lambda b_{i}x_{i0}^{4}$}

$\dst{\sum_{i=1}^{n}a_{i}x_{i0}^{2}=\lambda \sum_{i=1}^{n}
b_{i}x_{i0}^{4}=\lambda}$,\;
$x_{i0}^{2}=\dst{\frac{a_{i}}{\lambda b_{i}}}$,\quad
$\lambda=\dst{\sum_{i=1}^{n}a_{i}x_{i0}^{2}=
\frac{1}{\lambda}\sum_{i=1}^{n}\frac{a_{i}^{2}}{b_{i}}}$,\;
$\lambda=\dst{\left(\sum_{i=1}^{n}\frac{a_{i}^{2}}{b_{i}}\right)^{1/2}}$


{\bf \ref{exer:31}.}
\centerline{$L=\dst{\frac{1}{p}\sum_{i=1}^{n}a_{i}x_{i}^{p}-
\frac{\lambda}{q}\sum_{i=1}^{n}b_{i}x_{i}^{q}}$,\quad
$L_{x_{i}}=a_{i}x_{i}^{p}-\lambda b_{i}x_{i}^{q}$,\quad
$a_{i}x_{i0}^{p}=\lambda b_{i}x_{i0}^{q}$}

$\dst{\sum_{i=1}^{n}a_{i}x_{i0}^{p}=\lambda \sum_{i=1}^{n}
b_{i}x_{i0}^{q}=\lambda}$,\;
$x_{i0}^{q-p}=\dst{\frac{a_{i}}{\lambda b_{i}}}$,\quad
$x_{i0}=\dst{\left(\frac{a_{i}}{\lambda b_{i}}\right)^{1/(q-p)}}$,
$x_{i0}^{p}=\dst{\left(\frac{a_{i}}{\lambda b_{i}}\right)^{p/(q-p)}}$

$$
\lambda=\sum_{i=1}^{n}a_{i}x_{i0}^{p}=\lambda^{p/(p-q)}
 \sum_{i=1}^{n}a_{i}^{q/(q-p)} b_{i}^{p/(p-q)},\quad
$$
$$
\lambda^{q/(q-p)}=
 \sum_{i=1}^{n}a_{i}^{q/(q-p)} b_{i}^{p/(p-q)},\quad
\lambda=
\left(\sum_{i=1}^{n}a_{i}^{q/(q-p)} b_{i}^{p/(p-q)}\right)^{1-p/q}=
\sum_{i=1}^{n}a_{i}x_{i0}^{p}
$$
$\lambda$  is the constrained maximum if $p<q$, the constrained minimum if
$p>q$, undefined if $p=q$.

{\bf \ref{exer:32}.}
$L=\dst{\frac{x^{2}+2y^{2}+z^{2}+w^{2}}{2}}-\lambda(x+y+z+3w)-\mu(x+y+2z+w)$
$$
L_{x}=x-\lambda-\mu, \quad L_{y}=2y-\lambda-\mu, \quad
L_{z}=z-\lambda-2\mu, \quad L_{w}=w-3\lambda-\mu
$$
$$
x_{0}=\lambda+\mu, \quad y_{0}=\frac{\lambda+\mu}{2}, \quad
z_{0}=\lambda+2\mu,
\quad w_{0}=3\lambda+\mu
$$
$$
x_{0}+y_{0}+z_{0}+3w_{0}=\frac{23}{2}\lambda+\frac{13}{2}\mu=1, \quad
x_{0}+y_{0}+2z_{0}+w_{0}=\frac{13}{2}\lambda+\frac{13}{2}\mu=2
$$
$$
\lambda=-\frac{1}{5},\, \mu=\frac{33}{65},\, x_{0}=\frac{4}{13},\,
y_{0}=\frac{2}{13},\,
z_{0}=\frac{53}{65},\,
w_{0}=-\frac{6}{65},\,\
\min=\frac{689}{845}
$$

\bigskip
{\bf \ref{exer:33}.}
\centerline{$\dst{L=\frac{1}{2}\left(\frac{x^{2}}{a^{2}}+\frac{y^{2}}{b^{2}}+\frac{z^{2}}{c^{2}}\right)
-\lambda(p_{1}x+p_{2}y+p_{3}z)}$}

\medskip

\centerline{$L_{x}=\dst{\frac{x}{a^{2}}}-\lambda p_{1}$,\quad
$L_{y}=\dst{\frac{y}{b^{2}}}-\lambda p_{2}$,\quad
$L_{z}=\dst{\frac{z}{c^{2}}}-\lambda p_{3}$}

\medskip

\centerline{$x_{0}=\lambda p_{1}a^{2}$,\quad
$y_{0}=\lambda p_{2}b^{2}$,\quad
$z_{0}=\lambda p_{3}c^{2}$}

\medskip

\centerline{$p_{1}x_{0}+p_{2}y_{0}+p_{3}z_{0}=
\lambda(p_{1}^{2}a^{2}+p_{2}^{2}b^{2}+p_{3}^{2}c^{2})=d$}

\medskip

\centerline{$\lambda=\dst{\frac{d}{p_{1}^{2}a^{2}+p_{2}^{2}b^{2}+p_{3}^{2}c^{2}}}$,\quad
$\dst{\frac{x_{0}}{a}=\lambda p_{1}a}$,\quad
$\dst{\frac{y_{0}}{b}=\lambda p_{2}b}$, \quad
$\dst{\frac{z_{0}}{b}=\lambda p_{3}c}$}

\medskip

$$
\frac{x_{0}^{2}}{a^{2}}+\frac{y_{0}^{2}}{b^{2}}+\frac{z_{0}^{2}}{c^{2}}
=\lambda^{2}(p_{1}^{2}a^{2}+p_{2}^{2}b^{2}+p_{3}^{2}c^{2})
=\frac{d^{2}}{p_{1}^{2}a^{2}+p_{2}^{2}b^{2}+p_{3}^{2}c^{2}}
$$

\bigskip
{\bf \ref{exer:34}.}
\centerline{$L=p_{1}x+p_{2}y+p_{3}z-
\dst{\frac{\lambda}{2}\left(\frac{x^{2}}{a^{2}}+\frac{y^{2}}{b^{2}}+
\frac{z^{2}}{c^{2}}\right)}$}

\medskip

\centerline{$L_{x}=p_{1}-\lambda\dst{\frac{x}{a^{2}}}$,\quad
$L_{y}=p_{2}-\lambda\dst{\frac{y}{b^{2}}}$,\quad
$L_{z}=p_{3}-\lambda\dst{\frac{z}{c^{2}}}$}

\medskip

\centerline{$x_{0}=\dst{\frac{p_{1}a^{2}}{\lambda}}$,\quad
$y_{0}=\dst{\frac{p_{2}b^{2}}{\lambda}}$,\quad
$z_{0}=\dst{\frac{p_{3}c^{2}}{\lambda}}$}

$$
\frac{x_{0}^{2}}{a^{2}}+\frac{y_{0}^{2}}{b^{2}}+\frac{z_{0}^{2}}{c^{2}}
=\frac{p_{1}^{2}a^{2}+p_{2}^{2}b^{2}+p_{3}^{2}c^{2}}{\lambda^{2}}=1
$$
 $$
\lambda=\pm(p_{1}^{2}a^{2}+p_{2}^{2}b^{2}+p_{3}^{2}c^{2})^{1/2}
$$
$$
p_{1}x_{0}+p_{2}y_{0}+p_{3}z_{0}=
\frac{p_{1}^{2}a^{2}+p_{2}^{2}b^{2}+p_{3}^{2}c^{2}}{\lambda}
=\pm (p_{1}^{2}a^{2}+p_{2}^{2}b^{2}+p_{3}^{2}c^{2})^{1/2}
$$

\medskip
{\bf \ref{exer:35}.}
$L=\dst{\frac{(x+1)^{2}+(y-2)^{2}+(z-3)^{2}}{2}}-\lambda(x+2y-3z)-\mu(2x-y+2z)$

$$
L_{x}=x+1-\lambda-2\mu,\quad  L_{y}=y-2-2\lambda +\mu,\quad
L_{z}=z-3+3\lambda-2\mu
$$
$$
x_{0}=-1+\lambda+2\mu,\quad  y_{0}=2+2\lambda-\mu,  \quad
z_{0}=3-3\lambda+2\mu
$$
$$
x_{0}+2y_{0}-3z_{0}-4=-10+14\lambda-6\mu,\quad
2x_{0}-y_{0}+2z_{0}-5=-3-6\lambda+9\mu
$$
$$
7\lambda-3\mu=5, \quad -2\lambda+3\mu=1,\quad  \lambda=\frac{18}{15},
\quad \mu=\frac{17}{15}
$$
$$
(x_{0},y_{0},z_{0})=\left(\frac{37}{15},
\frac{49}{15},\frac{25}{15}\right),\quad
$$
$$
\sqrt{(x_{0}+1)^{2}+(y_{0}-2)^{2}+(z_{0}-3)^{2}}
=\left[\left(\frac{52}{15}\right)+\left(\frac{19}{15}\right)^{2}+
\left(\frac{20}{15}\right)^{2}\right]^{1/2}=\sqrt{\frac{693}{45}}
$$


\bigskip
{\bf \ref{exer:36}.}
\centerline{$L=2x+y+2z-\dst{\frac{\lambda}{2}}(x^{2}+y^{2})-\mu(x+z)$}
$$
L_{x}=2-\lambda x-\mu,\quad  L_{y}=1-\lambda y,\quad  L_{z}=2-\mu
$$
 $\mu=2$, so $\lambda x_{0}=0$. Since $\lambda y_{0}=1$, $\lambda\ne0$;
hence,
 $x_{0}=0$. Since $x_{0}^{2}+y_{0}^{2}=4$,
$y_{0}=\pm2$.
Therefore,
$(0,2,2)$ and $(0,-2,2)$, are constrained extreme points, and
the constrained extreme values  are
$f(0,2,2)=6$ and $f(0,-2,2)=2$.


\bigskip
{\bf \ref{exer:37}.}
Let $(x_{1},y_{1})$  be on the parabola,  $(x_{2},y_{2})$ on the line.
$$
L=\frac{(x_{1}-x_{2})^{2}+(y_{1}-y_{2})^{2}}{2}
-\lambda(y_{1}-x_{1}^{2})-\mu(x_{2}+y_{2}).
$$
$$
L_{x_{1}}=x_{1}-x_{2}+2\lambda x_{1},\,
L_{x_{2}}=x_{2}-x_{1}-\mu,\,
$$
$$
L_{y_{1}}=y_{1}-y_{2}-\lambda,\,
L_{y_{2}}=y_{2}-y_{1}-\mu
$$
\begin{eqnarray*}
x_{10}-x_{20}&=&-2\lambda x_{10}\\
x_{20}-x_{10}&=&\mu\\
y_{10}-y_{20}&=&\lambda\text{\quad (i)}\\
y_{20}-y_{10}&=&\mu\text{\quad (ii)}
\end{eqnarray*}
From (i) and (ii), $\lambda=-\mu$, so
\begin{eqnarray*}
x_{10}-x_{20}&=& 2\mu x_{10}\text{\quad (i)}\\
x_{20}-x_{10}&=&\mu\text{\quad\quad \;\,  (ii)}\\
y_{20}-y_{10}&=&\mu\text{\quad\quad \;\,  (iii)}
\end{eqnarray*}
From (i) and (ii), $x_{10}=-1/2$, so $y_{10}=1+x_{10}^{2}=5/4$ and
$$
2\mu=x_{20}+y_{20}-x_{10}-y_{10}=-1+\frac{1}{2}-\frac{5}{4}=-\frac{7}{4},
$$
since $x_{20}+y_{20}=-1$ (constraint). Therefore, $\mu=-7/8$ so (ii)
and (iii) imply that
$$
x_{20}=x_{10}=\mu=-\frac{1}{2}-\frac{7}{8}=-\frac{11}{8}
\text{\; and\;\;}
y_{20}=y_{10}-\frac{7}{8}=\frac{5}{4}-\frac{7}{8}=\frac{3}{8}.
$$
 The distance between the line and the
parabola is
$$
\sqrt{(x_{10}-x_{20})^{2}+(y_{10}-y_{20})^{2}}=\frac{7}{4\sqrt{2}}.
$$

\bigskip
{\bf \ref{exer:38}.}
Let $(x_{1},y_{1},z_{1})$ be on the ellipsoid and
$(x_{2},y_{2},z_{2})$ be on the plane.
$$
L=
\frac{(x_{1}-x_{2})^{2}+(y_{1}-y_{2})^{2}+(z_{1}-z_{2})^{2}}{2}
-\frac{\lambda}{2}(3x_{1}^{2}+9y_{1}^{2}+6z_{1}^{2})
-\mu(x_{2}+y_{2}+2z_{2}).
$$
$$
 L_{x_{1}}=x_{1}-x_{2}-3\lambda x_{1}=0,\quad
L_{y_{1}}=y_{1}-y_{2}-9\lambda y_{1}=0,\quad
L_{z_{1}}=z_{1}-z_{2}-6\lambda z_{1}
$$
$$
L_{x_{2}}=x_{2}-x_{1}-\mu,\quad Ly_{2}=y_{2}-y_{1}-\mu,\quad
L_{z_{2}}=z_{2}-z_{1}-2\mu
$$
\begin{eqnarray*}
x_{10}-x_{20}&=& 3\lambda x_{10}\\
y_{10}-y_{20}&=&    9\lambda y_{10}\\
z_{10}-z_{20}&=&  6\lambda z_{10}\\
x_{20}-x_{10}&=&  \mu\\
y_{20}-y_{10}&=&  \mu \\
z_{20}-z_{10}&=&  2\mu
\end{eqnarray*}
Therefore, $3\lambda x_{10}=-\mu$, $9\lambda y_{10}=-\mu$, and
 $3\lambda z_{10}=-\mu$, so
 $y_{10}=x_{1}/3$ and
$z_{10}=x_{10}$. Since $(x_{10},x_{10}/3,x_{10})$ is on the ellipsoid if
and only if $x_{10}=\pm1$, either
$$
\text{\; (a)\;\;}(x_{10},y_{10},z_{10})=\left(1,\frac{1}{3},1\right)
\text{\; or \quad (b)\;\;}
 (x_{10},y_{10},z_{10})=\left(-1,-\frac{1}{3},-1\right).
$$
Since
\begin{equation} \tag{A}
x_{2}=x_{1}+\mu,\quad  y_{2}=y_{1}+\mu,\quad  z_{2}=z_{1}+2\mu,
\end{equation}
\begin{equation} \tag{B}
(x_{10}-x_{20})^{2}+(y_{10}-y_{20})^{2}+(z_{10}-z_{20})=6\mu^{2}, \text{\;
so\;\;} d=\mu.\sqrt{6}.
\end{equation}
Since
$3x_{20}+3y_{20}+6z_{20}=70$, (A) implies that
$$
\mu=\frac{70-3x_{10}-3y_{10}-6z_{10}}{18},
$$

In Case (a) $\mu=\frac{10}{3}$ so (A) implies that  $d=\frac{10\sqrt{6}}{3}$
In case (b)
 $\mu=\frac{40}{9}>\frac{10}{3}$, so
the distance between the plane and the ellipsoid is
$\frac{10\sqrt{6}}{3}$.

\bigskip
{\bf \ref{exer:39}.}
\quad \quad \quad \quad $L=xy+yz+zx-\dst{\frac{\lambda}{2}
\left(\frac{x^{2}}{a^{2}}+\frac{y^{2}}{b^{2}}+\frac{z^{2}}{c^{2}}\right)}$
$$
L_{x}=y+z-\lambda\frac{x}{a^{2}},\quad
L_{y}=z+x-\lambda\frac{y}{b^{2}},\quad
L_{z}=x+y-\lambda\frac{z}{c^{2}}
$$
$$
y_{0}+z_{0}=\lambda\frac{x_{0}}{a^{2}},\quad
z_{0}+x_{0}=\lambda\frac{y_{0}}{b^{2}},\quad
x_{0}+y_{0}-\lambda\frac{z_{0}}{c^{2}}
$$
$$
\left[\begin{array}{ccccccc}
0&a^{2}&a^{2}\\ b^{2}&0&b^{2}\\c^{2}&c^{2}&0
\end{array}\right]
\left[\begin{array}{ccccccc}
x_{0}\\y_{0}\\z_{0}
\end{array}\right]=\lambda
\left[\begin{array}{ccccccc}
x_{0}\\y_{0}\\z_{0}
\end{array}\right]
$$
$$
x_{0}(y_{0}+z_{0})=\lambda\frac{x_{0}^{2}}{a^{2}},\quad
y_{0}(z_{0}+x_{0})=\lambda\frac{y_{0}^{2}}{b^{2}},\quad
z_{0}(x_{0}+y_{0})=\lambda\frac{z_{0}^{2}}{c^{2}},
$$

$$
x_{0}(y_{0}+z_{0})+
y_{0}(z_{0}+x_{0})+
z_{0}(x_{0}+y_{0})=
\lambda\left(\frac{x_{0}^{2}}{a^{2}}+\frac{y_{0}^{2}}{b^{2}}+\frac{z_{0}^{2}}{c^{2}}\right)
=\lambda
$$

\bigskip
{\bf \ref{exer:40}.}
\quad \quad \quad $L=xy+2yz+2zx-\lambda
\dst{\left(\frac{x^{2}}{a^{2}}+\frac{y^{2}}{b^{2}}+\frac{z^{2}}{c^{2}}\right)}$
$$
L_{x}=y+2z-2\lambda\frac{x}{a^{2}},\quad
L_{y}=x+2z-2\lambda\frac{y}{b^{2}},\quad
L_{z}=2x+2y-2\lambda\frac{z}{c^{2}}
$$
$$
y_{0}+2z_{0}=2\lambda\frac{x_{0}}{a^{2}},\quad
x_{0}+2z_{0}=2\lambda\frac{y_{0}}{b^{2}},\quad
2x_{0}+2y_{0}-2\lambda\frac{z_{0}}{c^{2}}
$$

$$
\left[\begin{array}{ccccccc}
0&a^{2}/2&a^{2}\\ b^{2}/2&0&b^{2}\\c^{2}&c^{2}&0
\end{array}\right]
\left[\begin{array}{ccccccc}
x_{0}\\y_{0}\\z_{0}
\end{array}\right]=\lambda
\left[\begin{array}{ccccccc}
x_{0}\\y_{0}\\z_{0}
\end{array}\right].
$$

$$
x_{0}(y_{0}+2z_{0})=2\lambda\frac{x_{0}^{2}}{a^{2}},\quad
y_{0}(x_{0}+2z_{0})=2\lambda\frac{y_{0}^{2}}{b^{2}};\quad
z_{0}(2x_{0}+2y_{0})=2\lambda\frac{z_{0}^{2}}{c^{2}},
$$
$$
\frac{x_{0}(y_{0}+2z_{0})+
y_{0}(x_{0}+2z_{0})+
z_{0}(2x_{0}+2y_{0})}{2}=
\lambda\left(\frac{x_{0}^{2}}{a^{2}}+\frac{y_{0}^{2}}{b^{2}}+\frac{z_{0}^{2}}{c^{2}}\right)
=\lambda,
$$

\bigskip
{\bf \ref{exer:41}.}
\quad \quad \quad $L=xz+yz-\dst{\frac{\lambda}{2}
\left(\frac{x^{2}}{a^{2}}+\frac{y^{2}}{b^{2}}+\frac{z^{2}}{c^{2}}\right)}$
$$
L_{x}=z-\lambda\frac{x}{a^{2}},\quad
L_{y}=z-\lambda\frac{y}{b^{2}},\quad
L_{z}=x+y-\lambda\frac{z}{c^{2}}
$$
$$
z_{0}=\lambda\frac{x_{0}}{a^{2}},\quad
z_{0}=\lambda\frac{y_{0}}{b^{2}},\quad
x_{0}+y_{0}=\lambda\frac{z_{0}}{c^{2}},\text{\; so\;\;}
\frac{a^{2}}{\lambda}+\frac{b^{2}}{\lambda}=\frac{\lambda}{c^{2}}.
$$
Therefore, $\lambda=\pm |c|\sqrt{a^{2}+b^{2}}$. To determine $z_{0}$,
 note  that $x_{0}=\dst{\frac{a^{2}z_{0}}{\lambda}}$ and
$y_{0}=\dst{\frac{b^{2}z_{0}}{\lambda}}$. Therefore,
$$
1=\frac{x_{0}^{2}}{a^{2}}
+\frac{y_{0}^{2}}{b^{2}}+\frac{z_{0}^{2}}{c^{2}} =
\left(\frac{a^{2}+b^{2}}{\lambda^{2}}+\frac{1}{c^{2}}\right)z_{0}^{2}=
\frac{2z_{0}^{2}}{c^{2}},
$$
so
$$
z_{0}=\pm\frac{|c|}{\sqrt2}\text{\; and\;\;}
(x_{0},y_{0},z_{0})=\pm
\left(\frac{a^{2}}{\sqrt{2(a^{2}+b^{2})}},
\frac{b^{2}}{\sqrt{2(a^{2}+b^{2})}},
\dst{\frac{|c|}{\sqrt{2}}}
\right)
$$
$$
(x_{0}+y_{0})z_{0}=\dst{\frac{\lambda
z_{0}^{2}}{c^{2}}}=\pm\frac{\lambda}{2}=\pm
\frac{|c|\sqrt{a^{2}+b^{2}}}{2}.
$$

\bigskip
{\bf \ref{exer:42}.}
\centerline{$L=x^{\alpha}y^{\beta}z^{\gamma}-\lambda(ax^{p}+by^{q}+cz^{r})$}
$$
L_{x}=\alpha x^{\alpha-1}y^{\beta}z^{\gamma}-\lambda pax^{p-1},\quad
L_{y}=\beta x^{\alpha}y^{\beta-1}z^{\gamma}-\lambda qby^{q-1}
$$
$$
L_{z}=\gamma x^{\alpha}y^{\beta}z^{\gamma-1}-\lambda rcz^{r-1}
$$
$$
\dst{\frac{p}{\alpha}ax_{0}^{p}=\frac{q}{\beta}by_{0}^{q}=\frac{r}{\gamma}cz_{0}^{q}=C}
$$
where $C$  is to be determined as follows:
$$
\dst{ax_{0}^{p}=\frac{C\alpha}{p},\quad by_{0}^{q}=\frac{C\beta}{q},\quad
cz_{0}^{q}=\frac{C\gamma}{r}}
$$
From the constraint,
$$
ax_{0}^p+by_{0}^{p}+cz_{0}^{r}=1,
$$
so
$$
C=\dst{\left(\frac{\alpha}{p}+\frac{\beta}{q}+\frac{\gamma}{r}\right)^{-1}}
\text{\; and\;\;}
\dst{x_{0}^{p}y_{0}^{q}z_{0}^{r}=\frac{\alpha\beta\gamma}{pqr}
\left(\frac{\alpha}{p}+\frac{\beta}{q}+\frac{\gamma}{r}\right)^{-3}}.
$$

\bigskip
{\bf \ref{exer:43}.}
\quad \quad \quad \quad $L=xw-yz-\dst{\frac{\lambda
(x^{2}+2y^{2})}{2}-\frac{\mu(2z^{2}+w^{2})}{2}}$

$$
 L_{x}=w-\lambda x,\quad
 L_{y}=-z-2\lambda y,\quad
 L_{z}=-y-2\mu z,\quad
 L_{w}=x-\mu w
$$
$$
 w_{0}=\lambda x_{0},\quad
 z_{0}=-2\lambda y_{0},\quad
 y_{0}=-2\mu z_{0},\quad
 x_{0}=\mu w_{0}
$$
The first and last equality imply that $w_{0}=\lambda\mu w_{0}$
and $z_{0}=4\lambda\mu z_{0}$.
Since\\ $2z_{0}^{2}+w_{0}^{2}=9$, $w_{0}$ and $z_{0}$ cannot both be zero,
so either
$\lambda\mu=1$ or $4\lambda\mu=1$.

\bigskip

 If $\lambda\mu=1$,
$z_{0}=y_{0}=0$,
$x_{0}^{2}=4$, and
$w_{0}^{2}=9$, so the constrained critical values    are
$$
f(2,0,0,3)=f(-2,0,0,-3)=6 \text{\; and\;\;}
f(-2,0,0,3)=f(2,0,0,-3)=-6.
$$

\bigskip
If $4\lambda\mu=1$,  then $x_{0}=w_{0}=0$, $y_{0}^{2}=2$ and
$z_{0}^{2}=9/2$, so the constrained critical values are
$$
f\left(0,\sqrt{2},\frac{3}{\sqrt{2}},0\right)=
f\left(0,-\sqrt{2},-\frac{3}{\sqrt{2}},0\right)=3
$$
and
$$
f\left(0,\sqrt{2},-\frac{3}{\sqrt{2}},0\right)=
f\left(0,-\sqrt{2},\frac{3}{\sqrt{2}},0\right)= -3.
$$
Hence the constrained maximum and minimum values are $3$  and
$-3$.


\bigskip
{\bf \ref{exer:44}.}
\centerline{$L=xw-yz-\dst{
\frac{\lambda}{2}(ax^{2}+by^{2})-\frac{\mu}{2}(cz^{2}+dw^{2})}$}
$$
 L_{x}=w-a\lambda x,\quad
 L_{y}=-z-b\lambda y,
$$
$$
 L_{z}=-y-c\mu z=0,\quad
 L_{w}=x-d\mu w=0
$$
$$
x_{0}=\mu dw_{0},\quad y_{0}=-c\mu z_{0},\quad z_{0}=-b\lambda y_{0},
\text{\; and\;\;}  w_{0}=\lambda a x_{0}.
$$
This implies that
$$
x_{0}w_{0}-y_{0}z_{0}=\lambda (ax_{0}^{2}+by_{0}^{2})=\lambda
\text{\; and\;\;}
x_{0}w_{0}-y_{0}z_{0}=\mu(cz_{0}^{2}+dw_{0}^{2}) =\mu,
$$
so $\lambda =\mu$. Therefore,
$$
x_{0}=\lambda dw_{0},\quad y_{0}=-c\lambda z_{0},\quad z_{0}=-b\lambda
y_{0},
\text{\; and\;\;}  w_{0}=\lambda a x_{0},
$$
so $z_{0}=bc\lambda^{2} z_{0}$ and $w_{0}=ad\lambda^{2}w_{0}$.
 Since $cz_{0}^{2}+dw_{0}^{2}=1$, $w_{0}$ and $z_{0}$ cannot
both be zero; hence, either $ad\lambda^{2}=1$ or $bc\lambda^{2}=1$.

\bigskip
{\bf (a)}
Suppose that $ad\ne bc$. If $\lambda^{2} ad=1$, then
$\lambda^{2} bc\ne1$, so
$z_{0}=y_{0}=0$, and the constraints imply that
$x_{0}^{2}=1/a$, and
$w_{0}^{2}=1/d$.
Therefore,
the constrained  maximum is
$$
\dst{\frac{1}{\sqrt{ad}}},\text{\; attained at\;\;}
\pm \dst{\left(\frac{1}{\sqrt{a}},0,0,\frac{1}{\sqrt{d}}\right)}
$$
and
the constrained  minimum is
$$
-\dst{\frac{1}{\sqrt{ad}}},\text{\; attained at\;\;}
\pm \dst{\left(-\frac{1}{\sqrt{a}},0,0,\frac{1}{\sqrt{d}}\right)}.
$$
If $\lambda^{2} bc=1$,  then  $\lambda^{2} ad\ne1$, so $x_{0}=w_{0}=0$
and the constraints imply that $y_{0}^{2}=1/b$ and
$z_{0}^{2}=1/c$.
Therefore, the constrained  maximum is
$$
\dst{\frac{1}{\sqrt{bc}}}, \text{\; attained at\;\;}
\pm \dst{\left(0,\frac{1}{\sqrt{b}},-\frac{1}{\sqrt{c}},0\right)},
$$
and the constrained  minimum is
$$
-\dst{\frac{1}{\sqrt{bc}}}, \text{\; attained at\;\;}
\pm \dst{\left(0,\frac{1}{\sqrt{b}},\frac{1}{\sqrt{c}},0\right)}.
$$

\medskip
{\rm (b)}
Suppose that $ad=bc$.
Since $x_{0}=\lambda dw_{0}$ and $y_{0}=-c\lambda z_{0}$,
$$
1=ax_{0}^{2}+by_{0}^{2}=\lambda^{2}[(ad)dw_{0}^2+(bc)cz_{0}^{2}]=
\lambda^{2}ad(cz_{0}^{2}+d(w_{0})^{2}=\lambda^{2}ad,
$$
so $\lambda=\pm \dst{\frac{1}{\sqrt {ad}}}=\pm\frac{1}{\sqrt{bc}}$.
Therefore, the constrained maximum value of $f$ is
$\dst{\frac{1}{\sqrt {ad}}}=\frac{1}{\sqrt{bc}}$,  is attained
at all points of the form
$\dst{\left(w_{0}\sqrt{\frac{d}{a}},-z_{0}\sqrt{\frac{c}{b}},z_{0},w_{0}\right)}$
and the constrained minimum value of $f$ is
$-\dst{\frac{1}{\sqrt {ad}}}=-\frac{1}{\sqrt{bc}}$,  attained
at all points of the form
$\dst{\left(-w_{0}\sqrt{\frac{d}{a}},z_{0}\sqrt{\frac{c}{b}},z_{0},w_{0}\right)}$
where, in both cases, $cz_{0}^2+dw_{0}^{2}=1$. Alternatively, all the
constrained maximum  points are of the  form
$\dst{\left(x_{0},y_{0},-y_{0}\sqrt{\frac{b}{c}},x_{0}\sqrt{\frac{a}{d}}\right)}$
and
all the
constrained minimum  points  are of the  form
$\dst{\left(x_{0},y_{0},y_{0}\sqrt{\frac{b}{c}},-x_{0}\sqrt{\frac{a}{d}}\right)}$
where, in both cases, $ax_{0}^{2}+by_{0}^{2}=1$.

\bigskip
{\bf \ref{exer:45}.}
\centerline{$L=\dst{\frac{\alpha x^{2}+\beta y^{2}+\gamma z^{2}}{2}}
-\lambda(a_{1}x+a_{2}y+a_{3}z)-\mu(b_{1}x+b_{2}y+b_{3}z)$}
$$
L_{x}=\alpha x-\lambda a_{1}-\mu b_{1},\quad
L_{y}=\beta y-\lambda a_{2}-\mu b_{2},    \quad
L_{z}=\gamma y-\lambda a_{3}-\mu b_{3}
$$

\begin{equation}\tag{A}
x_{0}=\frac{\lambda a_{1}+\mu b_{1}}{\alpha},\quad
y_{0}=\frac{\lambda a_{2}+\mu b_{2}}{\beta},\quad
z_{0}=\frac{\lambda a_{3}+\mu b_{3}}{\gamma}.
\end{equation}

\begin{equation}\tag{B}
\frac{a_{1}(\lambda a_{1}+\mu b_{1})}{\alpha}+
\frac{a_{2}(\lambda a_{2}+\mu b_{2})}{\beta}+
\frac{a_{3}(\lambda a_{3}+\mu b_{3})}{\gamma}=c.
\end{equation}

\begin{equation}\tag{C}
\frac{b_{1}(\lambda a_{1}+\mu b_{1})}{\alpha}+
\frac{b_{2}(\lambda a_{2}+\mu b_{2})}{\beta}+
\frac{b_{3}(\lambda a_{3}+\mu b_{3})}{\gamma}=d.
\end{equation}

Assume that
$$
{\bf u}=\frac{a_{1}}{\sqrt{\alpha}}{\bf i}+
\frac{a_{2}}{\sqrt{\beta}}{\bf j}+
\frac{a_{3}}{\sqrt{\gamma}}{\bf k}
\text{\; and\;\;}
{\bf v}=\frac{b_{1}}{\sqrt{\alpha}}{\bf i}+
\frac{b_{2}}{\sqrt{\beta}}{\bf j}+
\frac{b_{3}}{\sqrt{\gamma}}{\bf k}
$$
are linearly independent. Then (B) and (C) can be written as
\begin{equation}\tag{D}
|{\bf u}|^{2}\lambda+({\bf u}\cdot{\bf v})\mu=c,\quad
({\bf u}\cdot{\bf v})\lambda+|{\bf v}|^2\mu=d.
\end{equation}
Since ${\bf u}$ and ${\bf v}$ are linearly independent,
$\Delta=_\text{def}|{\bf u}|^{2}|{\bf v}|^{2}-({\bf u}\cdot{\bf
v})^{2}\ne0$. Therfore  the solution of (D) is
$$
\lambda=\frac{c|{\bf v}|^{2}-d({\bf u}\cdot{\bf v})}{\Delta},\quad
\mu=\frac{d|{\bf u}|^{2}-c({\bf u}\cdot{\bf v})}{\Delta}.
$$
From (A),
\begin{eqnarray*}
\alpha x_{0}^2+\beta y_{0}^{2}+\gamma z_{0}^{2} &=&
(\lambda a_{1}+\mu b_{1})^{2}+
(\lambda a_{2}+\mu b_{2})^{2}+
(\lambda a_{3}+\mu b_{3})^{2}\\
&=&
\lambda^{2} (a_{1}^{2}+a_{2}^{2}+a_{3}^{2})+
\mu^{2} (b_{1}^{2}+b_{2}^{2}+b_{3}^{2})\\
&&+ 2\lambda\mu(a_{1}b_{1}+a_{2}b_{2}+a_{3}b_{3}).
\end{eqnarray*}

\bigskip
{\bf \ref{exer:46}.}
\centerline{$L=\dst{\frac{1}{2}\sum_{i=1}^{n}(x_{i}-\alpha_{i})^{2}-
\lambda\sum_{i=1}^{n}a_{i}x_{i}-\mu\sum_{i=1}^{n}b_{i}x_{i}}$}

$$
L_{x_{i}}=x_{i}-\alpha_{i}-\lambda a_{i}-\mu_{i} b_{i},\quad
x_{i0}=\alpha_{i}+\lambda a_{i}+\mu b_{i}
$$
$$
c=\sum_{i=1}^{n}a_{i}x_{i0}=\sum_{i=1}^{n}a_{i}\alpha_{i}
+\lambda \sum_{i=1}^{n}a_{i}^{2}+\mu\sum_{i=1}^{n}a_{i}b_{i}
= \sum_{i=1}^{n}a_{i}\alpha_{i} +\lambda
$$

$$
d=\sum_{i=1}^{n}b_{i}x_{i0}=\sum_{i=1}^{n}b_{i}\alpha_{i}
+\lambda \sum_{i=1}^{n}a_{i}b_{i}+\mu\sum_{i=1}^{n}b_{i}^{2}
= \sum_{i=1}^{n}b_{i}\alpha_{i} +\mu
$$

$$
\lambda=c-\sum_{i=1}^{n}a_{i}\alpha_{i},\quad
\mu=d-\sum_{i=1}^{n}b_{i}\alpha_{i}
$$
\begin{eqnarray*}
\sum_{i=1}^{n}(x_{i0}-\alpha_{i})^{2}&=&
\sum_{i=1}^{n}(\lambda a_{i}+\mu b_{i})^{2}
=\lambda^{2}\sum_{i=1}^{n}a_{i}^{2}\\&+&2\lambda
\mu\sum_{i=1}^{n}a_{i}b_{i}
+\mu^{2}\sum_{i=1}^{n}b_{i}^{2}=\lambda^{2}+\mu^{2}\\
&=&\left(c-\sum_{i=1}^{n}a_{i}\alpha_{i}\right)^{2}
+\left(d-\sum_{i=1}^{n}b_{i}\alpha_{i}\right)^{2}
\end{eqnarray*}

\bigskip
{\bf \ref{exer:47}.}
$L=\dst{\frac{1}{2}}\sum_{i=1}^{n}x_{i}^{2}-
\lambda\sum_{i=1}^{n}x_{i}-\mu\sum_{i=1}^{n}jx_{i}$;
$L_{x_{i}}=x_{i}-\lambda-\mu i$, so $x_{i0}=\lambda+\mu i$.
To satisfy the constraints,
\begin{equation}
\dst{\sum_{i=1}^{n}(\lambda+ \mu i)=1}
\text{\; and\;\;}
\dst{\sum_{i=1}^{n}i(\lambda+ \mu i)=0}.  \tag{A}
\end{equation}
Let
$$
s_{0}=n, \quad  s_{1}=\sum_{j=1}^{n}i=\frac{n(n+1)}{2},\text{\; and\;\;}
s_{2}=\sum_{i=1}^{n}i^{2}=\frac{n(n+1)(2n+1)}{6}.
$$
Then (A) is equvalent to,
$$
\left[\begin{array}{ccccccc}
s_{0}&s_{1}\\ s_{1}&s_{2}
\end{array}\right]
\left[\begin{array}{ccccccc}
\lambda\\\mu
\end{array}\right] =
\left[\begin{array}{ccccccc}
1\\0
\end{array}\right].
$$

 By Cramer's rule,
$$
\lambda=\frac{s_{2}}{s_{0}s_{2}-s_{1}^{2}}=\frac{2(2n+1)}{n(n-1)}
\text{\; and\;\;}
\mu=-\frac{s_{1}}{s_{0}s_{2}-s_{1}^{2}}=-\frac{6}{n(n-1)}.
$$
Therefore,
$$
x_{i0}=\dst{\frac{4n+2-6i}{n(n-1)}},\quad  1\le i\le n.
$$

\medskip

If
$$
\sum_{i=1}^{n}y_{i}=1\text{\; and\;\;}\sum_{i=1}^{n}iy_{i}=0,
\text{\; then\;\;}\sum_{i=1}^{n}(y_{i}-x_{i0})x_{i0}=0,
$$
so
\begin{eqnarray*}
\sum_{i=1}^{n}y_{i}^{2}&=&\sum_{i=1}^{n}(y_{i}-x_{i0}+x_{i0})^{2}
+\sum_{i=1}^{n}(y_{i}-x_{i0})^{2}+
2\sum_{i=1}^{n}(y_{i}-x_{i0})x_{i0}
+\sum_{i=1}^{n}x_{i0}^{2}\\
&=&
\sum_{i=1}^{n}(y_{i}-x_{i0})^{2}+
\sum_{i=1}^{n}x_{i0}^{2}>\sum_{i=1}^{n}x_{i0}^{2}
\end{eqnarray*}
if $y_{i}\ne x_{i0}$ for some $i\in\{1,2,\dots,n\}$.

\bigskip
{\bf \ref{exer:48}.}
$L=f({\bf X})-
\lambda (x_{1}+x_{2}+\cdots+x_{n})$

$$
L_{x_{i}}=-\frac{p_{i}f({\bf X})}{s-x_{i}}- \lambda,\text{\; so\;\;}
\frac{s-x_{10}}{p_{1}}=
\frac{s-x_{20}}{p_{2}}=\cdots=
\frac{s-x_{n0}}{p_{n}}=_\text{ def}C.
$$
$x_{i0}=s-Cp_{i}$, $1\le i\le n$.\quad
Denote $P=p_{1}+p_{2}+\cdots +p_{n}$.

$$
x_{1}+x_{2}+\cdots+x_{n}=ns-C(p_{1}+p_{2}+\cdots+p_{n})=ns-CP=s.
$$
$$
\dst{C=\frac{(n-1)s}{P}};\quad
x_{i0}=\dst{\frac{[P-(n-1)]sp_{i}}{P}}.
$$
$$
f_\text{max}=C^{P}p_{1}^{p_{1}}p_{2}^{p_{2}}\cdots p_{n}^{p_{n}}=
\left[\frac{(n-1)s}{P}\right]^{P}p_{1}^{p_{1}}p_{2}^{p_{2}}\cdots
p_{n}^{p_{n}}
$$
\bigskip

{\bf \ref{exer:49}.}
$L({\bf X})=\dst{f({\bf X})-\lambda
\sum_{i=1}^{n}\frac{x_{i}}{\sigma_{i}}}$, \;
$L_{x_{i}}=\dst{\frac{p_{i}f({\bf X})}{x_{i}}-\frac{\lambda}{\sigma_{i}}}$,
so $\dst{\frac{x_{i0}}{\sigma_{i}}}=Cp_{i}$.
To satisfy the constraint, $C=(p_{1}+p_{2}\cdots+p_{n})^{-1}$, so
$$
x_{i0}=\dst{\frac{p_{i}\sigma_{i}S}{p_{1}+p_{2}+\cdots+p_{n}}}.
$$
and
$$
x_{10}^{p_{1}}x_{20}^{p_{2}}\cdots x_{n0}^{p_{n}}=
\left(\frac{S}{p_{1}+p_{2}+\cdots+ p_{n}}\right)^{p_{1}+p_{2}+\cdots+p_{n}}
(p_{1}\sigma_{1})^{p_{1}}
(p_{2}\sigma_{2})^{p_{2}} \cdots
(p_{n}\sigma_{n})^{p_{n}}
$$

\bigskip
{\bf \ref{exer:50}.}
$\dst{L=\sum_{i=1}^{n}\frac{x_{i}}{\sigma_{i}}-
\lambda
x_{1}^{p_{1}}x_{2}^{p_{2}}\cdots x_{n}^{p_{n}}}$,
$L_{x_{i}}=\dst{\frac{1}{\sigma_{i}}-\lambda\frac{p_{i}V}{x_{i}}}$,
$\dst{\frac{x_{i0}}{\sigma_{i}}}=Cp_{i}$,
where $C$ must be chosen to satisfy the constraints.

\medskip
$x_{i0}=C\sigma_{i}p_{i}$,
$x_{i0}^{p_{i}}=(C\sigma_{i}p_{i})^{p_{i}}$,
$V=(C\sigma_{1}p_{1})^{p_{1}}
(C\sigma_{2}p_{2})^{p_{2}}\cdots
(C\sigma_{n})^{p_{n}}$
$$
C^{p_{1}+p_{2}+\cdots+p_{n}}=\frac{V}{(\sigma_{1}p_{1})^{p_{1}}(\sigma_{2}p_{2})^{p_{2}}
\cdots (\sigma_{ n}p_{n})^{p_{n}}}
$$

$$
C=\left(\frac{V}{(\sigma_{1}p_{1})^{p_{1}}(\sigma_{2}p_{2})^{p_{2}}
\cdots (\sigma_{ n}p_{n})^{p_{n}}}\right)^{\frac{1}{p_{1}+p_{2}+\cdots+p_{n}}}
$$
$$
\frac{x_{i0}}{\sigma_{i}}=p_{i}
\left(\frac{V}{(\sigma_{1}p_{1})^{p_{1}}(\sigma_{2}p_{2})^{p_{2}}
\cdots (\sigma_{ n}p_{n})^{p_{n}}}\right)^{\frac{1}{p_{1}+p_{2}+\cdots+p_{n}}}
$$

$$
\sum_{i=1}^{n}\frac{x_{i0}}{\sigma_{i}}=(p_{1}+p_{2}+\cdots+p_{n})
\left(\frac{V}{(\sigma_{1}p_{1})^{p_{1}}(\sigma_{2}p_{2})^{p_{2}}
\cdots (\sigma_{ n}p_{n})^{p_{n}}}\right)^{\frac{1}{p_{1}+p_{2}+\cdots+p_{n}}}.
$$

\bigskip
{\bf \ref{exer:51}.}
\quad \quad \quad
$L=\dst{\frac{1}{2}\sum_{i=1}^{n}\frac{(x_{i}-c_{i})^{2}}{\alpha_{i}}}
-\lambda (a_{1}x_{1}+a_{2}x_{2}+\cdots+a_{n}x_{n})$

$$
L_{x_{i}}=\dst{\frac{x_{i}-c_{i}}{\alpha_{i}}}-\lambda a_{i},\quad
x_{i0}=c_{i}+\lambda a_{i}\alpha_{i}
$$
$$
\sum_{i=1}^{n}a_{i}x_{i0}=\sum_{i=1}^{n}a_{i}c_{i}+
\lambda\sum_{i=1}^{n}a_{i}^{2}\alpha_{i}=d,\quad
\lambda=\frac{d-\sum_{i=1}^{n}a_{i}c_{i}}{\sum_{i=1}^{n}a_{i}^{2}\alpha_{i}}
$$
$$
\sum_{i=1}^{n}\frac{(x_{i0}-c_{i})^{2}}{\alpha_{i}}=\lambda^{2}
\sum_{i=1}^{n}a_{i}^{2}\alpha_{i}=
\frac{(d-\sum_{i=1}^{n}a_{i}c_{i})^{2}}{\sum_{i=1}^{n}a_{i}^{2}\alpha_{i}}
$$

\bigskip
{\bf \ref{exer:52}.}
It suffices to extremize
$\dst{\sum_{i=1}^{n}a_{i}x_{i}}$  subject to
$\sum_{i=1}^{n}x_{i}^{2}=\sigma^{2}$ for arbitrary $\sigma>0$.
$$
L=\dst{\sum_{i=1}^{n}a_{i}x_{i}-\frac{\lambda}{2}\sum_{i=1}^{n}x_{i}^{2}},
\quad L_{y_{i}}=a_{i}-\lambda x_{i},\quad  a_{i}=\lambda x_{i0},
$$
$$
\sum_{i=1}^{n}a_{i}^{2}=\lambda^{2}\sum_{i=1}^{n}x_{i0}^{2}=\lambda^{2}\sigma^{2}
$$
$$
\sum_{i=1}^{n}a_{i}x_{i0}=\lambda
\sum_{i=1}^{n}x_{i0}^{2}=\lambda\sigma^{2}=(\lambda\sigma)\sigma =
\pm\left(\sum_{i=1}^{n}a_{i}^{2}\right)^{1/2}
\left(\sum_{i=1}^{n}x_{i0}^{2}\right)^{1/2}
$$

\bigskip
{\bf \ref{exer:53}.}
For every $\sigma>0$,
 $f({\bf X})=
x_{m})=x_{1}^{r_{1}}x_{2}^{r_{2}}\cdots x_{m}^{r_{m}}$ assumes a maximum
value on the closed set
$$
S_{\sigma}=\set{(x_{1},x_{2}, \dots, x_{m})}{x_{i}>0, \,1 \le i \le m,\,
r_{1}x_{1}+r_{2}x_{2}+\cdots+ r_{m}x_{m}=\sigma}.
$$
$$
L=x_{1}^{r_{1}}x_{2}^{r_{2}}\cdots x_{m}^{r_{m}}
-\lambda\sum_{i=1}^{m}r_{i}x_{i},\quad
L_{x_{i}}=r_{i}\left(\frac{x_{1}^{r_{1}}x_{2}^{r_{2}}\cdots
x_{m}^{r_{m}}}{x_{i}}-\lambda\right), \quad 1 \le i \le m.
$$
Therefore, the constrained extremum is attained  at
 $x_{1}=x_{2}=\cdots =x_{m}=\sigma/r$, and the value of the constrained
extremum is $(\sigma/r)^{r}$,   so
$$
\left(x_{1}^{r_{1}}x_{2}^{r_{2}}\cdots x_{m}^{r_{m}}\right)^{1/r}
 \le \frac{\sigma}{r}=\frac{r_{1}{x_{1}+r_{2}x_{2}+\cdots+r_{k}x_{k}}}{r}
$$
with equality if and only if $x_{1}=x_{2}=\cdots= x_{m}=\sigma/r$.


\bigskip
{\bf \ref{exer:54}.}
The statement is trivial if $\sigma_{i}=0$ for some $i$. If
$\sigma_{i}\ne0$, $1 \le i \le m$,
 then  Exercise~\ref{exer:53}
with $r_{i}=\dst{\frac{1}{p_{i}}}$ and
$x_{i}=\dst{\frac{|a_{ij}|^{p_{i}}}{\sigma_{i}}}$ implies that
$$
\frac{|a_{1j}||a_{2j}|\cdots|a_{mj}|}
{\sigma_{1}^{1/p_{1}}\sigma_{2}^{1/p_{2}}\cdots\sigma_{m}^{1/p_{m}}}
 \le \sum_{i=1}^{m} \frac{|a_{ij}|^{p_{i}}}{p_{i}\sigma_{i}}.
$$
Summing both sides from $j=1$ to $n$ yields the stated conclusion.

\bigskip
{\bf \ref{exer:55}.}
\quad \quad $\dst{L
=\frac{1}{2}\sum_{r=0}^{n}x_{r}^{2}-\sum_{s=0}^{m}
\lambda_{s}\sum_{r=0}^{n}x_{r}r^{s}}$,\quad
$L_{x_{r}}=x_{r}-\dst{\sum_{s=0}^{m}\lambda_{s}r^{s}}$
$$
x_{r0}=\sum_{s=0}^{m}\lambda_{s}r^{s},\quad 0\le r\le n.
$$
$$
\sum_{r=0}^{n}x_{r0}r^{s}=\sum_{r=0}^{n}\sum_{\ell=0}^{m}\lambda_{\ell}r^{\ell+s}
=\sum_{\ell=0}^{m}\lambda_{\ell}\sum_{r=0}^{n}r^{\ell+s}=
\sum_{\ell=0}^{m}\sigma_{s+\ell}\lambda_{\ell}=c_{s},
\quad 0\le s \le  m,
$$
so $(x_{10},x_{20},\dots,x_{n0})$ is a  critical point  of $L$.
To see that it is constrained minimum point of $Q$, suppose that
$(y_{0},y_{1},\dots,y_{n})$ also satisfies the constraints; thus,
$$
\sum_{r=0}^{n}y_{r}r^{s}=c_{s},\quad 0\le s \le  m.
$$
Then
$$
\sum_{r=0}^{n}(y_{r}-x_{r0})x_{r0}=\sum_{r=0}^{n}(y_{r}-x_{r0})
\sum_{s=0}^{m}\lambda_{s}r^{s}=\sum_{s=0}^{m}\lambda_{s}\sum_{r=0}^{n}
(y_{r}-x_{r0})r^{s}=0,
$$
so
\begin{eqnarray*}
\sum_{r=0}^{n}y_{r}^{2}&=&\sum_{r=0}^{n}(y_{r}-x_{r0}+x_{r0})^{2}
=\sum_{r=0}^{n}[(y_{r}-x_{r0})^{2}+2(y_{r}-x_{r0})x_{r0}
+x_{r0}^{2}]\\
&=&\sum_{r=0}^{n}[(y_{r}-x_{r0})^{2} +\sum_{r=0}^{n}x_{r0}^{2}>
\sum_{r=0}^{n}x_{r0}^{2}.
\end{eqnarray*}

\bigskip
{\bf \ref{exer:56}.}
Imposing the constraint  with $r=0$ and $P(x)=x^{s}$, $1\le s\le 2k$,
yields
the necessary condition
\begin{equation} \tag{A}
\sum_{i=-n}^{n}w_{i}i^{s}=
\begin{cases} 1& \text{if } s=0,\\ 0&\text{if  }1\le s\le 2k.  \end{cases}
\end{equation}
If $P$ is an arbitrary polynomial of degree $\le 2k$ and $r$ is an
arbitrary integer, then \\ $P(r-i)=P(r)+$ a linear combination of
$i$, $i^{2}$, \dots, $i^{2k}$, so (A) implies that
$$
\sum_{i=-n}^{n}w_{i}P(r-i)=P(r)
$$
whenever $r$  is an integer and $P$ is a polynomial of degree $\le 2k$.
Therefore,
$$
L=\frac{1}{2}\sum_{i=-n}^{n}w_{i}^{2}-\sum_{r=0}^{2k}\lambda_{r}
\sum_{i=-n}^{n}w_{i}i^{r},
$$
$$
L_{w_{i}}=w_{i}-\sum_{r=0}^{2k}\lambda_{r}i^{r},\quad
w_{i0}=\sum_{r=0}^{2k}\lambda_{r}i^{r}, \quad -n\le i\le n,
$$
and
$$
\sum_{i=-n}^{n}w_{i0}i^{s}= \sum_{i=-n}^{n}
\left(\sum_{r=0}^{2k} \lambda_{r}i^{r}\right)i^{s}
=\sum_{r=0}^{2k}\lambda_{r}\sigma_{r+s}\text{\; where\;\;}
\sigma_{m}=\sum_{i=-n}^{n}i^{m}.
$$

\medskip

If $\{w_{i}\}_{i=-n}^{n}$ also satisfies the constraint,
then
$$
\sum_{i=-n}^{n}(w_{i}-w_{i0})w_{i0}=
\sum_{i=-n}^{n}(w_{i}-w_{i0})\sum_{r=0}^{2k}\lambda_{r}i^{r}=0.
$$
Therefore,
\begin{eqnarray*}
\sum_{i=-n}^{n}w_{i}^{2}&=&\sum_{i=-n}^{n}(w_{i0}+w_{i}-w_{i0})^{2}=
\sum_{i=-n}^{n}\left(w_{i0}^{2}+2(w_{i}-w_{i0})w_{i0}+(w_{i}-w_{i0})^{2}\right)\\
&=&\sum_{i=-n}^{n}w_{i0}^{2}+\sum_{i=-n}^{n}(w_{i}-w_{i0})^{2}
>\sum_{i=-n}^{n}w_{i0}^{2}
\end{eqnarray*}
if $w_{i}\ne w_{i0}$ for some $i$.

\medskip
{\bf \ref{exer:57}.}
The coefficients $w_{0}$, $w_{1}$, \dots, $w_{k}$ satisfy the constraint
if and only if
$$
\sum_{i=0}^{n}w_{i}(r-i)^{j}=(r+1)^{j},\quad 0\le j\le k,
$$
for all integers $r$. This is equivalent to
$$
\sum_{i=0}^{n}w_{i}\sum_{s=0}^{j}(-1)^{s}
\binom{j}{s}s^{j}r^{j-s}
=\sum_{s=0}^{j}\binom{j}{s}r^{j-s},\quad 0\le j\le k,
$$
which is equivalent to
\begin{equation} \tag{A}
 \sum_{i=0}^{n}w_{i}i^{s}=(-1)^{s},\quad  0\le s\le k.
\end{equation}
$$
L=\frac{1}{2}\sum_{i=0}^{k}w_{i}^{2}-\sum_{r=0}^{k}\lambda_{r}
\sum_{i=0}^{k}w_{i}i^{r};\quad
L_{x_{i}}=w_{i}-\sum_{r=0}^{k}\lambda_{r} i^{r};\quad
w_{i0}=\sum_{r=0}^{k}\lambda_{r}i^{r}.
$$
Now
 must choose $\lambda_{1}$, $\lambda_{2}$, \dots, $\lambda_{k}$ to
satisfy (A).

$$
\sum_{i=0}^{n}w_{i0}i^{s}\sum_{r=0}^{k}\lambda_{r}i^{r}
=\sum_{r=0}^{k}\lambda_{r}\sum_{i=0}^{n}i^{r+s}
= \sum_{r=0}^{k}\sigma_{r+s}\lambda_{r}=(-1)^{s}, \quad 0\le s\le k.
$$

If $\{w_{i}\}_{i=0}^{n}$ also satisfies the constraint,
then
$$
\sum_{i=0}^{n}(w_{i}-w_{i0})w_{i0}=
\sum_{i=n}^{n}(w_{i}-w_{i0})\sum_{r=0}^{2k}\lambda_{r}i^{r}=0.
$$
Therefore,
\begin{eqnarray*}
\sum_{i=0}^{n}w_{i}^{2}&=&\sum_{i=0}^{n}(w_{i0}+w_{i}-w_{i0})^{2}=
\sum_{i=0}^{n}\left(w_{i0}^{2}+2(w_{i}-w_{i0})w_{i0}+(w_{i}-w_{i0})^{2}\right)\\
&=&\sum_{i=0}^{n}w_{i0}^{2}+\sum_{i=-n}^{n}(w_{i}-w_{i0})^{2}
>\sum_{i=0}^{n}w_{i0}^{2}
\end{eqnarray*}
if $w_{i}\ne w_{i0}$ for some $i$.

\bigskip
{\bf \ref{exer:58}.}
\quad \quad \quad \quad \quad \quad
$L=\dst{\frac{1}{2}\sum_{i=1}^{n}\frac{(x_{i}-c_{i})^{2}}{\alpha_{i}}
-\sum_{s=1}^{m}\lambda_{s}\sum_{i=1}^{n}a_{is}x_{i}}$

$$
L_{x_{i}}=\frac{x_{i}-c_{i}}{\alpha_{i}}, \quad
x_{i0}=c_{i}+\alpha_{i}\dst{\sum_{s=1}^{m}\lambda_{s}a_{is}}
$$

\medskip
$$
\dst{\sum_{i=1}^{n}a_{ir}x_{i0}}=\dst{\sum_{i=1}^{n}a_{ir}c_{i}+
\sum_{s=1}^{m}\lambda_{s}\sum_{i=1}^{n}\alpha_{i}a_{ir}a_{is}
= \sum_{i=1}^{n}a_{ir}c_{i}+\lambda_{r} =d_{r}}
$$

$$
\lambda_{r}=d_{r}-\dst{\sum_{i=1}^{n}a_{ir} c_{i}},\quad
\dst{\frac{(x_{i}-c_{i})^{2}}{\alpha_{i}}=\alpha_{i}\sum_{r,s=1}^{m}
\lambda_{r}\lambda_{s}a_{ir}a_{is}}
$$

$$
\sum_{i=1}^{n}\frac{(x_{i}-c_{i})^{2}}{\alpha_{i}}=\sum_{r,s=1}^{m}
\lambda_{r}\lambda_{s}\sum_{i=1}^{n}\alpha_{i}a_{ir}a_{is}
=\sum_{r=1}^{m}\lambda_{r}^{2} =\sum_{r=1}^{m}
\left(d_{r}-\sum_{i=1}^{n}a_{ir}c_{i}\right)^{2}
$$

\end{document}
\place %3903
